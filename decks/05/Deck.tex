\documentclass[11pt,ignorenonframetext,]{beamer}
\usetheme{Szeged}
\usecolortheme{crane}
\usefonttheme{structurebold}
\usepackage{amssymb,amsmath}
\usepackage{ifxetex,ifluatex}
\usepackage{fixltx2e} % provides \textsubscript
\ifxetex
  \usepackage{fontspec,xltxtra,xunicode}
  \defaultfontfeatures{Mapping=tex-text,Scale=MatchLowercase}
\else
  \ifluatex
    \usepackage{fontspec}
    \defaultfontfeatures{Mapping=tex-text,Scale=MatchLowercase}
  \else
    \usepackage[utf8]{inputenc}
  \fi
\fi
\usepackage{listings}

% Comment these out if you don't want a slide with just the
% part/section/subsection/subsubsection title:
% \AtBeginPart{
%   \let\insertpartnumber\relax
%   \let\partname\relax
%   \frame{\partpage}
% }
% \AtBeginSection{
%   \let\insertsectionnumber\relax
%   \let\sectionname\relax
%   \frame{\sectionpage}
% }
% \AtBeginSubsection{
%   \let\insertsubsectionnumber\relax
%   \let\subsectionname\relax
%   \frame{\subsectionpage}
% }

\setlength{\parindent}{0pt}
\setlength{\parskip}{6pt plus 2pt minus 1pt}
\setlength{\emergencystretch}{3em}  % prevent overfull lines
\setcounter{secnumdepth}{0}

%%% begin dwr insert
\usepackage{patchcmd}
\usepackage{tabulary}   % Support longer table cells
\usepackage{booktabs}   % Support better tables
\usepackage[sort&compress]{natbib}

\usepackage{framed}     % Allow background color for images
\definecolor{shadecolor}{named}{white}

%\usepackage{paralist}
\usepackage{xparse}

\usepackage{subfigure}
\usepackage{hyperref}
%%% end dwr insert
\NewDocumentCommand\Norm{m}{\lVert #1 \rVert}
\NewDocumentCommand{\vv}{}{\vec{v}}
\NewDocumentCommand{\ww}{}{\vec{w}}
\NewDocumentCommand{\uu}{}{\vec{u}}
\NewDocumentCommand{\i}{}{\hat{\imath}}
\NewDocumentCommand{\j}{}{\hat{\jmath}}
\NewDocumentCommand{\k}{}{\hat{k}}

\title{Dot and cross product}
\author{Math 251 Calculus 3}
\date{September 18, 2013 }

\usepackage{siunitx}

\begin{document}
\frame{\titlepage}

\section{Dot and cross product}

\begin{frame}\frametitle{Warm-up, I}

\begin{itemize}[<+->]
\itemsep1pt\parskip0pt\parsep0pt
\item
  Ready?
\item
  OK FAST what is the dot product--cosine formula?
\item
  $\vec{v} \cdot \vec{w} = \Norm{\vec{v}} \Norm{\vec{w}} \cos{\theta}$
\item
  Here, we agree to always choose $0 \leq \theta \leq \pi$.
\item
  Suppose $\Norm{\vec{v}} = 3$ and $\Norm{\vec{w}} = 2$. What are the
  maximal and minimal possible values of $\vec{v} \cdot \vec{w}$?
\end{itemize}

\end{frame}

\begin{frame}\frametitle{Warm-up, II}

\begin{itemize}[<+->]
\item
  Suppose $\vec{v} \cdot \vec{w} = \Norm{\vec{v}} \Norm{\vec{w}}$. What
  can you conclude about the orientation of the vectors? It helps to
  picture them in standard position.
\item
  Suppose $\vec{v} \cdots \vec{w} = -\Norm{\vec{v}} \Norm{\vec{w}}$. Now
  what can you conclude?
\item
  Suppose
\end{itemize}

\end{frame}

\end{document}
