\documentclass[11pt,ignorenonframetext,aspectratio=169,xcolor={svgnames}]{beamer}
\usetheme{Szeged}
\usecolortheme{crane}
\usefonttheme{structurebold}
\usefonttheme{professionalfonts}
\usepackage{amssymb,amsmath}
\usepackage{ifxetex,ifluatex}
\usepackage{fixltx2e} % provides \textsubscript
\ifxetex
  \usepackage{fontspec,xltxtra,xunicode}
  \defaultfontfeatures{Mapping=tex-text,Scale=MatchLowercase}
\else
  \ifluatex
    \usepackage{fontspec}
    \defaultfontfeatures{Mapping=tex-text,Scale=MatchLowercase}
  \else
    \usepackage[utf8]{inputenc}
  \fi
\fi
\usepackage{listings}

% Comment these out if you don't want a slide with just the
% part/section/subsection/subsubsection title:
% \AtBeginPart{
%   \let\insertpartnumber\relax
%   \let\partname\relax
%   \frame{\partpage}
% }
% \AtBeginSection{
%   \let\insertsectionnumber\relax
%   \let\sectionname\relax
%   \frame{\sectionpage}
% }
% \AtBeginSubsection{
%   \let\insertsubsectionnumber\relax
%   \let\subsectionname\relax
%   \frame{\subsectionpage}
% }

\setlength{\parindent}{0pt}
\setlength{\parskip}{6pt plus 2pt minus 1pt}
\setlength{\emergencystretch}{3em}  % prevent overfull lines
\setcounter{secnumdepth}{0}

%%% begin dwr insert
\usepackage{patchcmd}
\usepackage{tabulary}   % Support longer table cells
\usepackage{booktabs}   % Support better tables
\usepackage[sort&compress]{natbib}

\usepackage{framed}     % Allow background color for images
\definecolor{shadecolor}{named}{white}

%\usepackage{paralist}
\usepackage{xparse}
\usepackage{subfigure}
\usepackage{hyperref}
%%% end dwr insert
\usepackage{rosoff-vector-macros}
\usepackage[beamer]{cofi}
\title{Higher derivatives and modeling}
\author{Math 275 Multivariable Calculus}
\date{October 9, 2013}


\begin{document}
\frame{\titlepage}

\section{Higher derivatives and modeling}

\begin{frame}\frametitle{Recap: slices, partial derivatives, and tangent
lines}

We saw previously that if $f(x,y)$ is a function of two variables, each
of its partial derivatives $f_x(x,y)$ and $f_y(x,y)$ gives the slopes of
tangent lines to slice curves.

If $(a, b)$ is a point in the plane, the slice curves through $(a,b)$
are the graphs of $z = f(a, y)$ (in the plane $x = a$) and $z = f(x, b)$
(in the plane $y = b$).

\end{frame}

\begin{frame}\frametitle{Partial derivatives}

If the slice functions $f(a, y)$ and $f(x,b)$ are differentiable (in the
one-variable sense), their derivatives are $f_y(a,y)$ and $f_x(x,b)$.
These functions are ordinary derivatives, so they compute tangent slopes
in the usual way.

\begin{itemize}

\item
  The tangent line equations:
\item
  $z = f(a, b) + f_x(a,b)(x - a)$
\item
  $z = f(a, b) + f_y(a,b)(y - b)$
\end{itemize}

This is the old tangent line approximation formula, just twice in
different directions.

\end{frame}

\begin{frame}\frametitle{Higher derivatives}

The pure partials $f_{xx}$ and $f_{yy}$ are easy to understand in terms
of the slices. They tell us about concavity, just like ordinary second
derivatives. Namely,

\begin{itemize}

\item
  $f_{xx}(x,b)$ is positive (negative) if $f(x,b)$ is concave up
  (concave down)
\item
  $f_{yy}(a,y)$ is positive (negative) if $f(a,y)$ is concave up
  (concave down)
\end{itemize}

\end{frame}

\begin{frame}\frametitle{The mixed partial}

Recall that a smooth function has just one mixed partial, because
$f_{xy} = f_{yx}$ by Clairaut's theorem.

This partial is harder to visualize in terms of graphs. The activity
(\href{../../workshops/07/Workshop.pdf}{Workshop 07}) should illuminate
its meaning.

\end{frame}

\begin{frame}\frametitle{Curve shapes via derivatives}

Recall how graph-sketching works in ordinary calculus: one identifies
the locations where the first derivative changes sign (these are the
local maxima and minima) and where the second derivative changes sign
(these are the points of inflection). There are four possible ``curve
shapes'':

\begin{enumerate}
\def\labelenumi{\arabic{enumi}.}

\item
  Increasing, concave up
\item
  Increasing, concave down
\item
  Decreasing, concave up
\item
  Decreasing, concave down
\end{enumerate}

Once you know which shape occurs on which interval, just paste them
together.

\end{frame}

\begin{frame}\frametitle{Universality of quadratic curves}

Smooth functions are very well approximated \emph{locally} by their
quadratic approximations (second-order Taylor series). We'll use ideas
about quadric surfaces (corresponding to degree 2 polynomials in 2
variables) to do some modeling of more interesting multidimensional
phenomena.

\end{frame}

\end{document}
