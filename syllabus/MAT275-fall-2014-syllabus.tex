\documentclass[symmetric]{tufte-handout}
\usepackage{fontspec}
\usepackage[beamer]{rosoff}
\usepackage{xcolor,epigraph,nicefrac,booktabs}
\usepackage{dcolumn}
%\usepackage[top=1.0in,bottom=1.0in,left=1.2in,right=1.2in]{geometry}
\frenchspacing
\renewcommand\allcapsspacing[1]{{\addfontfeature{LetterSpace=15}#1}}
\renewcommand\smallcapsspacing[1]{{\addfontfeature{LetterSpace=10}#1}}
\NewDocumentCommand{\webwork}{}{\textsf{WeBWorK}}
\NewDocumentCommand{\officehours}{}{MW~9:10--10:10; T~10:20--11:20; F~2:10--3:10}
\linespread{1.05}
\setlength{\epigraphwidth}{0.7\linewidth}
\renewcommand{\epigraphsize}{\normalsize}
\setlength{\epigraphrule}{0pt}
\setlength{\beforeepigraphskip}{0\baselineskip}
\setlength{\afterepigraphskip}{0\baselineskip}
\setmainfont[Mapping={tex-text},
    Ligatures={Common},
    Numbers={OldStyle}]{Palatino Linotype}
\setsansfont[Mapping={tex-text},
    Ligatures={Common}]{Museo Sans 500}
\setmonofont[Mapping={tex-text},
    Ligatures={Common}]{Inconsolata}
\RenewDocumentCommand{\textflush}{}{flushright}

\NewDocumentCommand{\webworkpct}{}{0.08}
\NewDocumentCommand{\workshoppct}{}{0.10}
\NewDocumentCommand{\portfoliopct}{}{0.10}
\NewDocumentCommand{\presentationpct}{}{0.08}
\NewDocumentCommand{\quizpct}{}{0.32}
\NewDocumentCommand{\midtermonepct}{}{0.08}
\NewDocumentCommand{\midtermtwopct}{}{0.08}
\NewDocumentCommand{\finalexampct}{}{0.16}

\title{Mathematics 275 Multivariable Calculus}
\date{Fall 2014}
\author{The College of Idaho}
\begin{document}
\maketitle
\begin{fullwidth}
\epigraph{%
    The pursuit of knowledge, brother, is the askin' of many questions.%
}{R\smallcaps{aymond} C\smallcaps{handler} \\ \emph{Farewell, My Lovely}}
\end{fullwidth}

\subsection{Quick Reference} \label{ssec:quickreference}
    \begin{minipage}[t]{0.40\linewidth}
        \vspace{0pt}
        You have both online \webwork\ problems and written portfolio
        problems. See below for more details.
        You are free to work together on problems, but all work
        you submit must be your own. Portfolios will be collected
        at least four times. We have frequent quizzes and in-class
        activities that cannot be made up.
    \end{minipage}\marginnote{% Office hours and contact info
        \begin{description}
            \item[Instructor]
                Dr.\ Dave Rosoff
            \item[Office]
                Boone Hall 102C
            \item[Office hours]
                MW~9:10--10:10, T~10:20--11:20, F~2:10--3:10, or by 
                appointment
            \item[Email]
                \href{mailto:drosoff@collegeofidaho.edu}{drosoff@collegeofidaho.edu}
            \item[Twitter]
                \texttt{@daverosoff}
            \item[WeBWorK]
                \url{https://webwork.collegeofidaho.edu/webwork2/MAT275_F14}
            \item[Other web resources]
                \href{https://learn.collegeofidaho.edu/}{Moodle}
        \end{description}
        } \hspace{1em}
    \begin{minipage}[t]{0.44\linewidth}
        \vspace{0pt}
        \centering\allcaps{Grading}\vspace*{1.0ex}
        \begin{tabular}{lcr}
            \toprule
            Tier & Weight & Date \\
            \midrule
            \webwork      & \webworkpct      & continual    \\
            Workshops     & \workshoppct     & continual    \\
            Portfolios    & \portfoliopct    & continual    \\
            Presentations & \presentationpct & continual    \\  
            Quizzes       & \quizpct         & 1.5/week     \\
            Midterm 1     & \midtermonepct   & September 30 \\
            Midterm 2     & \midtermtwopct   & November 11  \\
            Final Exam    & \finalexampct    & December 10  \\
            \bottomrule
        \end{tabular}
    \end{minipage} \hspace*{1em}

\subsection{Preface: Learning outcomes}

    This course is designed to provide certain experiences, called “learning
    outcomes”, to students who successfully complete it. These outcomes are
    enumerated in the margin.%
    \sidenote[][-14ex]{\smallcaps{Learning outcomes:}
        \begin{compactenum}
            \item \label{lo:ideas} Recognize and describe fundamental ideas from course content described below.
            \item \label{lo:examp} Illustrate these ideas with examples and translate them into everyday terminology.
            \item \label{lo:demo} Demonstrate the use of calculus on specific numerical problems.
            \item \label{lo:class} Classify the techniques of calculus according to their use.
            \item \label{lo:write} Improve their mathematical writing skills.
            \item \label{lo:present} Develop their communication skills by presenting mathematical material.
            \item \label{lo:discuss} Effectively discuss and solve mathematical problems in group settings.
            \item \label{lo:infer} Infer connections made in arguments without specific direction.
            \item \label{lo:valid} Explain why arguments making use of calculus are sound or not, as appropriate.
            \item \label{lo:synth} Plan, organize, and combine arguments to solve new problems, as appropriate.
            \item \label{lo:generalize} Generalize or modify to create new arguments, as appropriate.
        \end{compactenum}
    }
    I explicitly include these outcomes in the syllabus so that it is clear
    why I have chosen the various course components (each of which is
    described below.) Each learning outcome is addressed by one or more
    components of the course: quizzes, \webwork\ exercises, workshops, portfolios,
    presentations to the class, and exams.
    See the \emph{Grading} section below for more information.

\subsection{Introduction}

    Welcome to MAT-275, Multivariable Calculus. I am very pleased to be
    teaching the course once more. It is the first college mathematics course
    that is designed specifically for people intending to study higher mathematics, 
    and as such
    will form the foundation of your continued work in the mathematical and
    physical sciences. It is more modern than your previous courses in
    calculus. The machinery and terminology of \emph{vectors}, in particular,
    is quite young, dating back to the work of Gibbs and Hamilton in the late
    19th century---less than 150 years ago, as compared to five centuries or
    more for the foundational results of single-variable
    calculus.\sidenote{
        Newton and Leibniz tend to get all the credit for
        inventing calculus in the late 1600s, but some key notions go all the
        way back to Archimedes, who lived about 23 centuries ago.
    }

    The name of the course is traditional, and at one time all colleges and
    universities throughout this country had very similar offerings called
    ``Calculus 3'', but matters have changed in recent years. Quick and reliable
    hand calculation was once a skill of paramount importance, but is less so
    today. It is more important that you have a good working understanding of the
    fundamental concepts of calculus. This understanding will lead to asking the
    correct questions of software, literature searches, and your eventual
    colleagues.

    \newthought{One fundamental idea and theme} of the course is what is called
    ``parametrization''.\footnote{Briefly, to parametrize a collection of
    objects is to put them in correspondence with some kind of numbers. When
    we list items, or rank preferences, we are parametrizing them with the
    positive integers (the collection that \emph{indexes} our list). Many
    different kinds of numbers can be used to parametrize collections, such as real
    or complex numbers, vectors, matrices, tensors, and more.} A main goal of
    the course is for you to have an enduring understanding of this idea that
    permits you to recognize, employ, and interpret parametrization in your
    future mathematical enterprises, and understand how it binds the other
    fundamental concepts of calculus and geometry together. 

    We generalize the main ideas and results of single-variable calculus to
    multivariate situations, with particular attention to applications and
    modeling. There are two ways to proceed. First, we allow the domains (inputs)
    of the functions at hand to be 2-, 3-, or higher-dimensional, while
    maintaining the restriction that the function's values are real numbers in the
    usual sense. Here we will meet the essential concepts of \emph{partial
    derivative} and \emph{total} or \emph{Jacobian derivative} and revisit the
    familiar themes of differential calculus (rates of change, chain rule, optimization, etc.).
    We will introduce the powerful modern algebraic framework of \emph{vectors} in
    which the rest of the theory is cast. At this level of abstraction there are
    many opportunities to discuss mathematical modeling, of which we avail
    ourselves as appropriate.

    Just as functions of several variables may be differentiated 
    via their partial derivatives, so too can they be integrated over
    appropriate \emph{regions} (other than intervals) in their domains. We
    study the important change-of-variable  theorem that allows integrals over
    complicated regions to be reckoned in terms of simpler ones (e.g.,
    rectangles) and examine some applications.

    \newthought{The dual part of our investigation} is to let the codomains of our functions be
    higher-dimensional Euclidean spaces while keeping the domains 1-dimensional.
    This approach entails an investigation into fundamental ideas like
    \emph{parametrization}.

    We conclude with a study of the important field of \emph{differential
    geometry}. The laws of electromagnetism, Maxwell's equations, are formulated
    in this language, as are the laws of aerodynamics. We will make as much
    progress as we can toward the fundamental theorems of G.\ Green, C.\ F.\
    Gauss, and G.\ Stokes\footnote{Famously, Stokes's theorem is due not to
    Stokes, but to Lord Kelvin; Stokes merely \emph{assigned} it, in 1854.}. All
    three are vast generalizations of the usual Fundamental Theorem of Calculus,
    and all have extremely important implications for physics, engineering, and
    the rest of mathematics.

\subsection{Catalog description}

    ``This course is an extension of calculus to higher-dimensional spaces.
    Main topics include differentiation of functions of two and three
    variables, an introduction to vector analysis and parameterization, and
    a study of definite integration in both rectangular and curved
    coordinate systems. Topics may include a review of functions of several
    variables, vector geometry of 3-dimensional space, partial derivatives,
    gradient vectors, optimization techniques, multiple integration in the
    three classical curvilinear coordinate systems, parametric equations,
    vector fields, line integrals and Green's Theorem, and the other
    classical integral theorems of differential geometry.''

\subsection{Text}

    The text is \emph{Calculus: Early Transcendentals} by Jon Rogawski, second
    edition. \sidenote[][-6ex]{
        It is OK if you have the first edition, although your section numbers
        may be different. 

        If you choose to use a different edition than the
        recommended one, it is your responsibility to make sure the problem numbers
        are correct (important for presentations/portfolio problems, see below).
    }

\subsection{Grading}

    Scores are computed as a weighted average, with the following weights:
    \webwork\ \webworkpct, presentations \presentationpct, portfolios
    \portfoliopct, workshops \workshoppct, quizzes \quizpct, two in-class
    exams \midtermonepct\ and \midtermtwopct, and final exam \finalexampct.
    Observe that the weights sum to 1 = 100\%.\sidenote{  
        Observe that the exams are relatively lightly weighted, at 32\% in 
        total, the same as the quizzes! It is my hope that this reduces the
        incentives and payoffs to cram and realigns them in the appropriate
        direction, supporting steady incremental effort.
    } The exact determination of letter grades
    from these scores depends on the final distribution of scores in the
    class, but you can expect a C for earning 75\% of the points, a C+ for
    78\%, a B-- for 80\%, and so on.

\subsection{Workshops} % (fold)
\label{sub:workshops}

    Experience has taught me that this course functions much better as an
    interactive course than as a pure lecture. I will use what I call
    \emph{workshops} to introduce many new ideas and guide you through 
    examples.\sidenote{
        Workshops address learning outcomes~\ref{lo:ideas}, \ref{lo:examp},
        \ref{lo:demo}, and \ref{lo:discuss}, and sometimes others.
    } Instead of doing these at home, we will use class time to
    work on them. Occasionally you may need to finish them outside of class,
    if we don't have enough time to finish. They have proved to be very
    valuable in helping students get main ideas before tackling \webwork\ or
    portfolio problems. The workshops and the homework are the heart of the
    course.\sidenote{
        When workshops are graded, they are usually graded for completion. You
        will have plenty of time in class to make sure you understand properly.
    }

% subsection workshops (end)

\subsection{Homework}

    Homework in this class comprises both online and traditional written assignments.
    \begin{compactenum}
        \item \webwork\sidenote{\webwork\ problems address learning 
        outcomes~\ref{lo:ideas}, \ref{lo:examp}, \ref{lo:demo}, and \ref{lo:class}.} 
        assigned almost-daily, available at
        \url{https://webwork.collegeofidaho.edu/webwork2/MAT275_F14/}
        \item \emph{Portfolio problems}, assigned in weekly batches and collected 
        regularly throughout the term.\sidenote{Portfolio problems address learning
        outcomes~\ref{lo:write}, \ref{lo:infer}, \ref{lo:valid}, and 
        \ref{lo:generalize}.} These problems mostly come from the textbook 
        and are intended to stretch your thinking beyond the level of the online 
        homework. It is these problems that you have opportunities to present at 
        the board. Note that you may record classmates' solutions to these problems 
        as they present. Your portfolio problems should be stored and submitted in
        a three-ring binder whose contents are well organized. For more details on
        the style and format of the portfolio, see the section \emph{How to do 
        homework}.
    \end{compactenum}

\subsection{Quizzes}

    To make sure each student is keeping up with the online problems, we will
    have frequent quizzes.\sidenote{Quizzes address learning
    outcomes~\ref{lo:ideas}, \ref{lo:demo}, \ref{lo:class}, and
    \ref{lo:valid}.} Quiz problems are simple  variants of \webwork\ problems
    and all quizzes are open-note (but not open-book).\sidenote{
        I would advise you to write up your \webwork\ solutions for reference
        during quizzes. The more thorough your write-up, the more it will
        help you.
    } Therefore, if you have
    solved the \webwork\ problems prior to the quiz day, a score of 100\% is
    very possible. Note that quizzes cannot be made up, so it is very
    important that you come to class each day.  Quizzes begin promptly at
    8:00.

\subsection{Presentations}

    Each student should attempt to earn 60 \emph{presentation points} during
    the semester. Presentations of portfolio problems are worth up to 10
    points each. A presentation consists of two or three active presenters who
    provide an explanation, a solution, and any justification of the solution
    requested by the class.\sidenote{Presentations address learning
    outcomes~\ref{lo:present},  \ref{lo:discuss}, \ref{lo:valid},
    \ref{lo:synth}, and \ref{lo:generalize}.} I will maintain a list of
    problems and their presentation metadata online for your reference.

\subsection{Exams}

    Two exams are given in class.\sidenote{
        Dates for Exams 1 and 2 are tentative.
        \begin{compactdesc}
            \item[Exam 1] Tuesday, September 30
            \item[Exam 2] Tuesday, November 11
            \item[Final Exam:] Wednesday, December 10, 8:30--11:30 am
        \end{compactdesc}
    } Exams comprise both
    portfolio problems and new problems intended to stretch your thinking. During
    the exam, you may refer freely to your portfolio. A missed exam results in an
    exam grade of zero. Arrangements for absences, again, must be made well in
    advance (two weeks suffices). \emph{If arrangements are not made in advance, I
    will consider make-ups only with compelling, documented reasons.} The final
    exam takes place at the indicated date and time. It cannot be rescheduled
    \emph{for any reason}. Make your travel plans accordingly.

    

\subsection{How to do homework}

Whenever you are writing a solution to a math problem, it is important to
strive for the clearest exposition you can manage. Good mathematical writing
is essential for anyone who wishes to think clearly about mathematics. The
process of making your ideas and reasoning \emph{clear, complete, and
unambiguously correct} is the most effective amplifier of mathematical power
there is. Hence your solutions should be composed in brilliant English prose.
This means employing accepted scientific usage, more or less correct grammar
and spelling, and above all \emph{complete sentences}---sprinkled here and
there with tangy, delicious equations. Solutions in the popular 
``pile-of-equations'' style with little or no explanatory text will not get much credit.
You must explain what is happening as the action unfolds.

The reason for all this is that the process of such writing and editing will
implant understanding more firmly in your mind. Similar problems appear on
exams. If you dash off a quick and dirty solution it is less likely you will
recall what you did at the appropriate time.

As mentioned above, portfolio problems should be stored and submitted in a
clean and well organized three-ring binder. About four times during the
semester, I will collect and assess your work. While I provide extensive
comments on your work, only two grades are possible on a portfolio problem: 0
points and 10 points.\sidenote[][-10ex]{
    Do not be dismayed if your early attempts are graded at 0 points. 
    Most people do not know how to write mathematically and it is not the kind of
    writing one learns to do in other classes. That is why I have this 
    ``revise-and-resubmit'' approach: it allows you to focus on writing once you
    have solved the problem, or to reëvaluate your solution and try again if
    it is incomplete.
} Because of this, portfolio problems may be submitted
more than once and scores are not final until the last day of class. \emph{No
portfolios will be accepted after the last day of class without a compelling,
documented reason.} Your write-ups should be clear, complete, and free of
typographical errors.\sidenote{
    If you are interested in typing your solutions, I encourage you to do so
    using the \href{https://www.writelatex.com/signup?ref=18d26a43be8f}{Write\LaTeX\ 
    service}. See Moodle for more information and a sign-up link, or ask me 
    about it any time.
} I am happy to talk about these problems in office hours
with you. 

\subsection{A note on studying math}

By now you have studied enough mathematics to have learned something about how
it is that the material passes through your shapely skull and into your soft,
wrinkly brain. Nevertheless, you may find that this course is rather more
difficult than your previous calculus courses. To really understand it, we
will have to dig into subtle distinctions and nuances that no one has asked
you to think about before. In the one-dimensional world, these nuances simply
do not signify, so they are novel by necessity at this time. It is also more
difficult to picture what is going on mentally, again owing to the presence of
extra dimensions. Part of what I'm here to tell you is that while the material
may seem wholly new and unfamiliar, the underlying principles of calculus
(what do derivatives \emph{do}? what are integrals \emph{for}?) are immutable.

\subsection{Academic integrity}

I encourage all students to form study groups and collaborate on homework;
each student is of course individually responsible for their own work.
Collaborators must be acknowledged. 

Students are expected to complete all graded work in accordance with the
College Honor Code. Plagiarism, cheating, or borrowing without proper credit
will not be tolerated.  Violations of academic honesty can result in loss of
credit on an assignment, failure on an exam, or failure in the course.
Referrals may be made to the Vice President for Academic Affairs for any party
involved in academic dishonesty.

\subsection{Special accommodations}

Students who have documented disabilities as addressed by the Americans With
Disabilities Act and who need any test or course materials to be furnished in
an alternative format should notify me immediately (during the first week of
class).  Reasonable efforts will be made to accommodate the needs of such
students.

\vspace*{\stretch{1}}

\begin{fullwidth}
    \centering {\Huge \allcaps{Good luck this semester!}}
\end{fullwidth}

\vspace*{\stretch{1}}

\end{document}
