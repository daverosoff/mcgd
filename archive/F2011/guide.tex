\begin{document}

This document is intended to bridge the gap between what professors expect in college mathematical writing and what students entering college have been trained to do (or not do) in their math homework. It is a free-form list of hints, techniques, and self-directed inquiries flowing solely from my (Prof.\ Rosoff's) personal experience as a learner, doer, and teacher of math. No claims are made as to its correctitude or to its exhaustivity.

\section{Pile-of-equations}
Most people learn through repetitive experience in high school that math homework is different from other kinds of homework. All homework consists more or less of answers to questions. Yet the answers vary substantively, so we learn, depending on the discipline from which the homework emerges. History homework, say, or English homework, or even geology homework, should consist largely of paragraphs that themselves consist of sentences. These sentences are constructed from words according to a more or less arcane and mysterious set of rules called ``English grammar''. Regardless of a specific person's ability or willingness to follow these grammatical rules, she likely knows they exist and that homework is to be completed in the ways they delineate.

Math is different. The sentences (we learn) of math are equations, its words algebraic symbols: $3$, $+$, $=$, progressing on eventually to $x$, $\leq$, $\sin$, various sub- and superscripts, and (for the few elect), eventually, $d$, $\int$, and the rest of the lot. A common (I'd venture, universal) complaint of mathematics instructors at the college level is that the students incline overwhelmingly toward egregious abuse of the equals sign. Students, however, are perhaps not entirely at fault for such orthographic transgressions as (undoubtedly familiar to all teachers of calculus)
\[
f(x) = x^3 + 3x^2 + 3x + 1 = 3x^2 + 6x + 3 = f'(x)
\]
in which the equals sign is used to mean something like ``and then I do this''. It is, to be sure, vexing to be confronted with such a chimera; yet how else is the student, who has seen no other mathematical verb but ``='', to relate his calculation? Knowing the inherent reasonability of the student's position still gives the frustrated instructor no response to the easily anticipated argument: ``You obviously know what I meant!" Somehow, this isn't the point---but then what is?

I submit for the reader's approval a partial solution to the conundrum of how much to write.

