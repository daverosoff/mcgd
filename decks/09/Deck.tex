\documentclass[11pt,ignorenonframetext,aspectratio=169,xcolor={svgnames}]{beamer}
\usetheme{Szeged}
\usecolortheme{crane}
\usefonttheme{structurebold}
\usefonttheme{professionalfonts}
\usepackage{amssymb,amsmath}
\usepackage{ifxetex,ifluatex}
\usepackage{fixltx2e} % provides \textsubscript
\ifxetex
  \usepackage{fontspec,xltxtra,xunicode}
  \defaultfontfeatures{Mapping=tex-text,Scale=MatchLowercase}
\else
  \ifluatex
    \usepackage{fontspec}
    \defaultfontfeatures{Mapping=tex-text,Scale=MatchLowercase}
  \else
    \usepackage[utf8]{inputenc}
  \fi
\fi
\usepackage{listings}

% Comment these out if you don't want a slide with just the
% part/section/subsection/subsubsection title:
% \AtBeginPart{
%   \let\insertpartnumber\relax
%   \let\partname\relax
%   \frame{\partpage}
% }
% \AtBeginSection{
%   \let\insertsectionnumber\relax
%   \let\sectionname\relax
%   \frame{\sectionpage}
% }
% \AtBeginSubsection{
%   \let\insertsubsectionnumber\relax
%   \let\subsectionname\relax
%   \frame{\subsectionpage}
% }

\setlength{\parindent}{0pt}
\setlength{\parskip}{6pt plus 2pt minus 1pt}
\setlength{\emergencystretch}{3em}  % prevent overfull lines
\setcounter{secnumdepth}{0}

%%% begin dwr insert
\usepackage{patchcmd}
\usepackage{tabulary}   % Support longer table cells
\usepackage{booktabs}   % Support better tables
\usepackage[sort&compress]{natbib}

\usepackage{framed}     % Allow background color for images
\definecolor{shadecolor}{named}{white}

%\usepackage{paralist}
\usepackage{xparse}
\usepackage{subfigure}
\usepackage{hyperref}
%%% end dwr insert
\usepackage{rosoff-vector-macros}
\usepackage[beamer]{cofi}
\title{Optimization and the second derivative test}
\author{Math 275 Multivariable Calculus}
\date{October 11, 2013}


\begin{document}
\frame{\titlepage}

\section{Optimization and the second derivative test}

\begin{frame}\frametitle{Local optima}

If $f(x,y)$ is a function of two variables, it probably has maxima and
minima.

\begin{itemize}
\item
  $f$ has a local maximum at $(x_0,y_0)$ if $f(x_0,y_0) \geq f(x,y)$ for
  every point $(x,y)$ in some small disk containing $(x_0,y_0)$.
\item
  $f$ has a local minimum at $(x_0,y_0)$ if $f(x_0,y_0) \leq f(x,y)$ for
  every point $(x,y)$ in some small disk containing $(x_0,y_0)$.
\item
  $f$ has a local optimum (or extremum) at $(x_0,y_0)$ if $f(x_0,y_0)$
  has either a local max or a local min at $(x_0, y_0)$.
\end{itemize}

\end{frame}

\begin{frame}\frametitle{Global extrema}

Change the phrase ``in some small disk containing $(x_0, y_0)$'' to ``in
the domain of $f$''.

\begin{itemize}

\item
  $f$ has a global maximum at $(x_0,y_0)$ if $f(x_0,y_0) \geq f(x,y)$
  for every point $(x,y)$ in some small disk containing $(x_0,y_0)$.
\item
  $f$ has a global minimum at $(x_0,y_0)$ if $f(x_0,y_0) \leq f(x,y)$
  for every point $(x,y)$ in some small disk containing $(x_0,y_0)$.
\item
  $f$ has a global optimum (or extremum) at $(x_0,y_0)$ if $f(x_0,y_0)$
  has either a global max or a global min at $(x_0, y_0)$.
\end{itemize}

\end{frame}

\begin{frame}\frametitle{Stationary points and critical points}

If we draw a couple of local optima, we notice something about the
tangent planes. They are horizontal, when they exist. This motivates
some more definitions.

\begin{itemize}

\item
  When a function has a horizontal tangent plane at a point $P$, its
  gradient at $P$ is zero. This is because
  $\nabla f_{P} = \angl{f_x(P), f_y(P)}$. We say that $P$ is a
  \emph{stationary point}.
\item
  When a function is not differentiable at a point, its gradient is
  typically undefined, although it's possible that the gradient is 0.
\item
  Points at which either of these occur are called \emph{critical
  points}.
\end{itemize}

Note that the gradient should be considered to be undefined if
\emph{either} of its entries is undefined.

\end{frame}

\begin{frame}\frametitle{Local optima occur at critical points}

If $f(x,y)$ has a local optimum at $(x_0, y_0)$, then $(x_0, y_0)$ is a
critical point of $f$.

\begin{quote}
Take special note of the logical asymmetry of this statement. Its
converse is not true!
\end{quote}

A stationary point that is not a local optimum is called a saddle point.

\end{frame}

\begin{frame}\frametitle{The discriminant}

It is impractical to test critical points of $f(x,y)$ for being local
optima using the first derivative. But there is a convenient analog of
the second derivative test, at least if $f(x,y)$ is smooth enough.
Interestingly, all three second derivatives are involved.

\begin{itemize}

\item
  Let $f(x,y)$ be a function with continuous second-order partials. The
  \emph{Hessian discriminant} of $f$ at $(a,b)$ is defined to be
  $D(a,b) = f_{xx}(a,b) f_{yy}(a,b) - f^2_{xy}(a,b)$.
\end{itemize}

\end{frame}

\begin{frame}\frametitle{Second derivative test}

\begin{itemize}

\item
  If $D(a,b) > 0$ and $f_{xx}(a,b) > 0$, then $f$ has a local minimum at
  $(a,b)$.
\item
  If $D(a,b) > 0$ and $f_{xx}(a,b) < 0$, then $f$ has a local maximum at
  $(a,b)$.
\item
  If $D(a,b) < 0$, then $f$ has a saddle point at $(a,b)$.
\item
  If $D = 0$, the test is inconclusive.
\end{itemize}

\end{frame}

\begin{frame}\frametitle{Inconclusive}

Remember, $D = 0$ doesn't mean ``saddle point''. It means ``test
fails''!

\end{frame}

\begin{frame}\frametitle{Global extrema}

If $f$ is everywhere smooth (everywhere means, on all of $\R^2$) then
its global optima will also be local optima. Of course it may not have
global optima.

But, if $f$ has a domain that is a proper subset of $\R^2$, it may have
global optima that are not local optima. If the domain is \emph{closed
and bounded}, global optima are guaranteed to exist.

\end{frame}

\begin{frame}\frametitle{A little topology}

Let $S$ be a subset of the plane. It's OK to assume that $S$ has a
reasonable shape: that it's possible to draw it, that its edges (if it
has any) are smooth, and so on.

Usually $S$ is defined by algebraic conditions on its coordinates. A
point qualifies for membership in $S$ if---and only if---its coordinates
meet the conditions.

\end{frame}

\begin{frame}\frametitle{Set-builder notation}

We describe such sets first via their conditions. You are familiar with
doing this. If $f$ is some one-variable function, then

\[ \{ (x,y) \in \R^2 : y = f(x) \} \]

is the graph of the function $f$. It is pronounced ``the set of $(x,y)$
in $\R^2$ such that $y = f(x)$''.

If we want to discuss 2-variable functions whose domain is smaller than
the plane, we describe their domains this way.

\end{frame}

\begin{frame}\frametitle{Closed and bounded}

A subset $S$ of the plane is called \emph{closed} if it contains all of
its edge points: that is, if all of the edge points meet the membership
conditions.

We say $S$ is \emph{bounded} if it is contained in a large enough disk;
equivalently, if it is contained in a disk centered at $(0,0)$;
equivalently, if it is possible to draw the set on a finite piece of
paper.

\begin{quote}
Note: the textbook calls edge points ``boundary points''. I prefer to
avoid this terminology because the presence of ``boundary points'' has
nothing to do with ``boundedness.''
\end{quote}

\end{frame}

\begin{frame}\frametitle{Existence of global extrema}

The symbol $\subset$ denotes set containment.

\begin{quote}
\textbf{Theorem}. Let $S \subset \R^2$ be closed and bounded, and let
$f \colon S \to \R^2$ be continuous. Then $f$ has a global maximum and a
global minimum on $S$.
\end{quote}

This is like the ``closed interval method'' from one-variable calculus.
The closed and bounded subset in that case is a finite closed interval.

The theorem guarantees the existence of global optima, but tells us
nothing about how to find them.

\end{frame}

\begin{frame}\frametitle{Interior and edge are separate}

We can look for critical points in the interior of the set (interior
just means the non-edge parts of the set) and make a table of values.
There will only be finitely many such points.

But \ldots{} we have to check the edge points, just like in the
one-variable case.

And there are more than just 2 endpoints, usually.

\end{frame}

\begin{frame}\frametitle{Work together:}

From section 14.7:

\begin{itemize}

\item
  Problem 28
\item
  Example 5
\item
  Problem 35 (in groups, with whiteboards)
\item
  Problem 36 if time permits
\end{itemize}

\end{frame}

\end{document}
