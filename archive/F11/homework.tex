\documentclass[10pt]{amsart}
\usepackage{enumerate,graphicx,wrapfig,amsmath,amsthm,amssymb,latexsym}%,wrapfig,floatflt}
\usepackage{hyperref}
%\usepackage[matrix,arrow,cmtip,curve,dvips]{xy}
\DeclareGraphicsExtensions{.png}
\DeclareMathOperator{\im}{im}
\newcommand{\id}[1][{}]{\mathrm{Id}_{#1}}
\newcommand{\Q}{\mathbf Q}      % adds blackboard bold macros for
\newcommand{\R}{\mathbf R}      % conventional notation for
\newcommand{\Z}{\mathbf Z}      % rationals, reals, integers,
\newcommand{\C}{\mathbf C}      % complexes, naturals, field of p
\newcommand{\N}{\mathbf N}      % elements, p-adic integers
\newcommand{\D}{\mathbf D}
\newcommand{\Zp}{\mathbb{Z}_p}      % and numbers.
\newcommand{\Qp}{\mathbb{Q}_p}
\newcommand{\Fp}{\mathbb{F}_p}
\newcommand{\ie}{\emph{i.e.}}
\newcommand{\eg}{\emph{e.g.}}
\newcommand{\ind}[2]{[#1:#2]}       % index of groups, degree of field
\newcommand{\aut}[2][{}]{\mathrm{Aut}_{#1}(#2)}  % automorphism group
\newcommand{\isom}{\cong}       % isomorphic
\newcommand{\fk}[1]{\mathfrak{#1}}  % fraktur shorthand
\newcommand{\abs}[2][{}]{|#2|_{#1}} % generalized absolute value
\newcommand{\ord}[2][{}]{v_{#1}(#2)}    % roman ord(x), ord_p(x)
\newcommand{\Ex}{\begin{ex}}
\newcommand{\Eex}{\end{ex}}
\newcommand{\tr}[1]{#1^t}
\newcommand{\mat}[2]{M_{#1}(#2)}
\newcommand{\GL}[2]{\mathrm{GL}_{#1}(#2)}
\newcommand{\Lie}[1]{\mathrm{Lie}(#1)}
\newcommand{\ddt}[1]{\left. \frac{\mathrm{d}}{\mathrm{d}t} \right|_{t=#1}}
\newcommand{\ten}{\otimes}
\newcommand{\mg}[1]{{#1}^{\times}}
\newcommand{\wt}[1]{\widetilde{#1}}
\newcommand{\Tor}[2]{\mathrm{Tor}^{#1}_{#2}}

\theoremstyle{plain}            % define numbered and naked theorem
\newtheorem{ntheorem}{Theorem}[section]     % environments, numbered
\newtheorem{nlemma}[ntheorem]{Lemma}        % consecutively within
\newtheorem{nprop}[ntheorem]{Proposition}   % chapters
\newtheorem{ncor}[ntheorem]{Corollary}
\newtheorem*{theorem}{Theorem}
\newtheorem*{lemma}{Lemma}
\newtheorem*{prop}{Proposition}
\newtheorem*{cor}{Corollary}

% \newcounter{ntheorem}

\theoremstyle{definition}
\newtheorem{ndefn}[ntheorem]{Definition}
\newtheorem{nex}[ntheorem]{Example}
\newtheorem*{defn}{Definition}
\newtheorem*{ex}{Example}

%\theoremstyle{remark}
\newtheorem*{rmk}{Remark}
\newtheorem*{ntn}{Notation}

\usepackage[total={7.0in,10.0in},left=1.0in,right=1.0in]{geometry}
\setlength{\parskip}{7.5pt} \setlength{\parindent}{0pt}
%\pretolerance=4000 \setlength{\topmargin}{-1.0in}
%\setlength{\textheight}{10.0in} \setlength{\textwidth}{7in}
%\setlength{\headheight}{26pt} \setlength{\headsep}{8pt}
%\setlength{\oddsidemargin}{-0.25in}
%\setlength{\evensidemargin}{-0.25in}
\title{Mathematics 251, Fall 2011 \\ Homework assignments}
\pagestyle{empty}
\begin{document}
\maketitle
\thispagestyle{empty}

\begin{center}
\textbf{HW 12: Due Friday, December 2}
\end{center}
\begin{enumerate}[I.]
    \item Warm-up
    \begin{enumerate}[A.]
        \item Section 15.5: Exercises 3, 5
        \item Section 16.1: Exercises 18--21
        \item Section 16.2: Exercises 2, 3
        \item Section 16.3: Preliminary Questions 2, 3
    \end{enumerate}
    \item A little harder now
    \begin{enumerate}[A.]
        \item Section 15.5: 21, 24
        \item Section 16.1: 27, 28
        \item Section 16.2: 4--8
        \item Section 16.3: 2, 3, 9
    \end{enumerate}
    \item Optional problems
    \begin{enumerate}[A.]
        \item Section 16.4: Preliminary Questions 1, 2; Exercises 3, 5
        \item Section 16.5: Preliminary Questions 3, 5; Exercises 1, 2, 5, 7
    \end{enumerate}
\end{enumerate}

\begin{center}
\textbf{HW 11: Due Wednesday, November 23}
\end{center}
Note. We didn't discuss some of the applications of triple integrals in class. Give section 15.3 a good read. You may find the section on moments and centers of mass particularly illuminating; these concepts are much more naturally expressed in terms of multiple integrals than in the Calc II single-variable way (for example, the coordinates of the centroid (this is the same as center of mass if we assume constant density) are average values of the coordinate functions). I also suggest a very close reading of Example 4 in 15.3, verifying all the calculations for yourself. We have not seen a complete example of this sort in class.
\begin{enumerate}[I.]
    \item Warm-up
    \begin{enumerate}[A.]
        \item Section 15.3: Preliminary Question 3, Exercise 3
        \item Section 15.4: Preliminary Question 2, Exercise 5 (draw a picture!)
    \end{enumerate}
    \item Conceptual practice
    \begin{enumerate}[A.]
        \item Section 15.3: 3, 8, 11, 16
        \item Section 15.4: Exercises 8, 10, 17, 19, 33, 54
    \end{enumerate}
    \item Presentation problems
    \begin{enumerate}[A.]
        \item Section 15.3: 29, 33
        \item Section 15.4: 67
    \end{enumerate}
\end{enumerate}

\begin{center}
\textbf{HW 10: Due Wednesday, November 16}
\end{center}
I will grade items IIB and III.
\begin{enumerate}[I.]
    \item Warm-up
    \begin{enumerate}[A.]
        \item Section 15.1: 17--27 odd
        \item Section 15.2: Preliminary Questions 2, 3; Exercises 3--7, 9
    \end{enumerate}
    \item Conceptual practice
    \begin{enumerate}[A.]
        \item Section 15.1: 46
        \item Section 15.2: 11, 14, 15, 31--34
    \end{enumerate}
    \item Presentation problems (note that we have a holdover presentation from the 11th)
    \begin{enumerate}[A.]
        \item Section 15.1: 45
        \item Section 15.2: 53
    \end{enumerate}
    \item \textsc{Bonus problem.} Section 15.1: 47.
\end{enumerate}
\begin{center}
\textbf{HW 9: Due Friday, November 11}
\end{center}
\begin{enumerate}[I.]
    \item Warm-up
    \begin{enumerate}[A.]
        \item Section 14.6: 1--15 odd (notice you don't need to write much by way of explanation; just organize the calculations in a neat, readable manner)
        \item Section 14.7: Preliminary Questions 2, 3; Exercises 4, 5--13 odd
        \item Section 14.8: Preliminary Question 2; Exercises 1, 2
    \end{enumerate}
    \item Conceptual practice. Make sure you do these.
    \begin{enumerate}[A.]
        \item Section 14.6: 18--21, 24, 25, 49
        \item Section 14.7: 24, 25, 27 (give two arguments, one with and one without calculus), 32, 38
        \item Section 14.8: 1, 2, 3--11 odd, 14, 37
    \end{enumerate}
    \item These are presentation problems.
    \begin{enumerate}[A.]
        \item Section 14.7: 37 (long, but illuminating!)
        \item Section 14.8: 15, 33, 40
    \end{enumerate}
\end{enumerate}

\begin{center}
\textbf{HW 8: Due Friday, November 4}
\end{center}
As usual, items I and II are for practice. I will grade item III carefully, but I will deduct points if items I or II aren't complete. Employ more discretion in deciding how much to write.
\begin{enumerate}[I.]
    \item Warm-up
    \begin{enumerate}[A.]
        \item Section 14.4: 11--19 odd
        \item Section 14.5: Preliminary Questions 2, 6; Exercises 3, 4, 9--19 odd
    \end{enumerate}
    \item Grasping the concepts
    \begin{enumerate}[A.]
        \item Section 14.4: 35--39 (\href{http://www.collegeofidaho.edu/academics/MathPhysics/courses/MAT-251/rosoff/error.pdf}{Here} is a brief explanation of the error calculations)
        \item Section 14.5: 21--31 odd, 39, 50, 64
    \end{enumerate}
    \item These are all presentation problems. Email me to volunteer to present one on Friday. The idea is that 14.4.42 is too long to be one presentation. It's broken into two. There is an error, moreover, in the problem formulation. Part (d) should read: ``Show that if a function~$f(x,y)$ is locally linear at~$(0,0)$ and~$0~=f(0,0)~=~f_x(0,0)~=~f_y(0,0)$, then $f(h,h)/h$ approaches~0 as $h~\to~0$.'' The conditions about~$f_x$ and~$f_y$ are missing from the first edition.
    \begin{enumerate}[A.]
        \item Section 14.4: 42(a--c), 42(d--f)
        \item Section 14.5: 45, 67
    \end{enumerate}
\end{enumerate}
\begin{center}
\textbf{HW 7: Due Wednesday, October 26}
\end{center}
This week I broke the problems up by section, with subparts inside each section-part. Submit all parts Wednesday, October 26; within each part I indicate presentation problems (presentations will also be the 26th) and which problems I plan to grade. When the problem is mostly concerned with drawing a figure, try to draw a big, beautiful picture. \emph{Note.} I let you all know which problems I will grade to aid in time management. I still want you to do all the assigned work. Solving problems like this is the only way to get a deep understanding of these ideas. The idea is that the homework should be flexible enough that it does not interfere unduly with your other work. Please do not neglect the ungraded items! It won't help you in the end.
\begin{enumerate}[I.]
    \item Section 14.1: Introduction to functions of several variables. \textbf{Note. It is essential that you read this section carefully!} Items A through C are for practice and to help you become familiar with functions of several variables; I will grade item D.
    \begin{enumerate}[A.]
        \item Preliminary Questions 1, 3, 4, 5.
        \item Exercises 6--12, 20, 23, 28, 42 (do not try to draw the graph first! just draw the level curves for a few values of $z = f(x,y)$).
        \item Read \textbf{all} of the Exercises in the section. Think, for at least 5 minutes, about how the contour plot of a function of two variables is related to its graph.
        \item Presentation Problem: Exercise 45.
    \end{enumerate}
    \item Section 14.2: Limits and continuity. \textbf{This section also merits an extra-careful reading.} Item A is for practice; I will grade items B and C.
    \begin{enumerate}[A.]
        \item Exercises 1--12.
        \item Exercises 23, 27, 30. For the latter two, there is a useful hint buried on page 805, after Example 2. It may also save you some work to recognize certain limits as derivatives of familiar functions, although this isn't essential.
        \item Presentation Problem: Exercises 33, 35, 43 (a and c only).
    \end{enumerate}
    \item Section 14.3: Partial derivatives. Items A and B are for practice taking partial derivatives; I will grade item C. No presentation problems from this section.
    \begin{enumerate}[A.]
        \item Preliminary Questions 1--5.
        \item Exercises 13--29 odd, 46.
        \item Exercises 50, 71, 72. Read, but do not hand in, exercises 76--80.
    \end{enumerate}
\end{enumerate}

\begin{center}
\textbf{HW 6: Due Wednesday, October 12}
\end{center}
    Through a perturbative process I am finally arriving at a nicer format for the class and homeworks. Each week or so we will have a presentation day. Everyone should have an opportunity to present once this term. I will designate some of the more challenging problems as Presentation Problems. You should be able to present a solution to these on the board on the specified class day. Anyone could be asked to present. The presentations of HW 5 were all good examples of what these should look like---and an extra thanks to our presenters, who volunteered with very little notice.

    This week I will grade items III and IV for both mathematical correctness and general writing style. The better the presentations are in class, the easier I will grade the written solutions. If the quality stays high then you will all be able to write less for the presentation problems' written solutions. The presentations of this week's problems will be on Wednesday, October 12.
\begin{enumerate}[I.]
    \item Section 13.5, Preliminary Questions 1--5. Keep your explanations brief, but complete. Two or three sentences per question should suffice. You are of course free to type this (or any) part of the homework assignment.
    \item Section 13.5, Exercises 2 (copy the entire figure), 11, 16, 25, 26, 28. These should be routine.
    \item Section 13.5, Exercises 19, 23, 36, 42; Section 13.6, Exercises 4, 7.
    \item \textbf{Presentation Problems.} Section 13.5, Exercise 47; Section 13.6, Exercises 25, 26.
\end{enumerate}

\begin{center}
\textbf{HW 5: Due Wednesday, October 5}
\end{center}
     Item I consists of problems to help you practice using the curvature formulas. I will grade items II and III. They all are of greater complexity than the previous homework problems. For full credit, your solutions must contain enough explanation of the calculations that I don't have to guess at your meaning. This explanation should consist of complete sentences that flow around the equations. Remember that numeric answers must be given in exact form (do not pass to decimal approximations). It isn't required, but it is good form to write down the problem above the beginning of your solution.
\begin{enumerate}[I.]
    \item Section 13.4: Exercises 1, 3, 11, 17 (find curvature at $\pi/3$ only).
    \item Section 13.4: Exercises 22 (answer: $x = \ln \sqrt{1/2}$), 23, 24 (make sure to handle all cases: $a < b$, $a = b$, $a > b$).
    \item \textbf{Supplementary Problem}. Let $\mathbf{r} \colon [a,b] \to \R^3$ be a curve (the notation indicates that the values of the parameter, say $t$, satisfy $a \leq t \leq b$).
        \begin{enumerate}
           \item First suppose that $\mathbf{r}$ is a parametrization by arc length, that is, suppose that $\|\mathbf{r}'(t)\| = 1$. Define a 2-argument arc length function by $L(t_1, t_2) = \int_{t_1}^{t_2} \|\mathbf{r}'(t)\| \; dt$. When $a \leq t_1 < t_2 \leq b$, $L(t_1,t_2)$ computes the arc length of the curve segment between $\mathbf{r}(t_1)$ and $\mathbf{r}(t_2)$. (A minus sign will appear if $t_1 > t_2$.) Show that, with $t_1 < t_2$, $L(t_1, t_2) = t_2 - t_1$.
           \item Now let $\mathbf{r}$ be a \emph{regular} curve with parameter domain $[a,b]$. This means that $\mathbf{r}'(t) \ne 0$ for all $t \in [a,b]$. Define $L(t_1,t_2)$ as above, and suppose that whenever $a \leq t_1 < t_2 \leq b$,
               \[ L(t_1,t_2) = t_2 - t_1. \]
               Show that $\mathbf{r}$ is a parametrization by arc length: that is, show that
               \[ \| \mathbf{r}'(t) \| = 1 \quad \text{for $t \in [a,b]$.} \]
               \emph{Hint.} Use the Fundamental Theorem of Calculus on a function related to $L$.
        \end{enumerate}
\end{enumerate}
Hints for exercise 24: A. It's long. Bring your stamina. B. The curvature function is a constant divided by another function $g(t)$. How do the maxes (mins) of curvature relate to those of $g$? C. Is there a simpler function than $g$ that has the same maxes and mins? D. When you optimize this simpler function, a trig identity will be of great assistance. (Hints B and C are intended to make the calculus easier to do. This function is sheer torture to optimize directly. Feel free to ask for help!)
\begin{center}
\textbf{HW 4: Due Wednesday, September 21}
\end{center}
\begin{itemize}
    \item Section 12.3: 23--31 odd, 52, 55 (hint: trig identities)
    \item Section 12.4: 6, 12, 15, 27
    \item Section 12.5: 7, 9, 13, 25, 34
    \item Section 12.6: 1--9 odd; think about, but don't hand in, Preliminary Question (PQ) 3 and Exercise 42
    \item Section 12.7: PQ 4, 5; Exercises 25, 27, 43, 45, 53, 61
\end{itemize}
In this assignment, you should begin to pay more attention to how you structure your written solutions. I will ask more writing of you. Your solutions should resemble the worked examples in the textbook: restatements of the problem, clearly visible definitions of notational symbols, deductions accompanying calculations, justification of calculations, and figures when appropriate are all elements of a complete solution (this is not an exhaustive list). For easy problems that really do consist of just one calculation (e.g. ``Find the cross product of \dots'', etc.) this is less important. For problems that are more involved it is essential to a good mathematical style.

\begin{center}
\textbf{HW 3: Due Wednesday, September 14}
\end{center}
    No more transcription homework. I don't think it's necessary. This is the final version of HW 3; no further problems will be added.
\begin{itemize}
    \item Section 11.3: Exercises 2, 3, 8, 12--17, 31.
    \item Section 12.1: Exercises 5--13, 16--20 (include big, beautiful pictures!), 37--45 odd.
    \item Challenge problem (optional): 12.1, 66.
    \item Section 12.2: Exercises 4, 13, 22, 24, 27, 29.
\end{itemize}
I'll grade 11.3 \#31, 12.1 \#20 and \#43, and 12.2 \#29, so be sure to include clear, complete explanations of your solutions to these problems, as well as big, beautiful pictures if appropriate.


\begin{center}
\textbf{HW 2: Due Wednesday, September 7} \\ \emph{(Note the due date!)}
\end{center}
\begin{itemize}
    \item Transcribe: Summary of Sec.\ 11.2 on p.\ 637.
    \item Solve: Preliminary questions 4, Exercises 4--6, 18--20, 23.
    \item Challenge problem: 24.
\end{itemize}
\textbf{Note on section 11.2.} There is no need to memorize the surface area formula at the end of the section. The curious are encouraged to derive it for themselves using an adaptation of today's arguments to the ``washer method'' familiar from Calculus 2.

\begin{center}
\textbf{HW 1: Due Monday, September 5}
\end{center}
\begin{itemize}
    \item Transcribe: Conceptual Insight on p.\ 622; Theorem 1 on p.\ 627; Summary on p.\ 628; Theorem 1, p.\ 634; Theorem 2, p. 635; The Mean Value Theorem, p. 230.
    \item Solve: Preliminary Questions 1--5; Exercises 1, 3--5, 7--13 odd, 19 (draw pictures), 23--31 odd, and 49--57 odd. Practice writing your solutions with verbal exposition to accompany your figures, equations, and calculations. \emph{Solutions lacking this exposition may not receive full credit.}
    \item Challenge problem (optional): Problems 78 and 85 from Section 11.1.
\end{itemize}

\end{document} 