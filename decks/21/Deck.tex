\documentclass[11pt,ignorenonframetext,aspectratio=169,xcolor={svgnames}]{beamer}
\usetheme{Szeged}
\usecolortheme{crane}
\usefonttheme{structurebold}
\usefonttheme{professionalfonts}
\usepackage{amssymb,amsmath}
\usepackage{ifxetex,ifluatex}
\usepackage{fixltx2e} % provides \textsubscript
\ifxetex
  \usepackage{fontspec,xltxtra,xunicode}
  \defaultfontfeatures{Mapping=tex-text,Scale=MatchLowercase}
\else
  \ifluatex
    \usepackage{fontspec}
    \defaultfontfeatures{Mapping=tex-text,Scale=MatchLowercase}
  \else
    \usepackage[utf8]{inputenc}
  \fi
\fi
\usepackage{listings}

% Comment these out if you don't want a slide with just the
% part/section/subsection/subsubsection title:
% \AtBeginPart{
%   \let\insertpartnumber\relax
%   \let\partname\relax
%   \frame{\partpage}
% }
% \AtBeginSection{
%   \let\insertsectionnumber\relax
%   \let\sectionname\relax
%   \frame{\sectionpage}
% }
% \AtBeginSubsection{
%   \let\insertsubsectionnumber\relax
%   \let\subsectionname\relax
%   \frame{\subsectionpage}
% }

\setlength{\parindent}{0pt}
\setlength{\parskip}{6pt plus 2pt minus 1pt}
\setlength{\emergencystretch}{3em}  % prevent overfull lines
\setcounter{secnumdepth}{0}

%%% begin dwr insert
\usepackage{patchcmd}
\usepackage{tabulary}   % Support longer table cells
\usepackage{booktabs}   % Support better tables
\usepackage[sort&compress]{natbib}

\usepackage{framed}     % Allow background color for images
\definecolor{shadecolor}{named}{white}

%\usepackage{paralist}
\usepackage{xparse}
\usepackage{subfigure}
\usepackage{hyperref}
%%% end dwr insert
\usepackage{rosoff-vector-macros}
\usepackage[beamer]{cofi}
\title{From Green to Stokes}
\author{Math 275 Calculus 3}
\date{December 2, 2013}


\begin{document}
\frame{\titlepage}

\section{From Green to Stokes}

\begin{frame}\frametitle{Recall some definitions}

The vector field $\vec{F}$ is \emph{conservative} if it is the gradient
of a function; that is, if there is a $V$ with $\nabla V = \vec{F}$.

\begin{equation*}
    \oint_{\mathcal{C}} \nabla V \cdot d\vec{s} = V(Q) - V(P)
\end{equation*}

where $P$ and $Q$ are the endpoints of $\mathcal{C}$.

$\vec{F}$ is \emph{path-independent} if, for every pair of points $P$
and $Q$, the value of the line integral

\begin{equation*}
    \int_{\mathcal{C}} \vec{F} \cdot d\vec{s}
\end{equation*}

is independent of the path $\mathcal{C}$ connecting $P$ with $Q$.

\end{frame}

\begin{frame}\frametitle{All conservative fields are path-independent}

We saw before that all conservative fields are path-independent
$V(Q) - V(Q) = 0$.

What about the converse statement? If a field is path-independent, must
it admit a potential function?

\end{frame}

\begin{frame}\frametitle{Only conservative fields are path-independent}

Suppose the vector field $F$ is known to be path-independent. Must it
admit a potential function? The answer, perhaps surprisingly, is yes, at
least if the domain of $F$ is \emph{connected}. Connected sets are ``all
one piece'': for us, if every pair of points of a set may be joined by a
curve that doesn't leave the set, then the set is connected.

The proof proceeds by choosing a point $P_0$ and constructing a
potential function for $F$ by integration. We define $V(P)$ by the
formula

\[ V(P) = V(x,y) = \int_{\mathcal{C}} \vec{F} \cdot d\vec{s}, \]

where $\mathcal{C}$ is any path from $P_0$ to $P$. Since the field $F$
is assumed to be path-independent, this definition makes sense.

\end{frame}

\begin{frame}\frametitle{Proof that $\nabla V = \vec{F}$}

The proof proceeds by recognizing the difference quotient
\[ \frac{V(x+h,y) - V(x,y)}{h} \] as the average value of $F_1(x,y)$
over the interval $[x,x+h]$. Since this average converges to $F_1(x,y)$
as $h \to 0$, so does the difference quotient. Hence
$\partial V/\partial x = F_1$.

The proof for $F_2$ is similar.

\end{frame}

\begin{frame}\frametitle{Conservation of energy}

In physics, the principle of conservation of energy says that the sum of
kinetic and potential energy of an isolated system does not change. That
is, energy neither enters nor leaves the system---it is
\emph{conserved}.

It is shown in the textbook that if $F$ is a force field that is
conservative in the sense we've been discussing, then particles moving
under its influence obey the principle of conservation of energy.

\end{frame}

\begin{frame}\frametitle{Testing for independence}

How could we ever recognize a field as path-independent? It's impossible
to test every path by integrating.

Observe that if $F$ is conservative, it satisfies the following
cross-partials equation:

\[ \frac{\partial F_1}{\partial y} = \frac{\partial F_2}{\partial x}. \]

There is set of similar equations for 3-dimensional conservative vector
fields.

\[ \frac{\partial F_2}{\partial z} = \frac{\partial F_3}{\partial y}, \frac{\partial F_3}{\partial x} = \frac{\partial F_1}{\partial z}, \frac{\partial F_1}{\partial y} = \frac{\partial F_2}{\partial x} \]

\end{frame}

\begin{frame}\frametitle{Nonzero circulation detects nonconservativity}

Last time we noticed that out of all the contour integrals we looked at
(line integrals around closed loops), there were a couple that had
nonzero values.

These nonzero line integrals are witnesses to the nonconservativity of
the corresponding vector fields.

If a field is conservative, all of its contour integrals are zero;
hence, if \emph{even one contour integral} of some vector field is
nonzero, that field admits no potential function.

All the fields we saw with this property seemed to have some kind of
rotational character: swirling down a drain, or something similar.

\end{frame}

\begin{frame}\frametitle{Curls make you stronger}

Let's focus on the 2-dimensional case: the scalar quantity
\[ \frac{\partial F_2}{\partial x} - \frac{\partial F_1}{\partial y} \]
is sometimes called the \emph{scalar curl} of the vector field
$\vec{F}$. It's a number, or if you like, a one-dimensional vector.

The scalar curl measures what is called the \emph{vorticity} of
$\vec{F}$. If it is zero at some point, $P$, say, then a little paddle
wheel dropped into $\vec{F}$ at $P$ might move with the flow, but it
won't spin.

If it is positive, the paddle wheel will spin counterclockwise, and vice
versa.

Thus our two criteria for path-independence are really related.

\end{frame}

\begin{frame}\frametitle{Nonzero curls violate conservation of energy}

\begin{itemize}[<+->]
\itemsep1pt\parskip0pt\parsep0pt
\item
  If $\vec{F}$ has a nonzero curl, we can cheat the laws of physics and
  get a free ride. Suppose the vector field is generally rotating
  counterclockwise; then by going with the flow, you can travel back to
  your starting point doing negative work.
\item
  All follow-up questions about conservation of energy referred to the
  Physics faculty
\end{itemize}

\end{frame}

\begin{frame}\frametitle{A topological criterion}

Remarkably, it is possible to find vector fields that satisfy the
cross-partials equation that are not path-independent (and hence, not
conservative). For example, the vortex vector field
\[ \angl{ \frac{-y}{x^2 + y^2}, \frac{x}{x^2 + y^2} } \] on $\RR^2$ has
this property. Yet integrating it around the unit circle yields a
nonzero value (see p.~971 ``Conceptual Insight'').

But notice that this vector field doesn't extend to any subset of the
plane that is free of \emph{holes} (because of the denominators).

A domain without holes is called \emph{simply connected}. If a vector
field on a simply connected domain satisfies the cross-partials
equations, then it is conservative.

\end{frame}

\begin{frame}\frametitle{Summary}

If $\vec{F}$ is a smooth vector field on a \emph{connected} domain, then
$\vec{F}$ is path-independent if and only if $\vec{F}$ is conservative.

If in addition the domain of $\vec{F}$ is \emph{simply connected}, then
these are also equivalent to the cross-partials property for $\vec{F}$;
or, in other words, to the vanishing of the scalar curl of $\vec{F}$.

\end{frame}

\begin{frame}\frametitle{Where does the scalar curl come from?}

The formula for the scalar curl can be obtained as the \emph{circulation
per unit area}. Its value at $P$ is the value of the expression
\[ \lim_{\Delta A \to 0} \frac{1}{\Delta A} \oint \vec{F} \cdot d\vec{s} \]
for every vector field $\vec{F}$, where we take our integrals around
little disks of area $\Delta A$ with center $P$.

If the scalar curl is really the circulation per unit area, shouldn't we
be able to integrate it over the \emph{interior} of a region and obtain
the circulation?

\end{frame}

\begin{frame}\frametitle{Green's theorem}

Let $\mathcal{D}$ be a region of the plane whose edge $\partial D$ is a
simple closed curve, oriented counterclockwise. Then

\[ \oint_{\mathcal{\partial D}} \vec{F} \cdot d\vec{s} = \iint_{\mathcal{D}} \left( \frac{\partial F_2}{\partial x} - \frac{\partial F_1}{\partial y} \right) \; dA.\]

In our alternate notation for line integrals, we obtain a more classical
expression of the result.

\[ \oint_{\mathcal{\partial D}} F_1 \; dx + F_2 \; dy = \iint_{\mathcal{D}} \left( \frac{\partial F_2}{\partial x} - \frac{\partial F_1}{\partial y} \right) \; dA.\]

\end{frame}

\end{document}
