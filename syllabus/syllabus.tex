\documentclass[12pt,twoside]{amsart}
\usepackage{%
%graphicx,
%wrapfig,
amsmath,amsthm,amssymb,latexsym,paralist}
\usepackage{fourier, MnSymbol}
%\DeclareGraphicsExtensions{.png}
%\DeclareMathOperator{\im}{im}
\newcommand{\id}[1][{}]{\mathrm{Id}_{#1}}
\newcommand{\Q}{\mathbf Q}      % adds blackboard bold macros for
\newcommand{\R}{\mathbf R}      % conventional notation for
\newcommand{\Z}{\mathbf Z}      % rationals, reals, integers,
\newcommand{\C}{\mathbf C}      % complexes, naturals, field of p
\newcommand{\N}{\mathbf N}      % elements, p-adic integers
\newcommand{\D}{\mathbf D}
\newcommand{\Zp}{\mathbb{Z}_p}      % and numbers.
\newcommand{\Qp}{\mathbb{Q}_p}
\newcommand{\Fp}{\mathbb{F}_p}
\newcommand{\ie}{\emph{i.e.}}
\newcommand{\eg}{\emph{e.g.}}
\newcommand{\ind}[2]{[#1:#2]}       % index of groups, degree of field
\newcommand{\aut}[2][{}]{\mathrm{Aut}_{#1}(#2)}  % automorphism group
\newcommand{\isom}{\cong}       % isomorphic
\newcommand{\fk}[1]{\mathfrak{#1}}  % fraktur shorthand
\newcommand{\abs}[2][{}]{|#2|_{#1}} % generalized absolute value
\newcommand{\ord}[2][{}]{v_{#1}(#2)}    % roman ord(x), ord_p(x)
\newcommand{\Ex}{\begin{ex}}
\newcommand{\Eex}{\end{ex}}
\newcommand{\tr}[1]{#1^t}
\newcommand{\mat}[2]{M_{#1}(#2)}
\newcommand{\GL}[2]{\mathrm{GL}_{#1}(#2)}
\newcommand{\Lie}[1]{\mathrm{Lie}(#1)}
\newcommand{\ddt}[1]{\left. \frac{\mathrm{d}}{\mathrm{d}t} \right|_{t=#1}}
\newcommand{\ten}{\otimes}
\newcommand{\mg}[1]{{#1}^{\times}}
\newcommand{\wt}[1]{\widetilde{#1}}
\newcommand{\Tor}[2]{\mathrm{Tor}^{#1}_{#2}}

\theoremstyle{plain}            % define numbered and naked theorem
\newtheorem{ntheorem}{Theorem}[section]     % environments, numbered
\newtheorem{nlemma}[ntheorem]{Lemma}        % consecutively within
\newtheorem{nprop}[ntheorem]{Proposition}   % chapters
\newtheorem{ncor}[ntheorem]{Corollary}
\newtheorem*{theorem}{Theorem}
\newtheorem*{lemma}{Lemma}
\newtheorem*{prop}{Proposition}
\newtheorem*{cor}{Corollary}

% \newcounter{ntheorem}

\theoremstyle{definition}
\newtheorem{ndefn}[ntheorem]{Definition}
\newtheorem{nex}[ntheorem]{Example}
\newtheorem*{defn}{Definition}
\newtheorem*{ex}{Example}

%\theoremstyle{remark}
\newtheorem*{rmk}{Remark}
\newtheorem*{ntn}{Notation}

\usepackage[top=1.0in,bottom=1.0in,left=1.2in,right=1.2in]{geometry}
\usepackage{parskip}
%\setlength{\parskip}{1.4ex}
%\setlength{\parindent}{0pt}
\linespread{1.135}
\frenchspacing
\hyphenation{pa-ra-met-riz-a-tion}
\title{{\LARGE Mathematics 251 \hfill Fall 2014}}
\pagestyle{empty}
\usepackage{hyperref}
\begin{document}
\maketitle
\thispagestyle{empty}
\vspace*{-2ex}
\begin{minipage}[t]{0.45\linewidth}
    \textbf{Instructor}: Dr.\ Dave Rosoff  \\
    \textbf{Office}: Boone Hall 102C \\
    \textbf{Office hours}: T 9:15--10:30, W 9:10--11:30, F 11:00--12:25, or by appointment \\
    \textbf{Email}: \href{mailto:drosoff@collegeofidaho.edu}{drosoff@collegeofidaho.edu} \\
\end{minipage} 
\hspace*{0.09\linewidth}
\begin{minipage}[t]{0.45\linewidth}\begin{flushright}
    \textbf{Website}: \url{https://zeus.collegeofidaho.edu/academics/MathPhysics/courses/MAT-251/} \\
    \textbf{Twitter}: \verb|@daverosoff|
\end{flushright}
\end{minipage}
\vspace*{-2ex}
\begin{center}

{\large \emph{The pursuit of knowledge, brother, is the askin' of many questions}\footnote{Raymond Chandler, \emph{Farewell, My Lovely}, 1940.}}.

\end{center}

\textbf{Greetings:} Welcome to MAT 251, Calculus 3. I am very pleased to be teaching the course once more. It is the first college mathematics course that is not designed to be approachable by students of all majors, and as such will form the foundation of your continued work in the mathematical and physical sciences. It is more modern than your previous courses in calculus. The machinery and terminology of \emph{vectors}, in particular, is quite young, dating back to the work of Gibbs and Hamilton in the late 19th century---less than 150 years ago, as compared to five centuries or more for the foundational results of single-variable calculus\footnote{Newton and Leibniz tend to get all the credit for inventing calculus in the 17th century, but some key notions go all the way back to Archimedes, who lived about 22 centuries ago.}.

The name of the course is traditional, and at one time all colleges and universities throughout this country had very similar offerings called ``Calculus 3'', but matters have changed in recent years. Quick and reliable hand calculation was once a skill of paramount importance, but is less so today. It is more important that you have a good working understanding of the fundamental concepts of calculus. This understanding will lead to asking the correct questions of software, literature searches, and your eventual colleagues.

The fundamental idea and theme of the course is what is called ``parametrization''. Too many related concepts are built into the meaning of this word to explain it more here, but the main goal of the course is for you to have an enduring understanding of this idea that permits you to recognize, employ, and interpret parametrization in your future mathematical enterprises. See the section \emph{Course overview} below for a more comprehensive introduction.

\textbf{Learning outcomes:} Students who successfully complete this course will:
\begin{compactitem}
    \item Recognize and describe fundamental ideas from course content described below.
    \item Illustrate these ideas with examples and translate them into everyday terminology.
    \item Demonstrate the use of calculus on specific numerical problems.
    \item Classify the techniques of calculus according to their use.
    \item Improve their mathematical writing skills.
    \item Develop their communication skills by presenting mathematical material.
    \item Effectively discuss and solve mathematical problems in group settings.
    \item Infer connections made in arguments without specific direction.
    \item Explain why arguments making use of calculus are sound or not, as appropriate.
    \item Plan, organize, and combine arguments to solve new problems, as appropriate.
    \item Generalize or modify to create new arguments, as appropriate.
\end{compactitem}

\textbf{Text}: The text is \emph{Calculus: Early Transcendentals} by Jon Rogawski, second edition. It is OK if you have the first edition, although your section numbers may be different. If you choose to use a different edition than the recommended one, it is your responsibility to make sure you submit homework assignments correctly. 

\textbf{Catalog description}: ``A study of real functions of several real variables. Topics include differentiability and continuity, differential geometry, extrema, Lagrange multipliers, multiple integration, line and surface integrals, and the theorems of Green, Gauss and Stokes.''

\textbf{Grading}: Scores are computed as a weighted average, with the following weights: WeBWorK $0.10 = 10\%$, presentations $0.15 = 15\%$, portfolios $0.05 = 5\%$, quizzes $0.40 = 40\%$, two in-class exams $0.18 = 18\%$, and final exam $0.12 = 12\%$. Observe that the weights sum to $1 = 100\%$. The exact determination of letter grades from these scores depends on the final distribution of scores in the class, but you can expect a C for earning $75\%$ of the points, a C+ for $78\%$, a B-- for $80\%$, and so on.

\textbf{Homework}: Homework in this class comprises both online and traditional written assignments.
\begin{compactitem}
    \item WeBWorK assigned almost-daily, available at \\
    \url{https://webwork.collegeofidaho.edu/webwork2/MAT251_01_F13/}
    \item \emph{Portfolio problems}, assigned in weekly batches and collected regularly throughout the term. These problems mostly come from the textbook and are intended to stretch your thinking beyond the level of the online homework. It is these problems that you have opportunities to present at the board. Note that you may record classmates' solutions to these problems as they present.
\end{compactitem}

\textbf{How to do homework:} Whenever you are writing a solution to a math problem, it is important to strive for the clearest exposition you can manage. Good mathematical writing is essential for anyone who wishes to think clearly about mathematics. The process of making your ideas and reasoning \emph{clear, complete, and unambiguously correct} is the most effective amplifier of mathematical power there is. Hence your solutions should be composed in brilliant English prose. This means employing accepted scientific usage, more or less correct grammar and spelling, and above all \emph{complete sentences}---sprinkled here and there with tangy, delicious equations. Solutions in the popular ``pile-of-equations'' style with little or no explanatory text will not get much credit. You must explain what is happening as the action unfolds. 

The reason for all this is that the process of such writing and editing will implant understanding more firmly in your mind. Similar problems appear on exams. If you dash off a quick and dirty solution it is less likely you will recall what you did at the appropriate time.

I encourage all students to form study groups and collaborate on homework; each student is of course individually responsible for their own work. Collaborators must be acknowledged. Portfolio problems may be submitted more than once and scores are not final until the last day of class. \emph{No portfolios will be accepted after the last day of class without a compelling, documented reason.}

\textbf{Quizzes}: To make sure each student is keeping up with the online problems, we will have frequent quizzes. Quiz problems are simple variants of WeBWorK problems and all quizzes are open-note. Therefore, if you have solved the WeBWorK problems prior to the quiz day, a score of $100\%$ is very possible.

\textbf{Presentations}: Each student should attempt to earn 60 \emph{presentation points} during the semester. Presentations of portfolio problems are worth up to 10 points each. A presentation consists of two or three active presenters who provide an explanation, a solution, and any justification of the solution requested by the class. I will maintain a list of problems and their presentation metadata online for your reference. 

\textbf{Exams}: Two exams are given in class (see below for dates). Exams comprise both portfolio problems and new problems intended to stretch your thinking. During the exam, you may refer freely to your portfolio. A missed exam results in an exam grade of zero. Arrangements for absences, again, must be made well in advance (two weeks suffices). \emph{If arrangements are not made in advance, I will consider make-ups only with compelling, documented reasons.} The final exam takes place at the indicated date and time. It cannot be rescheduled \emph{for any reason}. Make your travel plans accordingly.
\begin{compactitem}
	\item Exam 1 (tentative): Tuesday, September 24 (week 3)
    \item Exam 2 (tentative): Tuesday, November 12 (week 9)
	\item Final Exam: Monday, December 9, 1:30--4:30
\end{compactitem}

\textbf{Course overview}: We generalize the main ideas and results of single-variable calculus to multivariate situations, with particular attention to applications and modeling. There are two ways to proceed. First, we allow the domains (inputs) of the functions at hand to be 2-, 3-, or higher-dimensional, while maintaining the restriction that the function's values are real numbers in the usual sense. Here we will meet the essential concepts of \emph{partial derivative} and \emph{total} or \emph{Jacobian derivative} and revisit the familiar themes of differential calculus (related rates, optimization, etc.). We will introduce the powerful modern algebraic framework of \emph{vectors} in which the rest of the theory is cast. At this level of abstraction there are many opportunities to discuss mathematical modeling, of which we avail ourselves as appropriate.

Just as functions of several variables may be differentiated via their partial derivatives, so too can they be integrated over appropriate \emph{regions} (other than intervals) in their domains. We study the important change-of-variable theorem that allows integrals over complicated regions to be reckoned in terms of simpler ones (e.g., rectangles) and examine some applications. 

The dual part of our investigation is to let the codomains of our functions be higher-dimensional Euclidean spaces while keeping the domains 1-dimensional. This approach entails an investigation into fundamental ideas like \emph{parametrization}.

We conclude with a study of the important field of \emph{differential geometry}. The laws of electromagnetism, Maxwell's equations, are formulated in this language, as are the laws of aerodynamics. We will make as much progress as we can toward the fundamental theorems of G.\ Green, C.\ F.\ Gauss, and G.\ Stokes\footnote{Famously, Stokes's theorem is due not to Stokes, but to Lord Kelvin; Stokes merely \emph{assigned} it, in 1854.}. All three are vast generalizations of the usual Fundamental Theorem of Calculus, and all have extremely important implications for physics, engineering, and the rest of mathematics.

\textbf{A note on studying math}: By now you have studied enough mathematics to have learned something about how it is that the material passes through your shapely skull and into your soft, wrinkly brain. Nevertheless, you may find that this course is rather more difficult than your previous calculus courses. To really understand it, we will have to dig into subtle distinctions and nuances that no one has asked you to think about before. In the one-dimensional world, these nuances simply do not signify, so they are novel by necessity at this time. It is also more difficult to picture what is going on mentally, again owing to the presence of extra dimensions. Part of what I'm here to tell you is that while the material may seem wholly new and unfamiliar, the underlying principles of calculus (what do derivatives \emph{do}? what are integrals \emph{for}?) are immutable.

\textbf{Academic integrity}: Students are expected to complete all graded work in accordance with the College Honor Code. Plagiarism, cheating, or borrowing without proper credit will not be tolerated.  Violations of academic honesty can result in loss of credit on an assignment, failure on an exam, or failure in the course. Referrals may be made to the Vice President for Academic Affairs for any party involved in academic dishonesty.

\textbf{Special accommodations}: Students who have documented disabilities as addressed by the Americans With Disabilities Act and who need any test or course materials to be furnished in an alternative format should notify me immediately (during the first week of class).  Reasonable efforts will be made to accommodate the needs of such students.

\vspace*{\stretch{1}}

\begin{center}
\emph{ {\LARGE Good luck this semester!} }
\end{center}

\vspace*{\stretch{1}}

\end{document}
