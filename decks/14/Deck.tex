\documentclass[11pt,ignorenonframetext,aspectratio=169,xcolor={svgnames}]{beamer}
\usetheme{Szeged}
\usecolortheme{crane}
\usefonttheme{structurebold}
\usefonttheme{professionalfonts}
\usepackage{amssymb,amsmath}
\usepackage{ifxetex,ifluatex}
\usepackage{fixltx2e} % provides \textsubscript
\ifxetex
  \usepackage{fontspec,xltxtra,xunicode}
  \defaultfontfeatures{Mapping=tex-text,Scale=MatchLowercase}
\else
  \ifluatex
    \usepackage{fontspec}
    \defaultfontfeatures{Mapping=tex-text,Scale=MatchLowercase}
  \else
    \usepackage[utf8]{inputenc}
  \fi
\fi
\usepackage{listings}

% Comment these out if you don't want a slide with just the
% part/section/subsection/subsubsection title:
% \AtBeginPart{
%   \let\insertpartnumber\relax
%   \let\partname\relax
%   \frame{\partpage}
% }
% \AtBeginSection{
%   \let\insertsectionnumber\relax
%   \let\sectionname\relax
%   \frame{\sectionpage}
% }
% \AtBeginSubsection{
%   \let\insertsubsectionnumber\relax
%   \let\subsectionname\relax
%   \frame{\subsectionpage}
% }

\setlength{\parindent}{0pt}
\setlength{\parskip}{6pt plus 2pt minus 1pt}
\setlength{\emergencystretch}{3em}  % prevent overfull lines
\setcounter{secnumdepth}{0}

%%% begin dwr insert
\usepackage{patchcmd}
\usepackage{tabulary}   % Support longer table cells
\usepackage{booktabs}   % Support better tables
\usepackage[sort&compress]{natbib}

\usepackage{framed}     % Allow background color for images
\definecolor{shadecolor}{named}{white}

%\usepackage{paralist}
\usepackage{xparse}
\usepackage{subfigure}
\usepackage{hyperref}
%%% end dwr insert
\usepackage{rosoff-vector-macros}
\usepackage[beamer]{cofi}
\usepackage{siunitx}
\title{Evaluating double integrals}
\author{Math 275 Multivariable Calculus}
\date{October 30, 2013}


\begin{document}
\frame{\titlepage}

\section{Evaluating double integrals}

\begin{frame}\frametitle{WeBWorK problem 14}

A pile of earth standing on flat ground has height $36$ meters. The
ground is the $(x,y)$-plane. The origin is directly below the top of the
pile and the $z$-axis is upward. The cross-section at height $z$ is
given by $x^2+y^2=36-z$ for $0 \leq z \leq 36$, with $x$, $y$, and $z$
in meters.

\begin{itemize}
\itemsep1pt\parskip0pt\parsep0pt
\item
  What equation gives the edge of the base of the pile?
\item
  What is the area of the base of the pile?
\item
  What equation gives the cross-section of the pile with the plane
  $z = 6$?
\item
  What is the area of the cross-section $z = 6$?
\item
  What is $A(z)$, the area of a horizontal cross-section at height $z$?
\item
  What is the volume of the pile?
\end{itemize}

\end{frame}

\end{document}
