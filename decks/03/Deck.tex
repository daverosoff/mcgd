\documentclass[11pt,ignorenonframetext,aspectratio=169]{beamer}
\usetheme{Szeged}
\usecolortheme{crane}
\usefonttheme{structurebold}
\usepackage{amssymb,amsmath}
\usepackage{ifxetex,ifluatex}
\usepackage{fixltx2e} % provides \textsubscript
\ifxetex
  \usepackage{fontspec,xltxtra,xunicode}
  \defaultfontfeatures{Mapping=tex-text,Scale=MatchLowercase}
\else
  \ifluatex
    \usepackage{fontspec}
    \defaultfontfeatures{Mapping=tex-text,Scale=MatchLowercase}
  \else
    \usepackage[utf8]{inputenc}
  \fi
\fi
\usepackage{listings}

% Comment these out if you don't want a slide with just the
% part/section/subsection/subsubsection title:
% \AtBeginPart{
%   \let\insertpartnumber\relax
%   \let\partname\relax
%   \frame{\partpage}
% }
% \AtBeginSection{
%   \let\insertsectionnumber\relax
%   \let\sectionname\relax
%   \frame{\sectionpage}
% }
% \AtBeginSubsection{
%   \let\insertsubsectionnumber\relax
%   \let\subsectionname\relax
%   \frame{\subsectionpage}
% }

\setlength{\parindent}{0pt}
\setlength{\parskip}{6pt plus 2pt minus 1pt}
\setlength{\emergencystretch}{3em}  % prevent overfull lines
\setcounter{secnumdepth}{0}
%%% begin dwr insert
\usepackage{patchcmd}
\usepackage{tabulary}   % Support longer table cells
\usepackage{booktabs}   % Support better tables
\usepackage[sort&compress]{natbib}

\usepackage{framed}     % Allow background color for images
\definecolor{shadecolor}{named}{white}

%\usepackage{paralist}
\usepackage{xparse}

\usepackage{subfigure}
\usepackage{hyperref}
%%% end dwr insert
\title{Introduction to R Studio}
\author{Math 275 Multivariable Calculus}
\date{September 13, 2013 }

\usepackage{siunitx}

\begin{document}
\frame{\titlepage}

\section{A brief history of R}

\begin{frame}\frametitle{Acknowledgement}

This deck borrows some background material from Dr.~Roger Peng. Check
out his YouTube page for lots of videos about how to use R (mostly for
statistics).

\end{frame}

\begin{frame}\frametitle{Development of R}

R is a variant or descendant of an earlier language called S.

\begin{itemize}[<+->]

\item
  S was developed at Bell Labs by a team led by John Chambers.
\item
  In 1976 the S project began as a set of Fortran libraries for internal
  use only.
\item
  In 1988 the system was rewritten in C: this is version 3.
\item
  In 1998 Version 4 was released and is still in use---hasn't changed
  much.
\item
  R is an implementation of the S language.
\end{itemize}

\end{frame}

\begin{frame}\frametitle{S philosophy}

\begin{itemize}[<+->]

\item
  It is possible to use S (and R) in a solely interactive way.
\item
  As a user develops her abilities, this naturally turns into
  programming.
\item
  Then the language and system become more important
\end{itemize}

This philosophy carries down to R.

\end{frame}

\begin{frame}\frametitle{R development}

\begin{itemize}[<+->]

\item
  1991: R project initiated by R. Ihaka and R. Gentleman in NZ
\item
  1995: Decision is made to license under GNU General Public License
\item
  1997: R Core Group is formed; this group controls the source code of
  R.
\item
  2000: v1.0.0
\item
  2012: v2.15.1
\item
  today: v3.0.1 (I think?)
\end{itemize}

\end{frame}

\begin{frame}\frametitle{Features of R}

\begin{itemize}[<+->]

\item
  Similar to S, easy for legacy users to switch
\item
  Runs on most any OS (even PS3)
\item
  Modular packages
\item
  Sophisticated graphics capabilities, better than most stat packages
\item
  Active user community
\end{itemize}

\end{frame}

\begin{frame}\frametitle{Free software}

\begin{itemize}[<+->]

\item
  Freedom to run the program for any purpose
\item
  Freedom to change the program for your particular needs

  \begin{itemize}[<+->]
  
  \item
    Impossible without access to the source code
  \end{itemize}
\item
  Freedom to redistribute, even for a profit

  \begin{itemize}[<+->]
  
  \item
    Usually, the requirement is that copies you make are subject to the
    same freedoms
  \end{itemize}
\item
  Freedom to improve and release patches, new packages, etc.

  \begin{itemize}[<+->]
  
  \item
    Also impossible without source code.
  \end{itemize}
\item
  R is without question one of the greatest successes of the
  free-software movement.
\item
  \href{http://fsf.org/}{The Free Software Foundation fsf.org}
\end{itemize}

\end{frame}

\begin{frame}\frametitle{Reproducible research}

R is very widely used worldwide by all kinds of data scientists. Partly,
this is because its free nature encourages the development of
\emph{reproducible research}.

\begin{quote}
Conclusions whose soundness is a consequence of various calculations
must be verifiable.
\end{quote}

\begin{itemize}[<+->]

\item
  How should the community best verify the calculations?
\item
  Well, if the paper includes all the R code and scripts used to conduct
  them\ldots{}
\item
  \ldots{}just ``run the paper'' and see for yourself.
\item
  Since even the internals of R are open and free, there is no part of
  the enterprise that is hidden from view.
\end{itemize}

\end{frame}

\begin{frame}\frametitle{Some limitations}

\begin{itemize}[<+->]

\item
  40-year old technology
\item
  Support for dynamic graphics, animation, or 3D graphics is less than
  ideal
\item
  Loads all objects into memory (this isn't really a problem for us)
\end{itemize}

\end{frame}

\begin{frame}\frametitle{Design of the R system}

\begin{itemize}[<+->]

\item
  ``base'' system, available from CRAN

  \begin{itemize}[<+->]
  
  \item
    \textbf{C}omprehensive \textbf{R} \textbf{A}rchive \textbf{N}etwork
  \item
    contains the \textbf{base} package which provides the fundamentals
  \item
    also contains packages like \textbf{utils}, \textbf{datasets},
    \textbf{graphics}, \textbf{gr}Devices, \textbf{grid},
    \textbf{methods}, \textbf{tools}, \textbf{parallel},
    \textbf{compiler}, \textbf{splines}, \textbf{tcltk},
    \textbf{stats4}.
  \end{itemize}
\item
  CRAN contains about 4000 packages developed by users around the world
\item
  there's no way of knowing how many unofficial packages have been
  developed and released via private websites, etc.
\end{itemize}

\end{frame}

\begin{frame}\frametitle{Literate programming}

``Literate programming'' is an idea of Donald Knuth. Basically, it's a
philosophy of software development.

\begin{quote}
Programs should be written so that they are simultaneously
human-readable and computer-readable.
\end{quote}

R is at the center of a developing revival of literate programming, in
the context of a cultural focus on reproducibility. Packages like
\textbf{Sweave} and \textbf{knitr} combine with rendering engines like
LaTeX and Markdown and the text conversion utility Pandoc to generate
extremely portable, flexible, and professional scientific documents with
comparatively little effort.

\end{frame}

\begin{frame}\frametitle{Getting help}

There is a very large and vibrant user community online. Chances are,
someone has already asked your question and received an answer.

Check out \href{http://youtu.be/ZFaWxxzouCY}{this video
http://youtu.be/ZFaWxxzouCY} for tips on getting help. Many of the best
forums are quite serious and your question will be closed and ignored if
it does not measure up to community standards. Make sure to follow
proper forums etiquette (check FAQs and archives before posting).

\end{frame}

\end{document}
