\documentclass[12pt]{exam}
\usepackage{cofi}
\usepackage{fourier}
\frenchspacing
\usepackage{parskip}
\usepackage{rosoff-vector-macros}
\firstpageheader{}{}{}
\runningheader{\textbf{Fall 2013}}
 {}
 {\textbf{Math 275}}
 %{\emph{Page \thepage~of \numpages}}
\runningheadrule

\pagestyle{head}
\theoremstyle{definition}
\newtheorem{example}{Example}
\begin{document}

\noindent
\textbf{{\large Math 275 \hfill Workshop 08: Introducing the double integral}}
% \hfill Name: \underline{\hspace{0.5in}Answers\hspace{2in}}

\noindent
October 18, 2013 

\noindent
Due: Not collected \hfill Name: \underline{\hspace{3in}} 

\noindent

\section{Workshop 08: Introducing the double integral}

While the double integral is defined as the limit of a Riemann sum, just
like ordinary definite integrals are, they are usually computed using
Fubini's Theorem, which tells us how to write them as
\emph{iterated integrals}. These are expressions of the form
\[ \int_a^b \left( \int_c^d f(x,y) \; dy \right) \; dx. \]

Usually, we omit the parentheses with this understood grouping of
operations. Notice that the innermost differential $dy$ is matched with
the innermost integral. A notation that is less ambiguous would be
\[ \int_{x=a}^b \left( \int_{y=c}^d f(x,y) \; dy \right) \; dx. \]

\subsection{Evaluating iterated integrals}

Since $f(x,y)$ in the above expressions is a function of $x$ and $y$, if
we perform a definite integral of this function over $x$, we get a
function of $y$ alone. It's analogous to partial differentiation: if
you're integrating $dx$, $y$ is like a constant.

\begin{questions}

\question Evaluate the iterated integral. \emph{Answer}: $40(e^4 - e^2)$

$\displaystyle\int_{x=2}^4 \left( \displaystyle\int_{y=1}^9 ye^x \; dy \right) \; dx = $

\flexskip{1}

\newpage

\question Evaluate the iterated integral. \emph{Answer}: 84

$\displaystyle\int_{2}^{6} \displaystyle\int_{1}^{4} x^2 \; dx \; dy = $

\flexskip{1}



\question Evaluate the iterated integral. \emph{Answer}: 4/3

$\displaystyle\int_{-1}^{1} \displaystyle\int_0^{\pi} x^2 \sin{y} \; dy \; dx = $

\flexskip{1}

\question Evaluate the iterated integral. 

$\displaystyle\int_{-1}^{1} \displaystyle\int_0^{\pi} x^2 \sin{y} \; dx \; dy = $

\flexskip{1}

\end{questions} 

\end{document}