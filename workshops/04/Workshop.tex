\documentclass[12pt]{exam}
\usepackage{cofi}
\usepackage{fourier}
\frenchspacing
\usepackage{parskip}
\firstpageheader{}{}{}
\runningheader{\textbf{Fall 2014}}
 {}
 {\textbf{Math 275}}
 %{\emph{Page \thepage~of \numpages}}
\runningheadrule

\pagestyle{head}
\extrawidth{1in}
\extraheadheight[-0.6in]{-0.1in}
\extrafootheight{-0.5in}
\begin{document}
\noindent
\textbf{{\large Math 275 \hfill Workshop 04: Limits and continuity for several variables}}
% \hfill Name: \underline{\hspace{0.5in}Answers\hspace{2in}}

\noindent
\makebox[\textwidth]{September 20, 2013 \hfill}

\noindent
\makebox[\textwidth]{Due: Monday, September 30 \hfill Name: \underline{\hspace{3in}}}

\noindent

\newtheorem{fact}{Fact}

\section{Workshop 04: Limits and continuity}

The existence of limits is more complicated for functions of several
variables. In this workshop, you will investigate some techniques for
\emph{ruling out} their existence. As in the one-variable case,
establishing the value of a particular limit can be tricky in general
and there is no one procedure for doing it.

\subsection{Ruling out: by restriction}

\begin{questions}

\question Let $f_1(x,y) = x^2/(x^2 + y^2)$. Find the limit of $f_1(x,y)$ as $(x,y) \to (0,0)$ along the $x$-axis. In other words, find the (ordinary, 1-dimensional) limit of a suitable "slice function" obtained from $f_1$.

\flexskip{2}

\question Find the limit of $f_1(x,y)$ as $(x,y) \to (0,0)$ as $(x,y) \to (0,0)$ along the $y$-axis, again using a slice function.

\flexskip{2}

\question Does $\lim\limits_{(x,y) \to (0,0)} f_1 (x,y)$ exist? Why do you think so?

\flexskip{1}

\newpage

\question Now let $f_2(x,y) = xy/(x^2 + y^2)$. Show that the limit of $f_2(x,y)$ as $(x,y) \to (0,0)$ along either coordinate axis is 0.

\flexskip{2}

\question Find the limit of $f_2(x,y)$  as $(x,y) \to (0,0)$ along the line $y = x$.

\flexskip{2}

\question Does $\lim\limits_{(x,y) \to (0,0)} f_2 (x,y)$ exist? Why do you think so?

\flexskip{1}

\question Finally, let $f_3(x,y) = x^2 y/(x^4 + y^2)$. Show that the limit as $(x,y) \to (0,0)$ along every line through the origin is $0$.

\flexskip{2}

\question Show that, nevertheless, $\lim\limits_{(x,y) \to (0,0)} f_3 (x,y)$ does not exist by letting $(x,y)$ approach the origin along a suitable curve.

\flexskip{1}

\end{questions} 

\end{document}