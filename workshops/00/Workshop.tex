\documentclass[12pt]{exam}
\usepackage{cofi}
\usepackage{fourier}
\frenchspacing
\usepackage{parskip}
\firstpageheader{}{}{}
\runningheader{\textbf{Fall 2014}}
 {}
 {\textbf{Math 275}}
 %{\emph{Page \thepage~of \numpages}}
\runningheadrule

\pagestyle{head}
\extrawidth{1in}
\extraheadheight[-0.6in]{-0.1in}
\extrafootheight{-0.5in}
\begin{document}
%\setromanfont{Palatino Linotype}
\noindent
\textbf{{\large Math 275 \hfill Workshop 00}}
% \hfill Name: \underline{\hspace{0.5in}Answers\hspace{2in}}

\noindent
\makebox[\textwidth]{September 5, 2014 \hfill}

\noindent
\makebox[\textwidth]{Due: End of class today \hfill Name: \underline{\hspace{3in}}} 

\section{Coordinates revisited}

We'll do these once or twice a week. They are usually collected a
day later. This one will be collected at the end of class.

\subsection{Warm-up}

The first major topic in this course is the geometry of 3-dimensional
Euclidean space $\RR^3$. Many people struggle with visualizing in
this space or find it difficult to translate coordinate information into
geometric information. This workshop will encourage you to rethink
coordinates in the familiar Euclidean plane in preparation for the later
discussion.

\begin{compactitem}
    \item As you work through the questions, check your answers with either Chelsea or Dr.\ Rosoff before going on. 
    \item Raise a hand if you are really stuck. 
    \item Talk a lot! Class goes better the more you talk to each other.
    \item Get up and work in a group at the board! There's plenty of room, 
    and it is easier to see what others are up to.
\end{compactitem}

\subsection{Coordinates}

Draw a coordinate plane, and plot the line $y = 2$.

\begin{questions}

\question
What do all the points on the line $y = 2$ have in common? Answer as \emph{clearly} and \emph{succinctly} as you can. (Try to avoid referring to points' coordinates. The idea here is to remember how coordinates are defined in the first place.)
\flexskip{1}
\question
Give the best definition you can for the idea of ``distance between a point and a line''.
\flexskip{2}
\question
What do all the points on the line $x = -3$ have in common? What's the difference between $x = -3$ and $x = 3$?
\flexskip{1}
\newpage
\question 
How do you plot points? Where's the point $(2, 3)$? What does this point have to do with the lines you have been drawing?
\flexskip{1}
\question
Now draw yourself a 3-space, like on the board. What are the points $z = -1$? What shape do they make? Try to draw this shape.
\flexskip{1}
\question 
Draw $y = -2$ in your 3-space. What's the difference between $y = -2$ and $y = 2$? Now Try to draw $x = 4$ in your 3-space. Where does $(4, 2, -1)$ live?
\flexskip{1}
\question 
Given two points $(x_0, y_0, z_0)$ and $(x_1, y_1, z_1)$, write down a formula for the square of the distance between them. Use the definitions of the coordinates you've figured out. If you are having trouble, try the exercise in just 2 dimensions first.
\flexskip{1}
\end{questions} 

\end{document}