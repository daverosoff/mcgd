\documentclass[11pt,ignorenonframetext,aspectratio=169,xcolor={svgnames}]{beamer}
\usetheme{Szeged}
\usecolortheme{crane}
\usefonttheme{structurebold}
\usefonttheme{professionalfonts}
\usepackage{amssymb,amsmath}
\usepackage{ifxetex,ifluatex}
\usepackage{fixltx2e} % provides \textsubscript
\ifxetex
  \usepackage{fontspec,xltxtra,xunicode}
  \defaultfontfeatures{Mapping=tex-text,Scale=MatchLowercase}
\else
  \ifluatex
    \usepackage{fontspec}
    \defaultfontfeatures{Mapping=tex-text,Scale=MatchLowercase}
  \else
    \usepackage[utf8]{inputenc}
  \fi
\fi
\usepackage{listings}

% Comment these out if you don't want a slide with just the
% part/section/subsection/subsubsection title:
% \AtBeginPart{
%   \let\insertpartnumber\relax
%   \let\partname\relax
%   \frame{\partpage}
% }
% \AtBeginSection{
%   \let\insertsectionnumber\relax
%   \let\sectionname\relax
%   \frame{\sectionpage}
% }
% \AtBeginSubsection{
%   \let\insertsubsectionnumber\relax
%   \let\subsectionname\relax
%   \frame{\subsectionpage}
% }

\setlength{\parindent}{0pt}
\setlength{\parskip}{6pt plus 2pt minus 1pt}
\setlength{\emergencystretch}{3em}  % prevent overfull lines
\setcounter{secnumdepth}{0}

%%% begin dwr insert
\usepackage{patchcmd}
\usepackage{tabulary}   % Support longer table cells
\usepackage{booktabs}   % Support better tables
\usepackage[sort&compress]{natbib}

\usepackage{framed}     % Allow background color for images
\definecolor{shadecolor}{named}{white}

%\usepackage{paralist}
\usepackage{xparse}
\usepackage{subfigure}
\usepackage{hyperref}
%%% end dwr insert
\usepackage{rosoff-vector-macros}
\usepackage[beamer]{cofi}
\title{Introduction to flux; conservative vector fields}
\author{Math 275 Multivariable Calculus}
\date{November 22, 2013}


\begin{document}
\frame{\titlepage}

\section{Flux, conservativity, and the idea of Green's theorem}

\begin{frame}\frametitle{Warm-up}

(problem 2 from WeBWorK 13)

\end{frame}

\begin{frame}\frametitle{Integration along vs.~across}

The vector line integral $\int_{\mathcal{C}} \vec{F} \cdot d\vec{s}$ is
the total contribution of the vector field at each point along the
curve. The phrase ``along the curve'' is very important to the
interpretation, because we use the tangent vectors to the curve to
define the integral. The particulars have been stated here for curves in
$\RR^2$; see the text for the complete details.

We can also define a vector line integral using \emph{normal vectors to
the curve} if we confine ourselves to $\RR^2$. Thus we're somehow
measuring the contribution of the vector field \emph{across} the curve
instead of \emph{along} it.

The resulting integral computes what is called the ``flux'' of the
vector field across the curve.

\end{frame}

\begin{frame}\frametitle{What is flux?}

Loosely speaking it's the ``amount'' of something flowing across the
curve. Its value depends on the strength of the field, the length of the
curve, and how they are oriented relative to one another.

A flux integral is similar to the vector line integrals considered
previously, which we might call ``flow'' integrals in comparison. The
difference is that we use a unit \emph{normal} vector where in the flow
integral we used a unit \emph{tangent} vector.

\begin{equation*}
    \text{Flux across $\mathcal{C}$} = \int_{\mathcal{C}} (\vec{F} \cdot \vec{e}_n) \; ds = \int_{\mathcal{C}} \vec{F}(\vec{r}(t)) \cdot \vec{n}(t) \; dt,
\end{equation*}

where $\vec{n}(t) = \angl{-y'(t), x'(t)}$ and
$\vec{e}_n = \vec{n}/\norm{\vec{n}}$.

\end{frame}

\begin{frame}\frametitle{Normal vectors and orientation}

The unit tangent vector to a parametrized curve is \emph{canonical},
meaning there is only one sensible choice. This is the normalized
derivative $\vec{r}'(t)/\norm{\vec{r}'(t)}$. However, there are two good
choices for the unit normal vector at each point. The choice
$\vec{n} = \angl{-y(t), x(t)}$ is not canonical, but a convention.

\begin{quote}
If $\mathcal{C}$ is a closed curve parametrized by $\vec{r}(t)$ in such
a way that the ``inside'' of $\mathcal{C}$ is always to the left, then
$\vec{n}$ points outward from the region enclosed by $\mathcal{C}$.
\end{quote}

We leave this topic and the idea of flux for now.

\end{frame}

\begin{frame}\frametitle{Conservative fields}

One source of many examples of vector fields are the \emph{gradient}
vector fields. Choose your favorite smooth function
$V \colon \RR^2 \to \RR$ (or $\RR^3 \to \RR$). Then $\nabla V$ is a vector
field on $\RR^2$ (or $\RR^3$).

Most vector fields can be seen not to be gradients. In fact, gradients
have a very special property. Let us call such vector fields
``conservative''. If the vector field $\vec{F}$ is conservative, then
there is some function $V$ such that $\nabla V = \vec{F}$. Such a
function is called a \emph{potential function} for $\vec{F}$.

\end{frame}

\begin{frame}\frametitle{Integration of conservative fields}

If $\vec{F}$ admits a potential function, then line integrals of
$\vec{F}$ are easy to compute.

\begin{quote}
\emph{Theorem} (Fundamental theorem for conservative vector fields).
Assume that $\vec{F} = \nabla V$ throughout some domain $\mathcal{D}$.
Then for any points $P$ and $Q$ in $\mathcal{D}$ and any path
$\vec{r}(t)$ from $P$ to $Q$,
\[\int_{\vec{r}} \vec{F} \cdot d\vec{s} = V(Q) - V(P).\]
\end{quote}

In particular, the value of the line integral depends only on the pair
$(P,Q)$ and not on the path connecting them. Fields with this property
are called ``path-independent''. (Easy proof on p.~963.)

\end{frame}

\begin{frame}\frametitle{Circulation of conservative fields}

Recall that for a closed curve $\vec{r}(t)$ (that is, a loop; a curve
for which $\vec{r}(a) = \vec{r}(b)$), we define the \emph{circulation}
of $\vec{F}$ around $\vec{r}$ to be the line integral of $\vec{F}$
around the curve.

Evidently, path-independent fields have zero circulation around any
closed curve, since we may choose $P = Q$ for such a curve and then

\[ \oint_{\mathcal{C}} \vec{F} \cdot d\vec{s} = V(Q) - V(Q) = 0. \]

\end{frame}

\end{document}
