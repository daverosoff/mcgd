\documentclass[12pt]{article}

\usepackage[margin=1.0in, top=0.5in]{geometry}
%\setlength{\parskip}{1ex}
%\setlength{\parindent}{0pt}
\usepackage{parskip}

\usepackage{rosoff-vector-macros}
\usepackage{rosoff}
\usepackage{fourier, MnSymbol}
\usepackage{enumerate, paralist}
\title{12.3.48}

\date{2013-10-11}
\begin{document}
\thispagestyle{empty}
\maketitle

Assume that $\uu$ and $\vv$ are unit vectors with
$\norm{\uu + \vv} = \sqrt{3/2}$. We calculate $\norm{2\uu - 3\vv}$ using
some properties of the dot product.

Observe that $\norm{2\uu - 3\vv}^2 = (2\uu - 3\vv) \cdot (2\uu - 3\vv)$.
We expand using the bilinearity of dot product:

\begin{align*}
    \norm{2\uu - 3\vv}^2 &= (2\uu - 3\vv) \cdot (2\uu - 3\vv) \\
         &= 4 \uu \cdot \uu - 6 \vv \cdot \uu - 6 \uu \cdot \vv + 9 \vv \cdot \vv \\
         &= 4 \norm{\uu}^2 - 12 \uu \cdot \vv + 9 \norm{\vv}^2,
\end{align*}

where in the last step we use the identity
$\ww \cdot \ww = \norm{\ww}^2$ and the commutativity of dot product.

Since $\norm{\uu} = \norm{\vv} = 1$, we have shown that

\begin{equation*}
    \norm{2\uu - 3\vv}^2 = 13 - 12 \uu \cdot \vv.
\end{equation*}

Therefore, if we can compute $\uu \cdot \vv$, we will be done.

We are given that $\norm{\uu + \vv}^2 = 3/2$. This means that

\begin{align*}
3/2 &= (\uu + \vv) \cdot (\uu + \vv) \\
    &= \norm{\uu}^2 + 2 \uu \cdot \vv + \norm{\vv}^2 \\
    &= 2 + 2 \uu \cdot \vv, 
\end{align*}

since $\uu$ and $\vv$ are unit vectors. This shows that
$\uu \cdot \vv = -1/4$.

Putting it all together, we have

\begin{equation*}
    \norm{2\uu - 3\vv} = \sqrt{13 - 12 \uu \cdot \vv} = \sqrt{13 + 3} = 4.
\end{equation*}

\end{document}