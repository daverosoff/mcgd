\documentclass[12pt]{amsart}
\usepackage{enumerate,graphicx,wrapfig,amsmath,amsthm,amssymb,latexsym}%,wrapfig,floatflt}
\usepackage{hyperref}
%\usepackage[matrix,arrow,cmtip,curve,dvips]{xy}
\DeclareGraphicsExtensions{.png}
\DeclareMathOperator{\im}{im}
\newcommand{\id}[1][{}]{\mathrm{Id}_{#1}}
\newcommand{\Q}{\mathbf Q}      % adds blackboard bold macros for
\newcommand{\R}{\mathbf R}      % conventional notation for
\newcommand{\Z}{\mathbf Z}      % rationals, reals, integers,
\newcommand{\C}{\mathbf C}      % complexes, naturals, field of p
\newcommand{\N}{\mathbf N}      % elements, p-adic integers
\newcommand{\D}{\mathbf D}
\newcommand{\Zp}{\mathbb{Z}_p}      % and numbers.
\newcommand{\Qp}{\mathbb{Q}_p}
\newcommand{\Fp}{\mathbb{F}_p}
\newcommand{\ie}{\emph{i.e.}}
\newcommand{\eg}{\emph{e.g.}}
\newcommand{\ind}[2]{[#1:#2]}       % index of groups, degree of field
\newcommand{\aut}[2][{}]{\mathrm{Aut}_{#1}(#2)}  % automorphism group
\newcommand{\isom}{\cong}       % isomorphic
\newcommand{\fk}[1]{\mathfrak{#1}}  % fraktur shorthand
\newcommand{\abs}[2][{}]{|#2|_{#1}} % generalized absolute value
\newcommand{\ord}[2][{}]{v_{#1}(#2)}    % roman ord(x), ord_p(x)
\newcommand{\Ex}{\begin{ex}}
\newcommand{\Eex}{\end{ex}}
\newcommand{\tr}[1]{#1^t}
\newcommand{\mat}[2]{M_{#1}(#2)}
\newcommand{\GL}[2]{\mathrm{GL}_{#1}(#2)}
\newcommand{\Lie}[1]{\mathrm{Lie}(#1)}
\newcommand{\ddt}[1]{\left. \frac{\mathrm{d}}{\mathrm{d}t} \right|_{t=#1}}
\newcommand{\ten}{\otimes}
\newcommand{\mg}[1]{{#1}^{\times}}
\newcommand{\wt}[1]{\widetilde{#1}}
\newcommand{\Tor}[2]{\mathrm{Tor}^{#1}_{#2}}

\theoremstyle{plain}            % define numbered and naked theorem
\newtheorem{ntheorem}{Theorem}[section]     % environments, numbered
\newtheorem{nlemma}[ntheorem]{Lemma}        % consecutively within
\newtheorem{nprop}[ntheorem]{Proposition}   % chapters
\newtheorem{ncor}[ntheorem]{Corollary}
\newtheorem*{theorem}{Theorem}
\newtheorem*{lemma}{Lemma}
\newtheorem*{prop}{Proposition}
\newtheorem*{cor}{Corollary}

% \newcounter{ntheorem}

\theoremstyle{definition}
\newtheorem{ndefn}[ntheorem]{Definition}
\newtheorem{nex}[ntheorem]{Example}
\newtheorem*{defn}{Definition}
\newtheorem*{ex}{Example}

%\theoremstyle{remark}
\newtheorem*{rmk}{Remark}
\newtheorem*{ntn}{Notation}

\usepackage[total={7.0in,10.0in},left=1.0in,right=1.0in]{geometry}
\setlength{\parskip}{7.5pt} \setlength{\parindent}{0pt}
%\pretolerance=4000 \setlength{\topmargin}{-1.0in}
%\setlength{\textheight}{10.0in} \setlength{\textwidth}{7in}
%\setlength{\headheight}{26pt} \setlength{\headsep}{8pt}
%\setlength{\oddsidemargin}{-0.25in}
%\setlength{\evensidemargin}{-0.25in}
\title{Mathematics 251, Fall 2011 \\ Review}
\pagestyle{empty}
\begin{document}
\maketitle
\thispagestyle{empty}

Here is a list of equations and formulas you may want to include on your exam note sheet. It is not intended to be exhaustive.

First, you should know how to draw pictures of the following plane and solid figures, given equations describing their points.

\begin{itemize}
    \item Lines
    \item Planes
    \item Parabolas (opening in any of the four plane coordinate directions)
    \item Circles
    \item Spheres and hemispheres
    \item Elliptic paraboloids opening in a coordinate direction with circular cross-section
    \item Parallelograms and triangles; parallelepipeds (skew boxes in $\R^3$)
\end{itemize}

Standard parametrizations of lines, line segments, circles, and graphs of functions $y = f(x)$ or $x = g(y)$ should also be included in your notes if not engraved on your brain.

\begin{enumerate}
\item Chapter 11
\begin{itemize}
    \item Arclength and speed formulas on parametrized curves
    \item Polar coordinate transformation formulas
\end{itemize}
\item Chapter 12
\begin{itemize}
    \item Length of vectors (in coordinates and in terms of dot product)
    \item Normalization formula converting a nonzero vector to a unit vector pointing in the same direction
    \item Triangle inequality
    \item Distance formula
    \item Parametric equations of a line
    \item Dot product--cosine formula
    \item Dot product of orthogonal vectors
    \item Vector projection formula (component of $\vec{u}$ along $\vec{v}$)
    \item Cross product formula
    \item Geometric properties of cross product
    \item Area and volume of skew boxes via cross product
    \item Equation of plane in point--normal vector form and linear form
    \item Cylindrical and spherical coordinate transformation formulas (I would also include pictures indicating the meaning of the coordinates)
\end{itemize}
\item Chapter 13
\begin{itemize}
    \item Product rule for dot and cross products
    \item Chain/product rule for products of vector and scalar functions
    \item Parametric equation for the tangent line to a parametrized curve
    \item Arc length of parametrized curve
    \item Meaning of arc length parametrization (I won't ask you to \emph{find} them, but you need to know how to use them)
    \item Unit tangent vector to a parametrized curve
    \item Three formulas for curvature (in addition to the definition)
    \item Unit normal vector, binormal vector
    \item Relation between position, velocity, and acceleration vectors
    \item Solve vector differential equations of motion (Ch.\ 13.5) for gravitational and uniform circular acceleration
    \item Tangential and normal components of acceleration
\end{itemize}
\item Chapter 14
\begin{itemize}
    \item Traces and contour lines of a function of two variables
    \item Level surfaces of functions of three variables
    \item Continuity; composition of continuous functions is continuous
    \item Limits; how to rule out their existence using special paths; how to find them with the Squeeze Theorem
    \item Partial derivatives; gradient
    \item Chain rule (3 forms: for gradients; for paths; general)
    \item Clairaut's Theorem on the equality of mixed partials
    \item Differentiability: definition in terms of local linearity
    \item Linearization: what is its graph? What is its formula? What is its significance?
    \item Criterion for differentiability
    \item Directional derivatives and computation via gradient--dot product
    \item Interpretation of the gradient vector
    \item Local extreme values; critical points; how to find absolute extrema on closed and bounded regions
    \item Lagrange multipliers
\end{itemize}
\item Chapter 15
\begin{itemize}
    \item Double integrals over general regions; vertical and horizontal simplicity \& Fubini's theorem
    \item Mean value theorem for double integrals
    \item Decomposition of region of integration into subregions
    \item All that stuff also for triple integrals
    \item Integration in general coordinates: the change of variable theorem and its special cases in polar, cylindrical, and spherical coordinates
    \item Jacobian determinant: how to calculate it
\end{itemize}
\item Chapter 16
\begin{itemize}
    \item Vector fields; decomposition into several real-valued component functions
    \item Cross partial condition for recognizing gradients
    \item Potential functions; uniqueness
    \item Line integrals of scalar (real-valued) and vector functions using parametrizations
    \item Line integrals, force, and work
    \item Conservative vector fields; do they have potentials? Relation with gradients
    \item Cross-partials and conservatism
\end{itemize}
\end{enumerate} 

\end{document}