\documentclass[11pt,ignorenonframetext,aspectratio=169,xcolor={svgnames}]{beamer}
\usetheme{Szeged}
\usecolortheme{crane}
\usefonttheme{structurebold}
\usefonttheme{professionalfonts}
\usepackage{amssymb,amsmath}
\usepackage{ifxetex,ifluatex}
\usepackage{fixltx2e} % provides \textsubscript
\ifxetex
  \usepackage{fontspec,xltxtra,xunicode}
  \defaultfontfeatures{Mapping=tex-text,Scale=MatchLowercase}
\else
  \ifluatex
    \usepackage{fontspec}
    \defaultfontfeatures{Mapping=tex-text,Scale=MatchLowercase}
  \else
    \usepackage[utf8]{inputenc}
  \fi
\fi
\usepackage{listings}

% Comment these out if you don't want a slide with just the
% part/section/subsection/subsubsection title:
% \AtBeginPart{
%   \let\insertpartnumber\relax
%   \let\partname\relax
%   \frame{\partpage}
% }
% \AtBeginSection{
%   \let\insertsectionnumber\relax
%   \let\sectionname\relax
%   \frame{\sectionpage}
% }
% \AtBeginSubsection{
%   \let\insertsubsectionnumber\relax
%   \let\subsectionname\relax
%   \frame{\subsectionpage}
% }

\setlength{\parindent}{0pt}
\setlength{\parskip}{6pt plus 2pt minus 1pt}
\setlength{\emergencystretch}{3em}  % prevent overfull lines
\setcounter{secnumdepth}{0}

%%% begin dwr insert
\usepackage{patchcmd}
\usepackage{tabulary}   % Support longer table cells
\usepackage{booktabs}   % Support better tables
\usepackage[sort&compress]{natbib}

\usepackage{framed}     % Allow background color for images
\definecolor{shadecolor}{named}{white}

%\usepackage{paralist}
\usepackage{xparse}
\usepackage{subfigure}
\usepackage{hyperref}
%%% end dwr insert
\usepackage{rosoff-vector-macros}
\usepackage[beamer]{cofi}
\title{Practice with parametrizing lines}
\author{Math 251 Calculus 3}
\date{November 13, 2013}


\begin{document}
\frame{\titlepage}

\section{Practice with parametrizing lines}

\begin{frame}\frametitle{Starting point and direction vector}

Just like the standard form of an equation of a plane, the easiest way
to express a line involves a vector. Instead of a normal vector (which,
in a way, tells which direction the plane \emph{doesn't} go), we'll use
a vector $\vv$ \emph{parallel} to the line, called its direction vector.

If $(x_0, y_0, z_0)$ is a point in $\RR^3$, there is exactly one line
passing through it parallel to $\vv$. Evidently, these points are all
obtained by adding multiples of $\vv$ to the vector
$\vec{r}_0 = \angl{x_0, y_0, z_0}$.

\end{frame}

\begin{frame}\frametitle{Parametrizing the line}

A multiple of $\vv$ is a vector $t \vv$, where $t \in \RR$. We usually
think of $t$ as a time parameter and the expression

\begin{equation} \label{eq:vecline}
\vec{r}(t) = \vec{r}_0 + t \vv 
\end{equation}

as a moving point. Really it is a moving \emph{vector}; we imagine it in
standard position, so that the tail of $\vec{r}(t)$ is fixed at
$(0,0,0)$ while the head traces the line through $\vec{r}_0$ parallel to
$\vv$.

\end{frame}

\begin{frame}\frametitle{The scalar form of the equations}

If we write $\vv$ in coordinates $\vv = \angl{a,b,c}$, then we can
decompose Equation \ref{eq:vecline} into a system of scalar equations.
Write
\begin{equation*}
\vec{r}(t) = \angl{x_0 + at, y_0 + bt, z_0 + ct} = \angl{x(t), y(t), z(t)},
\end{equation*}
to obtain three equations:
\begin{equation*}
    x(t) &= x_0 + at, \quad y(t) &= y_0 + bt, \quad z(t) &= z_0 + ct.
\end{equation*}
These are evidently equivalent to Equation \ref{eq:vecline} and are
called parametric equations of the line.

\end{frame}

\begin{frame}\frametitle{Work together}

Using the whiteboards, find vector equations for all of the following
lines and scalar (parametric) equations for at least one of them.

\begin{enumerate}
\def\labelenumi{\arabic{enumi}.}

\item
  The line passing through $(-5, 6, -1)$ parallel to $\angl{9, 0, -6}$.
\item
  The line passing through $(3, 2, -3)$ and $(-1, 4, 2)$.
\item
  The line passing through $(3, 2, -3)$ orthogonal to the plane
  $2x-y-z=3$.
\end{enumerate}

\end{frame}

\begin{frame}\frametitle{Answers}

\begin{enumerate}
\def\labelenumi{\arabic{enumi}.}

\item
  $\vec{r}(t) = \angl{-5 + 9t, 6, -1 - 6t}$.
\item
  $\vec{r}(t) = \angl{3, 2, -3} + t\angl{-4, 2, 5}$.
\item
  $\vec{r}(t) = \angl{3, 2, -3} + t\angl{2, -1, -1}$.
\end{enumerate}

\end{frame}

\begin{frame}\frametitle{Circles}

The other kind of curve you should know how to parametrize is a circle.
A circle in the plane determined by its center $(x_0, y_0)$ and its
radius $r$. One parametrization is

\[\vec{r}(t) = \angl{x_0, y_0} + r \angl{\cos t, \sin \t}.\]

Write down parametrizations for the following.

\begin{enumerate}
\def\labelenumi{\arabic{enumi}.}

\item
  Unit circle
\item
  Radius $1$, center $(1,0)$
\item
  Radius $10$, center $(-3, -2)$
\end{enumerate}

\end{frame}

\begin{frame}\frametitle{Pasting parametrizations}

Often, we will be interested in parametrizing curves made up of line
segments and circular arcs. For example, consider the pseudotriangle
whose boundary is the segment connecting $(0,0)$ to $(1,0)$, the segment
from $(0,1)$ to $(0,0)$, and the arc of the unit circle lying in the
first quadrant (which connects $(0,1)$ to $(1,0)$). Try to find a
piecewise function $\vec{r}(t)$ that traces once around this path.

Start by parametrizing each boundary segment separately.

\end{frame}

\end{document}
