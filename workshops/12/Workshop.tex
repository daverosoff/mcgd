\documentclass[12pt]{exam}
\usepackage{cofi}
\usepackage{fourier}
\frenchspacing
\usepackage{parskip}
\usepackage{rosoff-vector-macros}
\firstpageheader{}{}{}
\runningheader{\textbf{Fall 2013}}
 {}
 {\textbf{Math 275}}
 %{\emph{Page \thepage~of \numpages}}
\runningheadrule

\pagestyle{head}
\theoremstyle{definition}
\newtheorem{example}{Example}
\begin{document}

\noindent
\textbf{{\large Math 275 \hfill Workshop 12: Helpful parametrizations and a vector field}}
% \hfill Name: \underline{\hspace{0.5in}Answers\hspace{2in}}

\noindent
November 18, 2013 

\noindent
Due: November 20, 2013 \hfill Name: \underline{\hspace{3in}} 

\noindent

\section{Workshop 12: Helpful parametrizations and a vector field}

\begin{questions}

\question Let $A = (2,2)$ and let $D = (4,6)$. Let $B$ and $C$ be the other two corners of the rectangle whose opposite corners are $A$ and $D$. Draw a figure and label it clearly.

\begin{parts}

    \part Parametrize the piecewise linear path from $A$ to $D$ passing through $B$.
    \part Parametrize the piecewise linear path from $A$ to $D$ passing through $C$.

\end{parts}

\flexskip{1}

\question Draw the following vector fields on the back of this page. Make \emph{big}, \emph{beautiful} pictures. Each picture should include at least 10 vectors.

\begin{parts}

    \part $\vec{F} = \angl{1, 0}$
    \part $\vec{G} = \angl{0,-2}$
    \part $\vec{H} = \angl{x,0}$
    \part $\vec{R} = \angl{0,x}$
    \part $\vec{S} = \angl{x,y}$
    \part $\vec{T} = \angl{-y,x}$

\end{parts}

\end{questions} 

\end{document}