\documentclass[12pt]{amsart}
\usepackage[total={7.0in,9.0in},left=1.0in,right=1.0in]{geometry}
\title{Addendum to presentation of 14.4.42a--c}
\author{Dave Rosoff}
\begin{document}
\maketitle

Victoria and Melissa wanted to show that, if $g(x,y) = 2xy(x+y)/(x^2+y^2)$, then 
\[|g(x,y)|~\leq~|x|+|y|.\] They were tripped up by the malicious $\pm$ sign. I will attempt to reproduce their calculation, up to a certain point, and then suggest an improvement.

We will show, as they attempted, that 
\[
    \left| \frac{2xy}{x^2+y^2} \right| \leq 1.
\] 
Begin by observing that $(x \pm y)^2 \geq 0$, since all squares of real numbers are nonnegative. Expanding the left-hand side yields
\begin{align}\label{eq:eq1}
    x^2 \pm 2xy + y^2 &\geq 0 \quad \text{and hence} \notag \\
    x^2 + y^2 &\geq \mp 2xy.
\end{align}
The convention regarding the signs $\pm$ and $\mp$ is that, if we choose the plus sign for each $\pm$, we are also agreeing to choose the \emph{minus} sign for each $\mp$. It is not necessary to use these signs. We could analyze each case separately and assemble our conclusion that way. I am only trying to indicate an efficient and correct way to organize the calculation along the lines of our presenters' attempt. Where we diverge is at the last displayed equation. In the presenters' calculation, the minus sign stayed with $(x^2+y^2)$, like this:
\begin{equation}\label{eq:presenter}
    \pm 2xy \geq -(x^2+y^2). 
\end{equation}
This is correct, but it is harder to proceed from here algebraically. Consider Equation~\ref{eq:eq1} first. We wish to take absolute values. Now $x^2 + y^2$ is nonnegative and hence coincides with its absolute value. On the other hand, $|2xy|$ is equal to either $2xy$ or $-2xy$. In either case, then, taking absolute values in Equation~\ref{eq:eq1} yields
\begin{align*}
    x^2 + y^2 &\geq |2xy| \quad \text{and hence} \\
    1 &\geq \frac{|2xy|}{x^2+y^2} = \left| \frac{2xy}{x^2+y^2} \right|,
\end{align*}
as required. Continuing where our presenters left off, at Equation~\ref{eq:presenter}, we now see how to proceed. Taking absolute values here gives us what we want as well; it is only different in that $|-(x^2+y^2)| = x^2+y^2$. Hence the inequality flips and we still get what we want.

I hope this clarifies the situation. Please write me an email or come see me if not. The rest of the argument is as follows.

Having shown that $|2xy/(x^2+y^2)| \leq 1$, we return to the function $g(x,y)$. Observe that
\begin{align*}
    |g(x,y)| &= \left| \frac{2xy(x+y)}{x^2+y^2} \right| &= \left| \frac{2xy}{x^2+y^2} \right| |x+y|, &\quad \text{so that} \\
    |g(x,y)| &\leq |x+y| & &\quad \text{by the previous calculation; then} \\
    |g(x,y)| &\leq |x| + |y| & &\quad \text{by the triangle inequality.}
\end{align*} 
This completes the proof that $|g(x,y)| \leq |x| + |y|$.

\end{document}