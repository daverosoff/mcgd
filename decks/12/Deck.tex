\documentclass[11pt,ignorenonframetext,aspectratio=169,xcolor={svgnames}]{beamer}
\usetheme{Szeged}
\usecolortheme{crane}
\usefonttheme{structurebold}
\usefonttheme{professionalfonts}
\usepackage{amssymb,amsmath}
\usepackage{ifxetex,ifluatex}
\usepackage{fixltx2e} % provides \textsubscript
\ifxetex
  \usepackage{fontspec,xltxtra,xunicode}
  \defaultfontfeatures{Mapping=tex-text,Scale=MatchLowercase}
\else
  \ifluatex
    \usepackage{fontspec}
    \defaultfontfeatures{Mapping=tex-text,Scale=MatchLowercase}
  \else
    \usepackage[utf8]{inputenc}
  \fi
\fi
\usepackage{listings}

% Comment these out if you don't want a slide with just the
% part/section/subsection/subsubsection title:
% \AtBeginPart{
%   \let\insertpartnumber\relax
%   \let\partname\relax
%   \frame{\partpage}
% }
% \AtBeginSection{
%   \let\insertsectionnumber\relax
%   \let\sectionname\relax
%   \frame{\sectionpage}
% }
% \AtBeginSubsection{
%   \let\insertsubsectionnumber\relax
%   \let\subsectionname\relax
%   \frame{\subsectionpage}
% }

\setlength{\parindent}{0pt}
\setlength{\parskip}{6pt plus 2pt minus 1pt}
\setlength{\emergencystretch}{3em}  % prevent overfull lines
\setcounter{secnumdepth}{0}

%%% begin dwr insert
\usepackage{patchcmd}
\usepackage{tabulary}   % Support longer table cells
\usepackage{booktabs}   % Support better tables
\usepackage[sort&compress]{natbib}

\usepackage{framed}     % Allow background color for images
\definecolor{shadecolor}{named}{white}

%\usepackage{paralist}
\usepackage{xparse}
\usepackage{subfigure}
\usepackage{hyperref}
%%% end dwr insert
\usepackage{rosoff-vector-macros}
\usepackage[beamer]{cofi}
\usepackage{siunitx}
\title{Lagrange multipliers}
\author{Math 275 Calculus 3}
\date{October 16, 2013}


\begin{document}
\frame{\titlepage}

\section{Lagrange multipliers}

\begin{frame}\frametitle{Setup}

We wish to find optimum values of some \emph{objective function}
$f(x,y,z)$ subject to the \emph{constraint} $g(x,y,z) = 0$.

Last time, you worked on an example of building a box of volume
$\SI{12}{m^3}$ of minimum cost, if the front costs $\$10$ per square
meter and the rest of the faces $\$5$ per square meter.

\end{frame}

\begin{frame}\frametitle{Approach via multiplier}

Each triple $(x,y,z)$ of real numbers corresponds to a box. Not all of
the triples correspond to realistic boxes (negative or zero volumes).
The constraint that the volume be 12 restricts the set of boxes of
interest. We only care about boxes for which $xyz = 12$; a suitable
constraint function would be $g(x,y,z) = xyz - 12$.

Here, the cost function is $C(x, y, z) = 10xy + 10xz + 15yz$.

We saw yesterday that we should find a level surface of $C$ that is
tangent to the constraint surface. We recognize this tangency by
comparing $\nabla C$ with $\nabla g$.

\end{frame}

\begin{frame}\frametitle{The Lagrange equations}

Equating two gradients gives us three scalar equations. We must try to
solve $\nabla C = \lambda \nabla g$, or

\begin{align*}
    10y + 10z &= \lambda yz \\
    10x + 15z &= \lambda xz \\
    10x + 15y &= \lambda xy,
\end{align*}

along with the original constraint equation $xyz = 12$.

The parameter $\lambda$ is the Lagrange multiplier. Without trying to
interpret it too directly, observe that it evidently carries units: its
dimension is the inverse of length, so $\lambda$ is measured in
$\si{m^{-1}}$.

\end{frame}

\begin{frame}\frametitle{Clarification from class}

I got a little confused about the units when discussing $\lambda$. The
coefficients in the LHS of the Lagrangian equations are all in units of
dollars per square meter. Each of $x$, $y$, and $z$ is a length, so the
unit of, e.g., $10y + 10z$ is dollars per meter.

On the other hand, the unit of $xz$ is $\si{m^2}$. Hence, $\lambda$ must
carry units of dollars per cubic meter.

Observe that this is the same as we would expect were we to calculate
the derivative of the cost function with respect to volume. This matter
will be explored in a future module.

\end{frame}

\begin{frame}\frametitle{One good technique}

Solving nonlinear equations is difficult and there are no general
techniques. One technique particular to this situation is to solve each
of the Lagrange equations for $\lambda$. We obtain

\[ \lambda = \frac{10y + 10z}{yz} = \frac{10x + 15z}{xz} = \frac{10x + 15y}{xy} \]

or equivalently

\[ \frac{10}{z} + \frac{10}{y} = \frac{10}{z} + \frac{15}{x} = \frac{10}{y} + \frac{15}{x}. \]

Thus: $y = z$ and $y = \frac{2}{3}x$. Because $xyz = 12$, this gives
$x = 3, y = z = 2$.

\end{frame}

\begin{frame}\frametitle{Work together}

\begin{enumerate}
\item
  Optimize the function $f(x,y) = xy$ on the ellipse $x^2 + 4y^2 = 16$.
  (The level curves of $xy$ are hyperbolas centered at $(0,0)$.)
\pause
\item Solution: The maximum value is $4$, attained at $(2 \sqrt 2, \sqrt{2})$
and $(-2 \sqrt 2, -\sqrt{2})$. The minimum value is $-4$, attained at
$(-2 \sqrt 2, \sqrt{2})$ and $(2 \sqrt 2, -\sqrt{2})$.
\item
  Show that among all rectangles with perimeter $P$, the one of maximal
  area is a square.
\pause
\item Hint: Represent a rectangle by its dimensions $x$ and $y$. Then
$2x + 2y - P = 0$ is the constraint equation.
\end{enumerate}

\end{frame}

\end{document}
