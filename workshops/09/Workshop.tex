\documentclass[12pt]{exam}
\usepackage{cofi}
\usepackage{fourier}
\frenchspacing
\usepackage{parskip}
\usepackage{rosoff-vector-macros}
\firstpageheader{}{}{}
\runningheader{\textbf{Fall 2014}}
 {}
 {\textbf{Math 275}}
 %{\emph{Page \thepage~of \numpages}}
\runningheadrule

\pagestyle{head}
\theoremstyle{definition}
\newtheorem{example}{Example}
\begin{document}

\noindent
\textbf{{\large Math 275 \hfill Workshop 09: Integration over planar regions}}
% \hfill Name: \underline{\hspace{0.5in}Answers\hspace{2in}}

\noindent
October 28, 2013 

\noindent
Due: October 30, 2013 \hfill Name: \underline{\hspace{3in}} 

\noindent

\section{Workshop 09: Integration over planar regions}

You saw in \href{../../decks/13/Deck.pdf}{the presentation} how to
decompose double integrals over a triangle into iterated integrals. The
same technique works for regions whose boundaries have nice algebraic
expressions. We will only concern ourselves with regions whose
boundaries are made up of line segments, arcs of circles, and pieces of
the graphs of functions. Our regions will also all be ``connected''.
This means, roughly speaking, that they are all one piece.

\subsection{More triangles}

The simplest regions meeting our descriptions other than rectangles are
triangles: specifically triangles with one side parallel to an axis.
Give your answers by writing an integral of an anonymous function
$f(x,y)$---that is, fill in the limits and the differentials, but don't
pick a function to integrate.

\begin{questions}

\question For each triangle, find the limits of an iterated integral in each of the possible orders of integration. Draw \emph{big, beautiful pictures}.

\begin{parts}

    \part The triangle with vertices $(-1,2)$, $(2, 2)$, and $(-1, -2)$.

    \flexskip{1}

    \part The triangle with vertices $(-1, 2)$, $(-1, -1)$, and $(1,1)$.

    \flexskip{1}

    \newpage

    \part The triangle with vertices $(-1, -1)$, $(3, -1)$, and $(0,2)$. 

    \flexskip{1}

\end{parts}

\subsection{Regions with curved boundaries}

\question Let $R$ be the region whose boundary is the lines $y = 0$ and $x = 2$ and the parabola $x = y^2$. Find limits for iterated integrals over $R$ in both orders.

\flexskip{2}

\question Let $U$ and $L$ be the parts of the unit disk that fall in the upper and lower half-planes, respectively. Find limits for iterated integrals over $U$ and $L$, in both orders.

\flexskip{1}

\newpage

\question Let $\mathbf{D}$ be the unit disk. Find limits for iterated integrals over $\mathbf{D}$ in both orders. (Use your answers to the previous part.)

\flexskip{1}

\subsection{When order matters}

\question Let $C$ be the circular cap consisting of those points of $\mathbf{D}$ whose $x$-coordinates are at least $-1/2$. Draw $C$, and try to find limits for an iterated integral over it in one or the other order. Does it make a difference which order you choose?

\flexskip{2}

\newpage

\question Let $S$ be the parabolic sector whose boundary is the parabola $y = x^2$ and the line $y = x + 2$. 

\begin{parts}

    \part Find limits for an iterated integral over $S$ in the order $dy \; dx$.

    \flexskip{1}

    \part Find limits for an iterated integral over $S$ in the order $dx \; dy$.

    \flexskip{1}

\end{parts}

\end{questions} 

\end{document}