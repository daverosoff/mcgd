\documentclass[12pt,addpoints]{exam}
\usepackage{cofi}
\usepackage{graphicx}
\DeclareGraphicsExtensions{.jpg, .png}
\usepackage{fourier}
\frenchspacing
\usepackage{parskip}
\firstpageheader{}{}{}
\runningheader{\textbf{Fall 2014}}
 {}
 {\textbf{Math 275}}
 %{\emph{Page \thepage~of \numpages}}
\runningheadrule
\extrawidth{1.0in}
\extraheadheight[-0.5in]{0in}
\extrafootheight{-0.5in}
\pagestyle{head}
% \NewDocumentCommand\N{}{\mathbf{N}}
% \NewDocumentCommand\R{}{\mathbf{R}}
% \NewDocumentCommand\Z{}{\mathbf{Z}}
% \NewDocumentCommand\Q{}{\mathbf{Q}}
% \NewDocumentCommand\flexskip{m}{\vspace*{\stretch{#1}}}

\begin{document}
\noindent
\textbf{{\large Math 275 \hfill Quiz 01}} 
% \hfill Name: \underline{\hspace{0.5in}Answers\hspace{2in}}

\noindent
September 10, 2014; 10 minutes \hfill Name: \underline{\hspace{3in}} \hspace{1em}
Score: \underline{\hspace{0.4in}}/\numpoints

\noindent

\noindent
This quiz is \emph{open-note}, but no books or calculators. On this
quiz, you do not need to justify your answers in any way.

\pointsdroppedatright \pointpoints{pt}{pts}
\pointformat{\hspace{0cm}{\small\fbox{\underline{\hspace{0.8cm}}/\thepoints}}}

\begin{questions} 

\question You are given the following points: $A = (9, -18, -16)$, $B = (18, 0, 9)$, $C = (-6, 19, 12)$.

\begin{parts}
    
    \part[2] Which point is closest to the $(x,y)$-plane? What is the distance from the $(x,y)$-plane to this point?

    \flexskip{1} \droppoints

    \part[2] Which point is closest to the origin? What is the distance from the origin to this point?

    \flexskip{1} \droppoints
\end{parts}


% \newpage

\question Consider the sphere $(x + 2)^2 + (y - 4)^2 + (z - 1)^2 = 16$. For each of the following sets (\ref{p1})--(\ref{p6}), say whether the intersection of the set with the sphere is zero points, one point, two points, a line, or a circle. Remember that mathematical spheres are hollow.

\begin{parts}
    \part[2] \label{p1} The $(y,z)$-plane.

    \flexskip{1} \droppoints

    \part[2] \label{p4} The $x$-axis.

    \flexskip{1} \droppoints

    \part[2] \label{p6} The $z$-axis.

    \flexskip{1} \droppoints

\end{parts}

\end{questions} 

\end{document}