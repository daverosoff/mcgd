\documentclass[11pt,ignorenonframetext,aspectratio=169]{beamer}
\usetheme{Szeged}
\usecolortheme{crane}
\usefonttheme{structurebold}
\usefonttheme{professionalfonts}
\usepackage{amssymb,amsmath}
\usepackage{ifxetex,ifluatex}
\usepackage{fixltx2e} % provides \textsubscript
\ifxetex
  \usepackage{fontspec,xltxtra,xunicode}
  \defaultfontfeatures{Mapping=tex-text,Scale=MatchLowercase}
\else
  \ifluatex
    \usepackage{fontspec}
    \defaultfontfeatures{Mapping=tex-text,Scale=MatchLowercase}
  \else
    \usepackage[utf8]{inputenc}
  \fi
\fi
\usepackage{listings}

% Comment these out if you don't want a slide with just the
% part/section/subsection/subsubsection title:
% \AtBeginPart{
%   \let\insertpartnumber\relax
%   \let\partname\relax
%   \frame{\partpage}
% }
% \AtBeginSection{
%   \let\insertsectionnumber\relax
%   \let\sectionname\relax
%   \frame{\sectionpage}
% }
% \AtBeginSubsection{
%   \let\insertsubsectionnumber\relax
%   \let\subsectionname\relax
%   \frame{\subsectionpage}
% }

\setlength{\parindent}{0pt}
\setlength{\parskip}{6pt plus 2pt minus 1pt}
\setlength{\emergencystretch}{3em}  % prevent overfull lines
\setcounter{secnumdepth}{0}
%%% begin dwr insert
\usepackage{patchcmd}
\usepackage{tabulary}   % Support longer table cells
\usepackage{booktabs}   % Support better tables
\usepackage[sort&compress]{natbib}

\usepackage{framed}     % Allow background color for images
\definecolor{shadecolor}{named}{white}

\usepackage{paralist}
\usepackage{xparse}

\usepackage{subfigure}
%%% end dwr insert
\title{Vectors as displacements; a slicing problem}
\author{Math 275 Multivariable Calculus}
\date{September 10, 2014}

\usepackage{siunitx}

\begin{document}
\frame{\titlepage}

\section{Displacements in the plane}

\begin{frame}\frametitle{What is ``change in position''?}

\begin{itemize}[<+->]
\item
  If I move from $(2,3)$ to $(-3, 1)$, what is the most natural way to
  express this change?
\item
  In everyday English, we separate it into two changes: a change in the
  east--west direction, and a change in the north--south direction.
\end{itemize}

\end{frame}

\begin{frame}\frametitle{Two changes in one}

``Five blocks south and two blocks west'' is a pretty natural way to
express the move from $(2,3)$ to $(-3, 1)$.

\begin{itemize}

\item
  Notice: it's not a sensible \emph{address} in any city or town. Why
  not?
\item
  It is a \emph{displacement}, not a \emph{location}.
\item
  These are different notions!
\end{itemize}

\end{frame}

\begin{frame}\frametitle{Universality of displacement}

\begin{itemize}

\item
  If the town is laid out on a square grid (as opposed to rectangles,
  some other kind of parallelograms, or worse) \ldots
\item
  \ldots~walking from $(2,3)$ to $(-3, 1)$ feels the same as walking
  from $(2013,2013)$ to $(2008,2010)$.
\end{itemize}

We call this displacement the \emph{vector} $\langle -5, -2 \rangle$.
Observe the following na\"ive ``equations'':

\begin{equation*}
(2,3) + \langle -5, -2 \rangle = (-3, 1)
\end{equation*}
\begin{equation*}
(2013,2013) + \langle -5, -2 \rangle = (2008,2010)
\end{equation*}

\end{frame}

\section{Slicing spheres}

\begin{frame}

\begin{itemize}[<+->]
\item You should now be pretty convinced that the intersection of a sphere and a
plane is a circle (provided the intersection contains more than one point). It
certainly seems \emph{plausible} enough. 

\item But is it completely, unambiguously
obvious that it's impossible to get an elliptical cross-section?

\item How would we \emph{prove} it?
\end{itemize}

\end{frame}
\begin{frame}

\frametitle{Brute force}

\begin{itemize}[<+->]
\item Solve with coordinates? 

\item A completely coordinatized approach to this problem involves choosing a
sphere with arbitrary center and radius and an arbitrary plane. 

\item One then solves
a system of two simultaneous nonlinear equation in eight variables:
\begin{align*}
    (x - a)^2 + (y - b)^2 + (z - c)^2 &= r^2 \\
    Ax + By + Cz &= D 
\end{align*}
\end{itemize}

\end{frame}

\begin{frame}

\begin{itemize}[<+->]
\item This approach is easy to set up, if you are familiar with the process.

\item Working out the details is horrendous and not very illuminating. 

\item Instead, we'll unleash the kung fu of symmetry to deal with this complicated geometric situation.

\end{itemize}
\end{frame}

\section{Symmetry and WLOG}

\begin{frame}
\frametitle{Symmetry arguments}
Symmetry argument: a ``special'' case that isn't really all that special. 

For us, the special case will be:

\begin{quote}
The sphere is the unit sphere and the plane is of the form $z = A$.
\end{quote}

\end{frame}

\begin{frame}
\frametitle{Close to generic}
\begin{itemize}[<+->]
\item It seems very special.

\item After all, there are many, many spheres other than the unit sphere, and many
planes that are not horizontal.

\item But think of it like a videographer: the right angle and zoom turns any
situation into this special one.
\end{itemize}

\end{frame}

\begin{frame}
\frametitle{WLOG}
That's what we mean by symmetry, or the mathematical abbreviation WLOG.
This stands for \textbf{W}ithout \textbf{L}oss \textbf{O}f
\textbf{G}enerality. The phrase signifies the idea encapsulated above:
that what appears to be a special case is in fact sufficiently general.

\end{frame}
\end{document}
