\documentclass[]{exam}
\usepackage{rosoff}
\usepackage{graphicx}
\DeclareGraphicsExtensions{.jpg,.png}
\usepackage{fourier, MnSymbol}
\usepackage{enumerate, paralist}
\frenchspacing
\usepackage{hyperref}
\firstpageheader{}{}{}
\runningheader{\textbf{Fall 2014}}
 {\textbf{Math 275}}
 {\emph{Page \thepage~of \numpages}}
\runningheadrule
%\setlength{\parskip}{1ex}
%\setlength{\parindent}{0pt}
\usepackage{parskip}
\pagestyle{head}
\usepackage{rosoff-vector-macros}
\usepackage{rosoff}
\begin{document}

\printanswers
%\addpoints

\noindent
\textbf{{\large Math 275 \\ Workshop 09: Integration over planar regions}}

\noindent
October 28, 2013  %\hfill Name: \underline{\hspace{3in}}
\hfill Name: \underline{\hspace{0.5in}Answers\hspace{2in}}

\section{Workshop 09: Integration over planar regions}

\% Rewrite all of this so that the integrals have integrand 1, not an
anonymous function $f(x,y)$.

You saw in \href{../../decks/13/Deck.pdf}{the presentation} how to
decompose double integrals over a triangle into iterated integrals. The
same technique works for regions whose boundaries have nice algebraic
expressions. We will only concern ourselves with regions whose
boundaries are made up of line segments, arcs of circles, and pieces of
the graphs of functions. Our regions will also all be ``connected''.
This means, roughly speaking, that they are all one piece.

\subsection{More triangles}

The simplest regions meeting our descriptions other than rectangles are
triangles: specifically triangles with one side parallel to an axis.
Give your answers by writing an integral of an anonymous function
$f(x,y)$---that is, fill in the limits and the differentials, but don't
pick a function to integrate.

\begin{questions}

\question For each triangle, find the limits of an iterated integral in each of the possible orders of integration. Draw \emph{big, beautiful pictures}.

\begin{parts}

\part The triangle with vertices $(-1,2)$, $(2, 2)$, and $(-1, -2)$.
\begin{solution}
    The edges of the triangle are segments lying in the lines $x = -1$, $y = 2$, and (by a quick calculation) $y - 2 = (4/3)(x - 2)$, or $3y - 6 = 4x - 8$, or $4x - 3y = 2$. If we integrate $dy \; dx$, we are using vertical segments. Evidently, these segments range from $x = -1$ to $x = 2$. The segment at $x$ has least height $4x/3 - 2/3$ and greatest height $2$. Thus, our integral would have the form
    \begin{equation*}
        \int_{-1}^2 \int_{4x/3-2/3}^{2} f(x,y) \; dy \; dx.
    \end{equation*}
    On the other hand, if we integrate using horizontal segments, the order of integration is $dx \; dy$. The segments lie between $y = -2$ and $y = 2$, and the segment at height $y$ has left endpoint $-1$ and right endpoint $(2-3y)/4 = 1/2 - 3y/4$. Thus, the integral should be
    \begin{equation*}
        \int_{-2}^2 \int_{-1}^{1/2-3y/4} f(x,y) \; dx \; dy.
    \end{equation*}
\end{solution}

\part The triangle with vertices $(-1, 2)$, $(-1, -1)$, and $(1,1)$.
\begin{solution}
    In this problem, each order of integration presents a different challenge. Let us first identify the bounding curves. These are $x = -1$, $y = x$, and $y - 1 = (-1/2)(x-1)$. This last is equivalent to the more convenient $x + 2y = 3$.

    If we use vertical segments, it is clear from a picture that the segments lie between $x = -1$ and $x = 1$. The bottom endpoint of the segment at $x$ is thus $x$, while he top endpoint is $3/2 - x/2$. Putting this together, we write
    \begin{equation*}
        \int_{-1}^{1} \int_{x}^{3/2-x/2} f(x,y) \; dy \; dx.
    \end{equation*}
    For the other order of integration, we have horizontal segments living between $y = -1$ and $y = 2$. The segment at height $y$ evidently has its left endpoint at $x = -1$, but the right endpoint is troublesome. We will need to use two integrals, because if $y > 1$, the right endpoint lies on the line $x + 2y = 3$, while if $y < 1$, the right endpoint lies on the line $y = x$. We obtain
    \begin{equation*}
        \int_{-1}^1 \int_{-1}^{x} f(x,y) \; dx \; dy + \int_{1}^2 \int_{-1}^{3-2y} f(x,y) \; dx \; dy. 
    \end{equation*}
    It is not unusual for one order of integration to be easier. Here, the shape of the boundary gives us a natural preference for the $dy \; dx$ order, because it allows us to use a single iterated integral.
\end{solution}
\part The triangle with vertices $(-1, -1)$, $(3, -1)$, and $(0,2)$. 
\begin{solution}
    This is similar to the previous problem. The bounding curves are $y = -1$, $3x - y = -2$, and $x+y = 2$. With horizontal segments we obtain
    \begin{equation*}
        \int_{-1}^{2}\int_{y/3-2/3}^{2-y} f(x,y) \; dx \; dy,
    \end{equation*}
    and with vertical ones we get
    \begin{equation*}
        \int_{-1}^{0} \int_{-1}^{3x+2} f(x,y) \; dy \; dx + \int_{0}^{3} \int_{-1}^{2-x} f(x,y) \; dy \; dx.
    \end{equation*}
\end{solution}
\end{parts}

\subsection{Regions with curved boundaries}

\question Let $R$ be the region whose boundary is the lines $y = 0$ and $x = 2$ and the parabola $x = y^2$. Find limits for iterated integrals over $R$ in both orders.
\begin{solution}
    This is not very different from the triangle examples. Using vertical segments, we get
    \begin{equation*}
        \int_{0}^{2} \int_{0}^{\sqrt{x}} f(x,y) \; dy \; dx,
    \end{equation*}
    while horizontal segments yield
    \begin{equation*}
        \int_{0}^{\sqrt{2}} \int_{y^2}^{2} f(x,y) \; dx \; dy.
    \end{equation*}
\end{solution}

\question Let $U$ and $L$ be the parts of the unit disk that fall in the upper and lower half-planes, respectively. Find limits for iterated integrals over $U$ and $L$, in both orders.

\begin{solution}
    In each case the better order is $dy \; dx$. For $U$ we find
    \begin{align*}
        \int_{-1}^{1} \int_{0}^{\sqrt{1-x^2}} f(x,y) \; dy &\; dx, \quad \text{or equivalently} \\
        \int_{0}^{1} \int_{-\sqrt{1-y^2}}^{\sqrt{1-y^2}} f(x,y) \; dx &\; dy.
    \end{align*}
    Similarly, for $L$ we find 
    \begin{align*}
        \int_{-1}^{1} \int_{-\sqrt{1-x^2}}^{0} f(x,y) \; dy &\; dx, \quad \text{or equivalently} \\
        \int_{-1}^{0} \int_{-\sqrt{1-y^2}}^{\sqrt{1-y^2}} f(x,y) \; dx &\; dy.
    \end{align*}
\end{solution}

\question Let $\mathbf{D}$ be the unit disk. Find limits for iterated integrals over $\mathbf{D}$ in both orders. (Use your answers to the previous part.)

\begin{solution}
    Here, both orders are equally inconvenient, and not much different. We have
    \begin{align*}
        \int_{-1}^{1} \int_{-\sqrt{1-x^2}}^{\sqrt{1-x^2}} f(x,y) \; dy &\; dx, \\
        \int_{-1}^{1} \int_{-\sqrt{1-y^2}}^{\sqrt{1-y^2}} f(x,y) \; dx &\; dy.
    \end{align*}
\end{solution}

\subsection{When order matters}

\question Let $C$ be the circular cap consisting of those points of $\mathbf{D}$ whose $x$-coordinates are at least $-1/2$. Draw $C$, and try to find limits for an iterated integral over it in one or the other order. Does it make a difference which order you choose?

\begin{solution}
    Here, the $dy \; dx$ order seems easier, because there is only one type of segment. We modify the integral from the previous part:
    \begin{equation*}
        \int_{-1/2}^{1} \int_{-\sqrt{1-x^2}}^{\sqrt{1-x^2}} f(x,y) \; dy \; dx    
    \end{equation*}
    In the $dx \; dy$ order, this is much more difficult, because there are three kinds of segments.
\end{solution}

\question Let $S$ be the parabolic sector whose boundary is the parabola $y = x^2$ and the line $y = x + 2$. 

\begin{parts}

    \part Find limits for an iterated integral over $S$ in the order $dy \; dx$.

    \begin{solution}
        All the segments begin on the parabola and end on the line, so we have
        \begin{equation*}
        \int_{-1}^{2} \int_{x^2}^{x+2} f(x,y) \; dy \; dx.
        \end{equation*}
    \end{solution}

    \part Find limits for an iterated integral over $S$ in the order $dx \; dy$.

    \begin{solution}
        Here, there are two kinds of segments. Segments with $y > 1$ begin on the line and end on the parabola:
        \begin{equation*}
            \int_{1}^{4} \int_{y-2}^{\sqrt{y}} f(x,y) \; dx \; dy
        \end{equation*}
        On the other hand, segments with $y < 1$ both begin and end on the parabola.
        \begin{equation*}
            \int_{0}^{1} \int_{-\sqrt{y}}^{\sqrt{y}} f(x,y) dx \; dy
        \end{equation*}
    \end{solution}
\end{parts}

\end{questions}

\end{document}