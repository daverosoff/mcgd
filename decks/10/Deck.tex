\documentclass[%
  11pt,%
  ignorenonframetext,%
  xcolor={svgnames},%
  aspectratio=169%
]{beamer}
\usetheme{Szeged}
\usecolortheme{crane}
\usefonttheme{structurebold}
\usefonttheme{professionalfonts}
\usepackage{amssymb,amsmath}
\usepackage{ifxetex,ifluatex}
\usepackage{fixltx2e} % provides \textsubscript
\ifxetex
  \usepackage{fontspec,xltxtra,xunicode}
  \defaultfontfeatures{Mapping=tex-text,Scale=MatchLowercase}
\else
  \ifluatex
    \usepackage{fontspec}
    \defaultfontfeatures{Mapping=tex-text,Scale=MatchLowercase}
  \else
    \usepackage[utf8]{inputenc}
  \fi
\fi
\usepackage{listings}

% Comment these out if you don't want a slide with just the
% part/section/subsection/subsubsection title:
% \AtBeginPart{
%   \let\insertpartnumber\relax
%   \let\partname\relax
%   \frame{\partpage}
% }
% \AtBeginSection{
%   \let\insertsectionnumber\relax
%   \let\sectionname\relax
%   \frame{\sectionpage}
% }
% \AtBeginSubsection{
%   \let\insertsubsectionnumber\relax
%   \let\subsectionname\relax
%   \frame{\subsectionpage}
% }

\setlength{\parindent}{0pt}
\setlength{\parskip}{6pt plus 2pt minus 1pt}
\setlength{\emergencystretch}{3em}  % prevent overfull lines
\setcounter{secnumdepth}{0}

%%% begin dwr insert
\usepackage{patchcmd}
\usepackage{tabulary}   % Support longer table cells
\usepackage{booktabs}   % Support better tables
\usepackage[sort&compress]{natbib}

\usepackage{framed}     % Allow background color for images
\definecolor{shadecolor}{named}{white}

%%% end dwr insert
\usepackage{rosoff-vector-macros}
\usepackage[beamer]{cofi}
\title{Optimization of functions with closed and bounded domains}
\author{Math 275 Multivariable Calculus}
\date{October 14, 2013}


\begin{document}
\frame{\titlepage}

\section{Optimization of functions with closed and bounded domains}

\begin{frame}\frametitle{Setup}

An analog of the intermediate value theorem tells us: if $f$ is a
continuous real-valued function whose domain is closed and bounded, then

\begin{itemize}

\item
  $f$ has a global maximum and $f$ has a global minimum.
\end{itemize}

If a global optimum is an \emph{interior point} of the domain, it is
also a local optimum. Hence the techniques of calculus will locate it.

If, on the other hand, it is an \emph{edge point} of the domain, we must
find it by less direct methods.

\end{frame}

\begin{frame}\frametitle{Exercise 35}

Let $f(x,y) = x + y - x^2 - y^2 - xy$, and suppose $f$ to be defined
only on the square $0 \leq x \leq 2$, $0 \leq y \leq 2$. This function
is smooth on its domain, so its critical points are stationary points.

A little computation yields
$\nabla f = \angl{-2x - y + 1, -2y - x + 1}$. The stationary points of
$f$ are thus the solutions (in the domain) of the system

\begin{align*}
    2x + y - 1 &= 0 \\
    x + 2y - 1 &= 0.
\end{align*}

\end{frame}

\begin{frame}\frametitle{Solving for the stationary points}

Multiplying the second equation by $-2$ yields

\begin{align*}
    2x + y - 1 &= 0 \\
    -2x - 4y + 2 &= 0, \\
\end{align*}

and therefore we find (by adding the equations)

\begin{equation*}
    -3y + 1 = 0.
\end{equation*}

\end{frame}

\begin{frame}\frametitle{A unique interior stationary point}

Hence, $y = 1/3$, and it is easy to see that $x = 1/3$ as well. We
compute:

\begin{equation*}
    f(1/3, 1/3) = 1/3.
\end{equation*}

Note: this does \emph{not} tell us whether $(1/3, 1/3)$ is a local
optimum. We are just interested in the value. We'll have finitely many
to compare it to, so we don't bother with the second derivative test.

\end{frame}

\begin{frame}\frametitle{The edge}

We must test the edge separately. In this case, the edge is actually
four edges: the segments that make up the edge of the square.

In your groups, find descriptions of these edges in terms of coordinates.
Most people would use the so-called \emph{set-builder notation} to express
them.

Testing the edges is a routine exercise in slice curves and one-variable
calculus.

\end{frame}

\begin{frame}\frametitle{The left}

On the left, our slice curve is $f(0, y) = y - y^2$. Its domain is the
interval $0 \leq y \leq 2$. Since the slice curve is continuous on this
closed interval, it possesses global optima.

\begin{itemize}
\item
  In your groups, find the global optima of the slice curve. (Remember
  to check the endpoints of the domain!)
  \pause
\item
  $f'(0,y) = 1 - 2y$
  \pause
\item
  Stationary at $y = 1/2$
  \pause
\item
  $f(0, 1/2) = 1/4$; $f(0,0) = 0$; $f(0,2) = -2$
  \pause
\item
  This shows that $f(0,y)$ is maximized at $(0,1/2)$ with value $1/4$
  \pause
\item
  and minimized at $(0,2)$ with value $-2$
\end{itemize}

\end{frame}

\begin{frame}\frametitle{The bottom}

On the bottom, our slice curve is $f(x, 0) = x - x^2$. Its domain is the
interval $0 \leq x \leq 2$. Since the slice curve is continuous on this
closed interval, it possesses global optima.

\begin{itemize}
\item
  In your groups, find the global max and min of this slice curve on the
  domain $0 \leq x \leq 2$.
  \pause
\item
  The function $f$ is symmetric is $x$ and $y$, so this was really easy.
  Usually, there will be a separate check for each curve segment of the
  edge.
\end{itemize}

\end{frame}

\begin{frame}\frametitle{The top}

On the top, our slice curve is $f(x, 2) = -x^2 - x - 2$, with domain
$0 \leq x \leq 2$.

\begin{itemize}
\item
  In your groups, find the global max and min of this slice curve.
  \pause
\item
  We have $f'(x,2) = -2x - 1$, so there is a stationary point at
  $x = -1/2$, not in the domain.
  \pause
\item
  Thus, the optima must occur at the endpoints: $f(0,2) = -2$, while
  $f(2,2) = -8$.
\end{itemize}

\end{frame}

\begin{frame}\frametitle{The right}

A similar symmetry argument disposes of the right edge of the square.
Evidently, the maximum value of $f$ on this edge is $-2$, attained at
$(2,0)$, while the minimum is $-8$, attained at $(2, 2)$.

\end{frame}

\begin{frame}\frametitle{Putting it all together}

Considering all the edge pieces together, the maximum edge value is
$1/4$, attained at $(0, 1/2)$ and $(1/2, 0)$, while the minimum edge
value is $-8$, attained at $(2,2)$.

The only interior critical point was $(1/3, 1/3)$, with value $1/3$.

This shows that the global maximum of $f$ occurs at $(1/3, 1/3)$, while
the global min occurs at $(2,2)$.

\end{frame}

\begin{frame}\frametitle{Work together}

Try problem 36 from 14.7:

Find the maximum of $f(x,y) = y^2 + xy - x^2$ on the square
$0 \leq x \leq 2$, $0 \leq y \leq 2$.

\end{frame}

\end{document}
