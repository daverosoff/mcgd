\documentclass[12pt]{exam}
% \usepackage{pslatex}
\usepackage{graphicx}
\DeclareGraphicsExtensions{.jpg}
\usepackage{amsmath}
\usepackage{amsfonts,eucal}
\usepackage{enumerate}
\firstpageheader{}{}{}
\runningheader{\textbf{Math 251: Fall 2011}}
 {\ifcontinuation{\textbf{Problem continues}}{}}
 {\emph{Page \thepage~of \numpages}}
\runningheadrule
\runningfooter{}{\ifincomplete{continues on next page}{}}{}
\setlength{\parskip}{1.2ex}
\setlength{\parindent}{0pt}
\pagestyle{headandfoot}
\newcommand{\N}{\mathbf{N}}
\newcommand{\Z}{\mathbf{Z}}
\newcommand{\R}{\mathbf{R}}
\DeclareGraphicsExtensions{.pdf,.png,.jpg}
\begin{document}

% \printanswers
\addpoints

\noindent
\textbf{{\large Mathematics 251 \\ Final Exam}}
% \hfill Name: \underline{\hspace{0.5in}Answers\hspace{2in}}

\noindent
December 5, 2011  \hfill Name: \underline{\hspace{3in}}

\noindent
\textbf{Instructions}: This exam is closed book: you may refer to one $8.5 \times 11$ page of handwritten notes, but no electronic aids or
other printed references are permitted. \emph{Justification of all answers is required
for partial credit.} Unless
specifically directed, leave all answers in \textbf{exact form}, e.g.\
$\sqrt{3}$ instead of~1.732 and $\pi/2$ instead of~$1.57$.

Show all pertinent work. \emph{Correct answers without accompanying work will receive little or no credit.} Results from homework or from class can be cited freely. It is in your interest to display your solution in a
clear, readable fashion. Note that standards of justification are not as high as for homework.

\emph{Budget your time wisely.} It is a good idea, whenever possible, to look through the entire exam before beginning work on a particular problem.

If your work continues onto the back of another page, please indicate
this. Check and make sure you have all of the pages in the
exam; there should be \numpages, including this one. If you have a
question, please raise your hand.

Be sure to read all questions carefully and completely.

\vspace*{1.0in}

\begin{center}
\combinedgradetable[h]
\end{center}

\vspace*{0.5in}

\begin{center}
{\Large \emph{Good luck!}}
\end{center}

\newpage

\begin{questions}

\question Consider the helix $\mathcal{C}$ pictured below. It is parametrized by the function $\mathbf{r} \colon \R \to \R^3$, $\mathbf{r} = \langle a \cos t, a \sin t, bt \rangle$.

\begin{parts}

\part Find the unit tangent vector $\mathbf{T}$ to the helix at time $t$.

\vspace{\stretch{1}}

\part Verify that the curvature of the helix is equal to $|a|/(a^2+b^2)$.

\vspace{\stretch{1}}

\part Find a parametrization of the helix using cylindrical coordinates, i.e., find functions $r$, $\theta$, $z \colon \R \to \R$ such that for each $t \in \R$, $\mathbf{s}(t) = \langle r(t), \theta(t), z(t) \rangle$ is the cylindrical expression of the point $\mathbf{r}(t)$. 

\vspace{\stretch{1}}

\part Write down, but \textbf{do not evaluate}, a scalar integral whose value is equal to the value of the line integral 
\[
\int_{\mathcal{C}} \mathbf{F} \cdot d\mathbf{s}, \quad \text{where } \mathbf{F}(t) = \frac{1}{\sqrt{a^2 + b^2}} \langle -a \sin t, a \cos t, b \rangle 
\]

\end{parts}

\newpage

\question Consider a differentiable function $w = f(x,y,z)$ under the cylindrical coordinate transform $\Phi \colon \R^3 \to \R^3$, $\Phi(\langle x, y , z\rangle) = \langle r \cos \theta, r \sin \theta, z \rangle$ .

\begin{parts}

\part[12] Use the chain rule to express the partial derivatives $\partial w / \partial r$ and $\partial w / \partial \theta$ in terms of the partial derivatives $\partial w / \partial x$ and $\partial w / \partial y$. This means your answer should \emph{not} contain any terms like $\partial x / \partial r$ or $\partial y / \partial \theta$.

\vspace{\stretch{2}}

\part[12] Referring to your answer above, show that
\[
|| \nabla \! f ||^2 = \left( \frac{\partial w}{\partial r} \right)^2 + \frac{1}{r^2} \left( \frac{\partial w}{\partial \theta} \right)^2 + \left( \frac{\partial w}{\partial z} \right)^2.
\]
\emph{Hint.} Expand the left-hand side using the definition of the gradient as a vector and properties of dot products.

\vspace{\stretch{3}}

\end{parts}

\newpage

\question[12] Find the critical points of the function $f(x,y) = \sin x \cos y$ in the region $0 \leq x \leq \pi$, $0 \leq y \leq \pi$ and identify the global maximum and minimum of $f$ in the region. Justify your answers for full credit. Elementary properties of the sine and cosine functions may be cited without justification.

\vspace{\stretch{1}}

\question[12] COPY ONE FROM THE BOOK

\vspace{\stretch{2}}

\newpage

\question Mark each item TRUE or FALSE. Recall that unless a statement is \emph{always true} (that is, for all permitted values of its variables), then it is false.

\begin{tabular}{p{5in}cc}
Every smooth curve in $\R^3$ admits a parametrization with speed 1. & TRUE & FALSE \\
 & & \\
If the cost function $f$ and constraint function $g$ are differentiable and $P$ is a local minimum of $f$ along the level set $g = 0$, then the level sets $f= 0 $ and $g = 0$ are tangent at $P$ provided that $\nabla G_P \ne 0$. & TRUE & FALSE \\
 & & \\
Every vector field whose cross-partials are equal is conservative. & TRUE & FALSE \\
 & & \\
There exist a vector field $\mathbf{F}$ defined on the punctured unit disc $D^{\ast} = \{ (x, y) \in \R^2 : 0 < x < 1, 0 < y < 1 \}$ and a closed curve $\mathcal{C}$ contained in $D^{\ast}$ such that $\oint_{\mathcal{C}} \mathbf{F}\cdot d\mathbf{s} \ne 0$. & TRUE & FALSE \\
 & & \\
A function $f$ is differentiable at the point $P$ if all derivatives $D_{\mathbf{u}} f$ exist at $P$. & TRUE & FALSE \\
& & \\
The cross product of two vectors is orthogonal to both vectors. & TRUE & FALSE \\
& & \\
There is exactly one vector in $\R^n$ of length zero. & TRUE & FALSE \\
& & \\

\end{tabular}

\newpage

\question Green's theorem states that if $D$ is a bounded subset of the plane whose boundary is a smooth simple closed curve $\mathcal{C}$ and $\mathbf{F}$ is a smooth vector field on $D$, then
\[
    \oint_{\mathcal{C}} \mathbf{F}\cdot d\mathbf{s}
\]

\end{questions}

\end{document}