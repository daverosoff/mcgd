\documentclass[11pt,ignorenonframetext,aspectratio=169]{beamer}
\usetheme{Szeged}
\usecolortheme{crane}
\usefonttheme{structurebold}
\usefonttheme{professionalfonts}
\usepackage{amssymb,amsmath}
\usepackage{ifxetex,ifluatex}
\usepackage{fixltx2e} % provides \textsubscript
\ifxetex
  \usepackage{fontspec,xltxtra,xunicode}
  \defaultfontfeatures{Mapping=tex-text,Scale=MatchLowercase}
\else
  \ifluatex
    \usepackage{fontspec}
    \defaultfontfeatures{Mapping=tex-text,Scale=MatchLowercase}
  \else
    \usepackage[utf8]{inputenc}
  \fi
\fi
\usepackage{listings}

% Comment these out if you don't want a slide with just the
% part/section/subsection/subsubsection title:
% \AtBeginPart{
%   \let\insertpartnumber\relax
%   \let\partname\relax
%   \frame{\partpage}
% }
% \AtBeginSection{
%   \let\insertsectionnumber\relax
%   \let\sectionname\relax
%   \frame{\sectionpage}
% }
% \AtBeginSubsection{
%   \let\insertsubsectionnumber\relax
%   \let\subsectionname\relax
%   \frame{\subsectionpage}
% }

\setlength{\parindent}{0pt}
\setlength{\parskip}{6pt plus 2pt minus 1pt}
\setlength{\emergencystretch}{3em}  % prevent overfull lines
\setcounter{secnumdepth}{0}
%%% begin dwr insert
\usepackage{patchcmd}
\usepackage{tabulary}   % Support longer table cells
\usepackage{booktabs}   % Support better tables
\usepackage[sort&compress]{natbib}

\usepackage{framed}     % Allow background color for images
\definecolor{shadecolor}{named}{white}

\usepackage{paralist}
\usepackage{xparse}

\usepackage{subfigure}
%%% end dwr insert
\title{Displacement in several directions: vectors}
\author{Math 251 Calculus 3}
\date{September 10, 2013 }

\usepackage{siunitx}

\begin{document}
\frame{\titlepage}

\section{Displacements in the plane}

\begin{frame}\frametitle{What is ``change in position''?}

\begin{itemize}[<+->]
\item
  If I move from $(2,3)$ to $(-3, 1)$, what is the most natural way to
  express this change?
\item
  In everyday English, we separate it into two changes: a change in the
  east--west direction, and a change in the north--south direction.
\end{itemize}

\end{frame}

\begin{frame}\frametitle{Two changes in one}

``Five blocks south and two blocks west'' is a pretty natural way to
express the move from $(2,3)$ to $(-3, 1)$.

\begin{itemize}

\item
  Notice: it's not a sensible \emph{address} in any city or town. Why
  not?
\item
  It is a \emph{displacement}, not a \emph{location}.
\item
  These are different notions!
\end{itemize}

\end{frame}

\begin{frame}\frametitle{Universality of displacement}

\begin{itemize}

\item
  If the town is laid out on a square grid (as opposed to rectangles,
  some other kind of parallelograms, or worse) \ldots
\item
  \ldots~walking from $(2,3)$ to $(-3, 1)$ feels the same as walking
  from $(2013,2013)$ to $(2008,2010)$.
\end{itemize}

We call this displacement the \emph{vector} $\langle -5, -2 \rangle$.
Observe the following na\"ive ``equations'':

\begin{equation*}
(2,3) + \langle -5, -2 \rangle = (-3, 1)
\end{equation*}
\begin{equation*}
(2013,2013) + \langle -5, -2 \rangle = (2008,2010)
\end{equation*}

\end{frame}

\end{document}
