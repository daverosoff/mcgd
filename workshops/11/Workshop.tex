\documentclass[12pt]{exam}
\usepackage{xparse}
\usepackage{graphicx}
\DeclareGraphicsExtensions{.jpg, .png}
\usepackage{fourier}

\usepackage{amsthm}
\usepackage{paralist,enumerate,listings}
\usepackage{siunitx}
\frenchspacing
\usepackage{parskip}
\usepackage{rosoff-vector-macros}
\usepackage{rosoff}
%\usepackage{MnSymbol}
\usepackage{hyperref}
\firstpageheader{}{}{}
\runningheader{\textbf{Fall 2013}}
 {}
 {\textbf{Math 251}}
 %{\emph{Page \thepage~of \numpages}}
\runningheadrule

\pagestyle{head}
\theoremstyle{definition}
\newtheorem{example}{Example}
\begin{document}
\lstset{language=R}
\noindent
\textbf{{\large Math 251 \hfill Workshop 11: integrals in curved coordinates}}
% \hfill Name: \underline{\hspace{0.5in}Answers\hspace{2in}}

\noindent
November 6, 2013 

\noindent
Due: November 11, 2013 \hfill Name: \underline{\hspace{3in}} 

\noindent

\section{Workshop 11: integrals in curved coordinates}

This workshop gives you a chance to practice integrals with curved
coordinate systems.

\subsection{A polar example with an application}

\begin{questions}

\question The region $D$ is that enclosed by the circle of radius $2$ centered at the origin, but outside the circle of radius $1$ centered at $(1,0)$. Hence the area of $D$ is $3\pi$. Derive this result using an integral. (The polar equation of the small circle is $r = 2\cos\theta$.)

\dwrspace{2}

\question The \emph{center of mass} of a lamina is, roughly speaking, the point on which the lamina will balance if placed on a needle. Consider the region $D$ above with uniform density $\rho(x,y) = 1$. Let $(\hat{x}, \hat{y})$ denote its center of mass.

\begin{parts}

\part Argue by symmetry that $\hat{y} = 0$ for the region $D$. Is this true if the density of the lamina is not required to be uniform?

\dwrspace{1}

\part The definition of the center of mass is formulated using integrals.
$$ \hat{x} = \frac{1}{M} \iint_D x\rho(x,y) \; dA, \quad \hat{y} = \frac{1}{M} \iint_D y\rho(x,y) \; dA.$$
Here $M = \iint_D \rho(x,y) \; dA$ is the mass of the solid. Calculate the coordinate $\hat{x}$.
\dwrspace{4}
\end{parts}

\newpage

\question Let $E$ be the region of $\R^3$ above the cone $z = \sqrt{x^2 + y^2}$ and below the unit sphere $x^2 + y^2 + z^2 = 1$. Suppose $\rho(x,y,z) = z$ is the density function. Use a curved coordinate system to compute the center of mass of the resulting solid, using symmetry to simplify the argument if appropriate. Consult tables of integrals as necessary. \emph{Hint.} The angle at the cone point is a right angle.

\dwrspace{1}

\end{questions} 

\end{document}