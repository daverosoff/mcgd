\documentclass[11pt,ignorenonframetext,aspectratio=169]{beamer}
\usetheme{Szeged}
\usecolortheme{crane}
\usefonttheme{structurebold}
\usepackage{amssymb,amsmath}
\usepackage{ifxetex,ifluatex}
\usepackage{fixltx2e} % provides \textsubscript
\ifxetex
  \usepackage{fontspec,xltxtra,xunicode}
  \defaultfontfeatures{Mapping=tex-text,Scale=MatchLowercase}
\else
  \ifluatex
    \usepackage{fontspec}
    \defaultfontfeatures{Mapping=tex-text,Scale=MatchLowercase}
  \else
    \usepackage[utf8]{inputenc}
  \fi
\fi
\usepackage{listings}

% Comment these out if you don't want a slide with just the
% part/section/subsection/subsubsection title:
% \AtBeginPart{
%   \let\insertpartnumber\relax
%   \let\partname\relax
%   \frame{\partpage}
% }
% \AtBeginSection{
%   \let\insertsectionnumber\relax
%   \let\sectionname\relax
%   \frame{\sectionpage}
% }
% \AtBeginSubsection{
%   \let\insertsubsectionnumber\relax
%   \let\subsectionname\relax
%   \frame{\subsectionpage}
% }

\setlength{\parindent}{0pt}
\setlength{\parskip}{6pt plus 2pt minus 1pt}
\setlength{\emergencystretch}{3em}  % prevent overfull lines
\setcounter{secnumdepth}{0}
%%% begin dwr insert
\usepackage{patchcmd}
\usepackage{tabulary}   % Support longer table cells
\usepackage{booktabs}   % Support better tables
\usepackage[sort&compress]{natbib}

\usepackage{framed}     % Allow background color for images
\definecolor{shadecolor}{named}{white}

%\usepackage{paralist}
\usepackage{xparse}

\usepackage{subfigure}
\usepackage{hyperref}
%%% end dwr insert
\title{R Markdown and reproducible research}
\author{Math 251 Calculus 3}
\date{September 16, 2013 }

\usepackage{siunitx}

\begin{document}
\frame{\titlepage}

\section{R Markdown and reproducible research}

\begin{frame}\frametitle{The idea of markup languages}

Word processing software---mostly, Microsoft Word---is so pervasive that
most people fire it up without thinking whenever they need to do a bit
of writing. For the last 20 years or so, WYSIWYG word processors have
dominated when it comes to text documents.

In a word processor, you style individual text elements directly: bold,
indented, red, and so on.

Markup languages are different. The document author ``marks up'' the
document with logical structure rather than specific display
instructions.

\end{frame}

\begin{frame}[fragile]\frametitle{An example of logical markup}

\begin{verbatim}
<title>Example document</title>
<body>
    <h1>Excellent example</h1>
    This document is a toy example. Its markup format
    is <em>modified</em> HTML.
    <ul>
        <li>These</li>
        <li>Are</li>
        <li>List</li>
        <li>Items.</li>
    </ul>
</body>
\end{verbatim}

\end{frame}

\begin{frame}\frametitle{HTML}

Most people's direct experience of markup is limited, instead being
mediated by web browsers. HTML is the language of the Web, and it stands
for \textbf{H}yper\textbf{t}ext \textbf{M}arkup \textbf{L}anguage.

HTML consists of text and ``HTML elements'', which instruct the browser
on the various structural elements of the document: title and author,
headings at various levels, paragraphs, block quotes, ordered or
unordered lists, emphasized text, links, images, and so on.

HTML does \emph{not} tell the browser how to render the document to the
screen. The browser makes these decisions autonomously, often with help
from an external style sheet.

\end{frame}

\begin{frame}\frametitle{Plain text: the way of the Jedi}

\begin{itemize}[<+->]

\item
  HTML documents are stored as \emph{plain text}

  \begin{itemize}[<+->]
  
  \item
    Editors, word processors, and file formats come and go
  \item
    as anyone who has dealt with Word's ``compatibility issues'' knows
  \item
    If you want your documents to be readable in 20 years, or even 5
    years, plain text is your best bet
  \end{itemize}
\item
  Plain text is also fun to use

  \begin{itemize}[<+->]
  
  \item
    The best programs for editing plain text are obscure and powerful,
    so you will feel like a hacker kung fu wizard
  \item
    Plain text will make you seem mysterious and attractive at parties
  \end{itemize}
\end{itemize}

\begin{quote}
The browsers that render HTML may pass on into oblivion, but the files
will remain human-readable.
\end{quote}

\end{frame}

\begin{frame}\frametitle{Markdown: a plain-text road to HTML}

Markdown is another plain text format, designed to be easy to write. It
was invented for the blogging community, but has found wide use in
academia and scientific circles as well.

The idea is that the simpler HTML elements are easily generated by plain
text shorthand. For example, headings are set off by one or more \#
characters (up to six, since HTML provides six levels of subheading).
Paragraphs are separated---naturally---by blank lines. Text is
\emph{emphasized} by surrounding it with * asterisks.

\begin{enumerate}[<+->]
\def\labelenumi{\arabic{enumi}.}

\item
  Ordered lists are simple to generate.
\item
  So are unordered lists; they can be

  \begin{itemize}[<+->]
  
  \item
    Nested
  \item
    or even mixed, using indentation to indicate nesting.
  \end{itemize}
\end{enumerate}

\end{frame}

\begin{frame}\frametitle{Markdown is usually parsed into HTML}

There are many utilities that can read Markdown source text and generate
a fully-fledged web page. All the modules for this course are written in
Markdown, and then the utility \texttt{pandoc} is used, together with a
basic template and some CSS to weave them into HTML files.

In fact, the workshops and slide decks are also written in Markdown.
Using different templates and \texttt{pandoc} commands, they are passed
to the sophisticated text rendering engine LaTeX to produce the handouts
and slide shows.

\begin{quote}
You don't need any of this fancy stuff to make good use of Markdown.
There is extensive R support for Markdown built into R Studio.
\end{quote}

\end{frame}

\begin{frame}\frametitle{R Markdown and scientific computing with
reporting}

The literate programming tradition has necessitated the development of
tools to combine visually pleasing text with source code. In the R
world, the tool of choice for many years was Sweave (S + ``weave'', for
weaving the code and text together).

Sweave allowed computations to be performed \emph{in the actual
reporting document} rather than in a separate environment. However, to
use Sweave, you must know how to write documents in LaTeX, which is
beyond the scope of this course.

A modern, lightweight alternative to Sweave is \texttt{knitr}, a package
available from within R Studio.

\end{frame}

\begin{frame}\frametitle{What is knitr?}

The \texttt{knitr} package allows you to use Markdown and R code in the
same document, which is then easily (seriously: one click) rendered into
HTML. Any R computations or plots you generate are included into the
final HTML document.

By default, the output is like a standard web page, but with the help of
templates, CSS style information, and powerful text processing engines
such as \texttt{pandoc}, any kind of HTML document can be generated. In
particular, it is possible to create impressive HTML5 slideshows
containing professionally typeset mathematics and R-generated images
without surprisingly little effort.

\end{frame}

\end{document}
