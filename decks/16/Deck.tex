\documentclass[11pt,ignorenonframetext,aspectratio=169,xcolor={svgnames}]{beamer}
\usetheme{Szeged}
\usecolortheme{crane}
\usefonttheme{structurebold}
\usefonttheme{professionalfonts}
\usepackage{amssymb,amsmath}
\usepackage{ifxetex,ifluatex}
\usepackage{fixltx2e} % provides \textsubscript
\ifxetex
  \usepackage{fontspec,xltxtra,xunicode}
  \defaultfontfeatures{Mapping=tex-text,Scale=MatchLowercase}
\else
  \ifluatex
    \usepackage{fontspec}
    \defaultfontfeatures{Mapping=tex-text,Scale=MatchLowercase}
  \else
    \usepackage[utf8]{inputenc}
  \fi
\fi
\usepackage{listings}

% Comment these out if you don't want a slide with just the
% part/section/subsection/subsubsection title:
% \AtBeginPart{
%   \let\insertpartnumber\relax
%   \let\partname\relax
%   \frame{\partpage}
% }
% \AtBeginSection{
%   \let\insertsectionnumber\relax
%   \let\sectionname\relax
%   \frame{\sectionpage}
% }
% \AtBeginSubsection{
%   \let\insertsubsectionnumber\relax
%   \let\subsectionname\relax
%   \frame{\subsectionpage}
% }

\setlength{\parindent}{0pt}
\setlength{\parskip}{6pt plus 2pt minus 1pt}
\setlength{\emergencystretch}{3em}  % prevent overfull lines
\setcounter{secnumdepth}{0}

%%% begin dwr insert
\usepackage{patchcmd}
\usepackage{tabulary}   % Support longer table cells
\usepackage{booktabs}   % Support better tables
\usepackage[sort&compress]{natbib}

\usepackage{framed}     % Allow background color for images
\definecolor{shadecolor}{named}{white}

%\usepackage{paralist}
\usepackage{xparse}
\usepackage{subfigure}
\usepackage{hyperref}
%%% end dwr insert
\usepackage{rosoff-vector-macros}
\usepackage[beamer]{cofi}
\title{Gluing parametrizations}
\author{Math 251 Calculus 3}
\date{November 15, 2013}


\begin{document}
\frame{\titlepage}

\section{Gluing parametrizations}

\begin{frame}\frametitle{Warm-up}

\begin{itemize}[<+->]
\itemsep1pt\parskip0pt\parsep0pt
\item
  Parametrize the arc of the circle of radius $3$ centered at $(0,0)$ in
  the left half of the plane, in the positive direction. (Feel free to
  use multiples of $\tau = 2\pi$ in your parametrization. It may be
  easier to think about this way.)
\end{itemize}

\end{frame}

\begin{frame}\frametitle{Piecewise paths}

We are very often interested in parametrizations of curves with corners,
such as squares and rectangles. The easiest way to parametrize such a
curve is to paste together smaller parametrizations of its pieces. To do
this, we need to be able to shift parametrizations in time.

Let $\vec{r}(t)$ be a parametrized curve, say for $a \leq t \leq b$. We
can \emph{shift} the ``time'' domain of the parametrization so that $t$
runs over whatever interval we choose. For example, define
$\vec{s}(t) = \vec{r}(t - t_0)$, for some positive real number $t_0$.
(The transformation $t \mapsto t - t_0$ is called a \emph{translation}.)

This shifts the time domain of $\vec{r}$ into the future, since
$a \leq t - t_0 \leq b$ if and only if $a + t_0 \leq t \leq b + t_0$.
Thus the domain of $\vec{s}$ is $[a + t_0, b+t_0]$.

\end{frame}

\begin{frame}\frametitle{Reparametrize}

Can you adjust your parametrization of the circular arc to begin at
$t = 0$?

\end{frame}

\begin{frame}\frametitle{Adjusting the speed}

We can also speed up or slow down our parametrizations, by scaling in
the time domain. Suppose $\vec{r}(t)$ is a parametrization over the
interval $[0, k]$. Observe that

\begin{equation*}
    0 \leq t \leq k \quad \text{if and only if} \quad 0 \leq 2t \leq 2k.
\end{equation*}

The transformation $t \mapsto 2t$ is called a \emph{scaling} or
\emph{dilation}. Evidently, it speeds up the motion of our moving
particle, halving the time it takes to traverse its path. It's
counterintuitive, but think of it this way: however long it takes $t$ to
move from $a$ to $b$, it takes $2t$ half as long! Because

\begin{equation*}
    \frac{b}{2} - \frac{a}{2} = \frac{b-a}{2} = \frac{1}{2}(b-a).
\end{equation*}

\end{frame}

\begin{frame}\frametitle{Standardizing the time domain}

\begin{itemize}[<+->]
\itemsep1pt\parskip0pt\parsep0pt
\item
  Can you find a parametrization of the circular arc that begins at
  $t = 0$ and ends at $t = 1$? Hint: it's always easier to scale first
  and then translate.
\item
  So first get the time domain to be the appropriate length (1), and
  then translate so that the starting and ending times are right.
\item
  Answer:
  $\vec{r}(t) = \angl{3 \cos{(\pi t - \frac{1}{2})}, 3 \sin{(\pi t - \frac{1}{2})}}$
\end{itemize}

\end{frame}

\begin{frame}\frametitle{Gluing parametrizations}

Let $C$ be the boundary of the unit square: so $C$ consists of four
segments connecting the four points $(0,0)$, $(0,1)$, $(1,0)$, and
$(1,1)$. Let the four segments beginning with the bottom and proceeding
counterclockwise be called $C_1$ up through $C_4$.

\begin{itemize}[<+->]
\itemsep1pt\parskip0pt\parsep0pt
\item
  Parametrize each of the four segments. Don't worry for now about the
  time intervals; we can always scale and translate later. Just do it in
  the way that seems easiest to write down. Often this means starting at
  $t = 0$. You should have four different vector-valued functions
  $\vec{r}_i(t)$, one for each $C_i$.
\item
  Now \emph{glue} your parametrizations for $C_1$ and $C_2$. This means
  to adjust $\vec{r}_2(t)$ so that it begins at the same $t$-value at
  which $\vec{r}_1(t)$ ends.
\item
  Glue the rest of your parametrizations in the same way to parametrize
  the whole square.
\end{itemize}

\end{frame}

\begin{frame}\frametitle{Smoothness of parametrizations}

We won't talk much about limits or continuity for parametrized curves,
but it seems like the square is not very smooth. However, your glued
parametrization is continuous, and while it is not differentiable, it is
\emph{piecewise} differentiable, which in practice is good enough.

\end{frame}

\begin{frame}\frametitle{Parametrizing graphs}

If a curve comes to you as the graph of some function $y = f(x)$, it is
easy to come up with a parametrization.

\begin{itemize}[<+->]
\itemsep1pt\parskip0pt\parsep0pt
\item
  Let $t = x$.
\item
  Then $\vec{r}(t) = \angl{t, f(t)}$.
\item
  Boom.
\end{itemize}

\end{frame}

\begin{frame}\frametitle{Some parametrized curves of more exotic type}

The MAA (Mathematical Association of America) has provided really
excellent applets for playing with parametrically described families of
curves.

\href{http://www.maa.org/publications/periodicals/loci/resources/the-beauty-of-parametric-curves-the-applets}{This
text, while not underlined, is a clickable link to the applets.}

\href{http://www.desmos.com}{This clickable sentence links to excellent
online graphing calculator Desmos.} Desmos supports parametrized curves.
Enter them in vector form, but use round brackets $()$ instead of angle
brackets $\angl{}$.

\end{frame}

\begin{frame}\frametitle{Tangent vectors, velocity, speed, and
arclength}

If $\vec{r}(t)$ is a vector function, we say it is continuous if all its
entries are continuous functions. Same thing for differentiable. We
think of $\vec{r}(t)$ as a position function, because at each time $t$
it shows how to get to our particle from the origin. Then its derivative
$\vec{r'}(t)$ turns out to be the instantaneous velocity vector of the
moving particle.

\begin{itemize}[<+->]
\itemsep1pt\parskip0pt\parsep0pt
\item
  If exactly at time $t$ you detach the moving particle from the
  function $\vec{r}$ and let it move in the exact direction and speed it
  had at that moment, then at time $t+1$ it has moved by exactly
  $\vec{r'}(t)$.
\item
  How do we differentiate? One entry at a time.
  $\vec{r'}(t) = \angl{x'(t), y'(t)}$ and so on.
\end{itemize}

\end{frame}

\begin{frame}\frametitle{Velocity as a vector}

Velocity has always been a vector. You know that speed is nonnegative,
while in one dimension velocity carries a sign. This is information
about the direction of travel, just as meaningful as the speed itself.
Since motion in a plane isn't confined to a backward and a forward
direction, we need the complete generality of vectors in the plane to
describe the possible velocities a moving particle might have.

Usually, when we draw parametrized curves, we don't actually draw the
vectors $\vec{r}(t)$. Instead we draw the trajectory: the collection of
endpoints of $\vec{r}(t)$. But we frequently will draw the vector
$\vv(t) = \vec{r'}(t)$ attached to the curve, with its tail at the
endpoint of $\vec{r}(t)$. It's a moving tangent vector!

\end{frame}

\begin{frame}\frametitle{Speed as the length of velocity}

The entries of the velocity vector (tangent vector---same thing) tell
you the velocities in each of the coordinate directions. But the speed
is something else. It is the magnitude of the velocity vector. In two
dimensions,

\begin{equation*}
    \frac{ds}{dt} = \norm{\angl{x'(t), y'(t)}} = \sqrt{x'(t)^2 + y'(t)^2}
\end{equation*}

\end{frame}

\begin{frame}\frametitle{Arc length}

We can obtain the length of a parametrized curve (not the same thing as
the length of its time domain or parameter interval) by integrating
speed over the parameter interval $[a,b]$:

\begin{equation*}
    s = \int_a^b \frac{ds}{dt} \; dt = \int_a^b \sqrt{\left( \frac{dx}{dt} \right)^2 + \left( \frac{dy}{dt} \right)^2} \; dt
\end{equation*}

\end{frame}

\begin{frame}\frametitle{Practice translating, scaling, gluing}

\begin{itemize}
\itemsep1pt\parskip0pt\parsep0pt
\item
  Circle, center at origin, traced once counterclockwise, starting at
  $(0,3)$ with $t \in [0,1]$.
\item
  Circle, center at $(a,b)$, traced once counterclockwise, starting at
  $(a+r, b)$ with $t \in [0, 2\pi] = [0, \tau]$.
\item
  Circle, center at origin, traced once \emph{clockwise}, starting at
  $(1,0)$.
\item
  Pseudotriangle with boundary the segments from $(0,1)$ to $(0,0)$,
  from $(0,0)$ to $(1,0)$, and the arc of the unit circle connecting
  $(1,0)$ to $(0,1)$. Traverse counterclockwise with time interval
  $[0,1]$.
\end{itemize}

\end{frame}

\end{document}
