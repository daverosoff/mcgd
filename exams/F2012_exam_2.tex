\documentclass[11pt]{exam}
% \usepackage{pslatex}
\usepackage{graphicx}
\DeclareGraphicsExtensions{.jpg,.png}
\usepackage{amsmath}
\usepackage{amsfonts}
\usepackage{enumerate}
\firstpageheader{}{}{}
\runningheader{\textbf{Fall 2012}}
 {\textbf{Math 251}}
 {\emph{Page \thepage~of \numpages}}
\runningheadrule
\setlength{\parskip}{1ex}
\setlength{\parindent}{0pt}
\pagestyle{head}
\newcommand{\N}{\mathbb{N}}
\newcommand{\Z}{\mathbb{Z}}
\newcommand{\R}{\mathbb{R}}
\newcommand{\dwrspace}[1]{\vspace*{\stretch{#1}}}
\renewcommand{\vec}[1]{\mathbf{#1}}
\begin{document}

% \printanswers
\addpoints

\noindent
\textbf{{\large Mathematics 251 \\ Exam 2}}
% \hfill Name: \underline{\hspace{0.5in}Answers\hspace{2in}}

\noindent
October 30, 2012  \hfill Name: \underline{\hspace{3in}}


\noindent
\textbf{Instructions}: This exam is closed book. You may refer to one double-sided page of handwritten notes, but no electronic aids or other
printed references are permitted. Justification of all answers is required
for partial credit; please \fbox{\textbf{box}} your final answers. Unless
specifically directed, leave all answers in \textbf{exact form}, e.g.\
$\sqrt{3}$ instead of~$1.732$ and~$\pi/2$ instead of~$1.57$.

Show all pertinent work. \emph{Correct answers without accompanying work will receive little or no credit.} Results from class or from homework or from class can be cited freely. It is in your interest to display your solution in a clear, readable fashion.

If you need to include more pages, staple them to the back of your exam and make sure that they are clearly labeled by problem number. Indicate in the main body of the exam that your work continues on another page.

Check and make sure you have all of the pages in the
exam; there should be \numpages, including this one. If you have a
question, please raise your hand.

Be sure to read all questions carefully and completely.

\vspace*{2in}

\begin{center}
\combinedgradetable[h]
\end{center}

\vspace*{0.5in}

\begin{center}
{\Large \emph{Good luck!}}
\end{center}

\newpage

\begin{questions}

\question[20] Find the velocity, acceleration, and speed of a particle with the given position function~$\vec{r}(t)$ on the specified domain. Simplify your answers for full credit.

\begin{parts}
    
    \part $\vec{r}(t) = \langle t^2, 2t, \ln t \rangle, \quad t > 0$

    \dwrspace{1}

    \part $\vec{r}(t) = \langle t^2, \sin t - t \cos t, 
    \cos t + t \sin t \rangle, \quad t \geq 0$

    \dwrspace{1}

\end{parts}

\question[10] Consider a particle moving in~$\R^3$ in such a way that for all times~$t$, its acceleration is given by~$\vec{a}(t) = \langle 2t, \sin t, \cos 2t \rangle$. Suppose in addition that its velocity and position when~$t = 0$ are the vectors
\[
    \vec{r}_0 = \vec{r}(0) = \langle 0, 1, 0 \rangle, \quad 
    \vec{v}_0 = \vec{v}(0) = \langle 1, 0, 0 \rangle.
\]
Find a formula for the particle's position at time~$t$; in other words, give the position function~$\vec{r}(t)$.

\dwrspace{2}

\newpage

\question[8] Consider the function~$f \colon \R \to \R$ given by~$f(t) = e^{t^2}$. What is the curvature of its graph at the point~$(0,1)$? You need not simplify your answer.

\dwrspace{3}

\question

    \begin{parts}

    \part[8] Find the tangential component $a_{\vec{T}}$ of acceleration for the parametrized curve $\vec{r}(t) = \langle t^2 , t^3 \rangle$ when $t = 1$. 

    \dwrspace{3}

    \bonuspart[4] Find the normal component $a_{\vec{N}}$ of acceleration for the parametrized curve $\vec{r}(t) = \langle t^2 , t^3 \rangle$ when $t = 1$.

    \end{parts}

\dwrspace{2}

\newpage

\question Raging at life's injustices, Dr.\ Rosoff hurls his empty coffee mug at an angle of $60^{\circ}$ (measured up from the ground). It lands $20 \sqrt{3}$ meters (measured horizontally) from him. It may help to recall that in this situation, the throwing angle $\theta$, initial speed $v_0$, and horizontal mug-flight distance $x$ are related by the equation $x = (v_0^2/g) \sin 2\theta$, assuming the launch happens from ground level (improbable though this may seem). 

\begin{parts}

\part[4] With what initial speed was the fearsome container discharged? With what initial velocity? \textbf{In this problem, use} $g=10$~m/s$^2$.

\dwrspace{1}

\part[16] Make an appropriate choice of coordinates and find a vector-valued function $\mathbf{r}(t)$ that describes the position of the mug, assuming the fateful launch occurs at $t = 0$ and that $g=10$~m/s$^2$. Use your function to verify that the mug returns to launch height at $t = 2 \sqrt{3}$.

\dwrspace{2}

\bonuspart[2] Determine, by any method you like, the mug's maximum height above the fertile earth, who someday shall shelter us all in her boundless embrace. Justify your answer, again assuming that launch height is ground level.

\dwrspace{1}

\end{parts}

\newpage

\question Convert the equations or expressions as directed. Solutions without accompanying work involving the polar coordinate conversion equations will receive very little credit.

\begin{parts}
 
    \part[5] $r = 2 \csc \theta$ (polar to rectangular)

    \dwrspace{1}

    \part[5] $r = 2 \cos \theta$ (polar to rectangular)

    \dwrspace{1}

    \part[4] $y = -\sqrt{3} \cdot x$ (rectangular to polar)

    \dwrspace{1}

    \part[3] $(1, 5 \pi/4, \pi/3)$ (spherical to rectangular)

    \dwrspace{1}

    \part[3] $(1, 5 \pi/4, \pi/3)$ (spherical to cylindrical)

    \dwrspace{1}

    \bonuspart[3] $r = \cos \theta + 2 \sin \theta$ (polar to rectangular)

    \dwrspace{1}

\end{parts}

\newpage

\question Evaluate the indicated limits ($\pm \infty$ are acceptable answers), or state that they do not exist.  Briefly justify your answers. 

\begin{parts}
    
    \part[3] $\lim\limits_{(x,y) \to (2,3)} \dfrac{\sin(\pi x)}{\cos(\pi x)}$

    \dwrspace{1}

    \part[3] $\lim\limits_{(x,y) \to (-2,-1)} 
        \dfrac{2x+4y+4}{e^{(x+2)^2+(y+1)^2}}$ % was y+2 on 2012 actual; whoops

    \dwrspace{2}

\end{parts}

\question 

    \begin{parts}
    \part[8] Consider a curve $\vec{r}(t)$.% = \langle t, t^2, t^3 \rangle$.
    Suppose that when $t = 1$, the unit tangent and unit normal vectors are, respectively,
    \[
        \vec{T}(1) = \langle 1, 2, 3 \rangle, \quad
        \vec{N}(1) = \langle -11, -8, 9 \rangle.
    \]
    Give an equation (in rectangular coordinates) for the plane containing these two vectors and passing through the point $\vec{r}(1) = \langle 1, 1, 1 \rangle$.

    \dwrspace{2}

    \bonuspart[3] The plane containing $\vec{T}$ and $\vec{N}$ at any point along a parametrized curve is called the \emph{osculating plane}. What does the Latin-derived word ``osculating'' mean in English? Note that this word has no relation to the more familiar ``oscillating'', derived from a different Latin root.

    \dwrspace{1}

    \end{parts}

\end{questions}

\end{document} 