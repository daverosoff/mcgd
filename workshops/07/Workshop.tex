\documentclass[12pt]{exam}
\usepackage{cofi}
\usepackage{fourier}
\frenchspacing
\usepackage{parskip}
\usepackage{rosoff-vector-macros}
\firstpageheader{}{}{}
\runningheader{\textbf{Fall 2013}}
 {}
 {\textbf{Math 251}}
 %{\emph{Page \thepage~of \numpages}}
\runningheadrule

\pagestyle{head}
\theoremstyle{definition}
\newtheorem{example}{Example}
\begin{document}

\noindent
\textbf{{\large Math 251 \hfill Workshop 07: Modeling with higher derivatives}}
% \hfill Name: \underline{\hspace{0.5in}Answers\hspace{2in}}

\noindent
October 9, 2013 

\noindent
Due: Friday, October 11 \hfill Name: \underline{\hspace{3in}} 

\noindent

\section{Workshop 07: Modeling with higher derivatives}

This workshop is a little different from the previous ones: it is more
applied, and concerns mathematical \emph{modeling}.

You saw in \href{../../decks/08/Deck.pdf}{the presentation} how there
are four basic curve shapes for differentiable functions of a single
variable. A similar phenomenon holds for functions of two variables, but
there are too many ``basic surface shapes'' to draw and memorize. Here
is the general fact:

\begin{quote}
Every smooth surface can be pasted together from the graphs of quadric
surfaces.
\end{quote}

Quadric surfaces are graphs of functions of the form

\[z = A + Bx + Cy + Dx^2 + Exy + Fy^2.\]

The quadric modeling process for some interesting quantity $z$ would
work as follows:

\begin{enumerate}
\def\labelenumi{\arabic{enumi}.}
\itemsep1pt\parskip0pt\parsep0pt
\item
  Identify the variables $x$ and $y$ on which $z$ depends.
\item
  Use heuristic reasoning to decide which of $A$, $B$, \ldots{}, $F$ are
  nonzero.

  \begin{enumerate}
  \def\labelenumii{\alph{enumii}.}
  \itemsep1pt\parskip0pt\parsep0pt
  \item
    Usually, $A$, $B$, and $C$ will all be nonzero; $B$ and $C$ are
    almost always nonzero.
  \end{enumerate}
\item
  Use statistical techniques (nonlinear least-squares regression) to
  ``fit the data to the model'', that is, to choose the best possible
  coefficients for the existing data.
\end{enumerate}

In this workshop we focus on step 2.

\subsection{Scenario 1: An afternoon in the sun}

The number $z$ of students relaxing on the lawn on a sunny day depends
on (at least) two variables: the temperature $x$ outside, and the time
$y$ that has elapsed since the last warm, sunny day. We will try to
model $z$ as a quadric. In this first scenario, we step through the
reasoning for each coefficient.

Our analysis is purely qualitative in the sense that we are only trying
to decide whether each coefficient is zero, negative, or positive. So,
it's not really necessary to worry about units. A particular coefficient
is either zero, positive, or negative, whether we measure temperature in
kelvin or degrees, or whether we measure time in hours, days, or weeks.

\newpage

\begin{enumerate}
\def\labelenumi{\arabic{enumi}.}
\itemsep1pt\parskip0pt\parsep0pt
\item
  \textbf{The constant term $A$}. The constant term is present if there
  is a nonzero ``baseline'', that is, if there is a contribution to $z$
  that doesn't depend on the present values of $x$ and $y$. Do you think
  $A$ should be present in our model for $z$? Why or why not?
\end{enumerate}

\flexskip{1}

\begin{enumerate}
\def\labelenumi{\arabic{enumi}.}
\setcounter{enumi}{1}
\item
  \textbf{The linear coefficients $B$ and $C$}. We assume these are
  present (with nonzero values, that is). It might happen that one of
  them is 0, but empirical experience shows this just doesn't happen
  very often. (There is a geometric reason why not---see Dr.~Rosoff for
  more.)
\item
  \textbf{The ``pure'' quadratic coefficients $D$ and $F$}. To decide
  whether these should be present, we focus on the slice curves and
  think about their shapes.
\end{enumerate}

Let us first think about the coefficient $D$ of $x^2$. Fix $y$ and think
of the corresponding slice curve. \emph{Does the slice curve have a
local max or min?}

If it does, then $D$ should be present with a nonzero value (linear
functions have no optima). Do you think $D$ should be zero, negative, or
positive in our model? Why? What about $F$?

\flexskip{1}

\newpage

\begin{enumerate}
\def\labelenumi{\arabic{enumi}.}
\setcounter{enumi}{3}
\itemsep1pt\parskip0pt\parsep0pt
\item
  \textbf{The ``mixed'' coefficient $E$}. To decide whether this
  coefficient appears with a nonzero value, we think about the mixed
  partial derivative of $z$. This one is a little trickier.
\end{enumerate}

When will $\partial^2 z / \partial x \partial y$ be nonzero? This
happens when \emph{the change in $z$ corresponding to an incremental
change in $y$ depends on the present value of $x$}. To see that this is
true, observe that if said change in $z$ were instead \emph{independent}
of the present value of $x$, we would have
\[ \frac{\partial}{\partial x} \left( \frac{\partial z}{\partial y} \right) = 0. \]

But $Exy$ is the only term that can contribute to this mixed partial. It
appears or doesn't, together with the mixed partial. (Why is this true?)

Do you think $E$ should be zero, negative, or positive in our model? Why
or why not?

\flexskip{1}

\begin{enumerate}
\def\labelenumi{\arabic{enumi}.}
\setcounter{enumi}{4}
\itemsep1pt\parskip0pt\parsep0pt
\item
  What is the form of your model? (That is, what is the right-hand side
  of your equation describing $z$ in terms of $x$ and $y$?)
\end{enumerate}

If we had some empirical data, we could determine the statistical ``best
possible'' values of the nonzero coefficients to fit it.

\flexskip{1}

\newpage

\subsection{Scenario 2: Gettin' juiced}

The demand $z$ for orange juice depends on (at least) two variables: the
price $x$ of orange juice and the price $y$ of apple juice, a competing
good. (Again, don't worry about units here; the coefficients would have
to come with some suitable dimensions to make the units of $z$ work out
properly. Such considerations are beyond the scope of this workshop.)
Carry out the steps as above, and give a quadric model for $z$ in terms
of $x$ and $y$.

\flexskip{1}

\subsection{Scenario 3: The \emph{Einstellung} effect}

The \href{http://en.wikipedia.org/wiki/Einstellung_effect}{Einstellung
effect} is a well-documented behavioral phenomenon, aptly described as
``the negative effect of previous experience when solving new
problems''.

From the abstract to a recent
submission\footnote{Sheridan H, Reingold EM (2013) The Mechanisms and Boundary Conditions of the Einstellung Effect in Chess: Evidence from Eye Movements. PLoS ONE 8(10): e75796. doi:10.1371/journal.pone.0075796}
to the free and open journal PLoS ONE:

\begin{quote}
In a wide range of problem-solving settings, the presence of a familiar
solution can block the discovery of better solutions (i.e., the
Einstellung effect). To investigate this effect, we monitored the eye
movements of expert and novice chess players while they solved chess
problems that contained a familiar move (i.e., the Einstellung move), as
well as an optimal move that was located in a different region of the
board.
\end{quote}

The fraction of time a player's eyes linger on the familiar Einstellung
move depends on the skill of the player and on the quality of the
Einstellung move. Assume that move quality is a continuum, with
``optimal'' at one end and ``catastrophic blunder'' at the other.

What do you think a model for this phenomenon would look like?

\flexskip{1} 

\end{document}