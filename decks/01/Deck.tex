\documentclass[11pt,ignorenonframetext,aspectratio=169]{beamer}
\usetheme{Szeged}
\usecolortheme{crane}
\usefonttheme{structurebold}
\usefonttheme{professionalfonts}
\usepackage{amssymb,amsmath}
\usepackage{ifxetex,ifluatex}
\usepackage{fixltx2e} % provides \textsubscript
\ifxetex
  \usepackage{fontspec,xltxtra,xunicode}
  \defaultfontfeatures{Mapping=tex-text,Scale=MatchLowercase}
\else
  \ifluatex
    \usepackage{fontspec}
    \defaultfontfeatures{Mapping=tex-text,Scale=MatchLowercase}
  \else
    \usepackage[utf8]{inputenc}
  \fi
\fi
\usepackage{listings}

% Comment these out if you don't want a slide with just the
% part/section/subsection/subsubsection title:
% \AtBeginPart{
%   \let\insertpartnumber\relax
%   \let\partname\relax
%   \frame{\partpage}
% }
% \AtBeginSection{
%   \let\insertsectionnumber\relax
%   \let\sectionname\relax
%   \frame{\sectionpage}
% }
% \AtBeginSubsection{
%   \let\insertsubsectionnumber\relax
%   \let\subsectionname\relax
%   \frame{\subsectionpage}
% }

\setlength{\parindent}{0pt}
\setlength{\parskip}{6pt plus 2pt minus 1pt}
\setlength{\emergencystretch}{3em}  % prevent overfull lines
\setcounter{secnumdepth}{0}

%%% begin dwr insert
\usepackage{patchcmd}
\usepackage{tabulary}   % Support longer table cells
\usepackage{booktabs}   % Support better tables
\usepackage[sort&compress]{natbib}

\usepackage{framed}     % Allow background color for images
\definecolor{shadecolor}{named}{white}

%\usepackage{paralist}
\usepackage{xparse}
\usepackage{subfigure}
\usepackage{hyperref}
%%% end dwr insert
\title{Squares of distances and 2-variable functions}
\author{Math 275 Calculus 3}
\date{September 9, 2013 }

\usepackage{siunitx}

\begin{document}
\frame{\titlepage}

\section{Squares of distances}

\begin{frame}\frametitle{Distances in the plane}

\begin{itemize}[<+->]

\item
  Find the distance between $(x_1, y_1), (x_2, y_2)$ in the plane:
\end{itemize}

\[ d((x_1, y_1), (x_2, y_2)) = \sqrt{(y_2 - y_1)^2 + (x_2 - x_1)^2} \]

Works because $(x_1, y_1), (x_2, y_2)$ are the endpoints of the
hypotenuse of a right triangle.

\end{frame}

\begin{frame}\frametitle{Cleaning up the square root}

\begin{itemize}[<+->]
\item
  Often better to work with squares of distances
\item
  This is because there are no square roots involved
\end{itemize}

\[ d^2 = (y_2 - y_1)^2 + (x_2 - x_1)^2 \]

\begin{itemize}[<+->]

\item
  Because two positive numbers are equal if and only if their squares
  are equal.
\end{itemize}

\end{frame}

\section{Distance from a point to an axis}

\begin{frame}\frametitle{Coordinate planes and axes}

\begin{itemize}[<+->]
\item
  Planes $x = 0$, $y = 0$, $z = 0$ are called the \emph{coordinate}
  planes: the $(y,z)$-plane, $(x,z)$-plane, and $(x,y)$-plane,
  respectively
\item
  Intersect any pair of coordinate planes, we get a line.
\end{itemize}

Intersection of the $(x,z)$-plane with the $(y,z)$-plane is a line whose
points evidently all satisfy $y = x = 0$. This line is called the
$z$-axis.

\end{frame}

\begin{frame}\frametitle{Workshop 00: Distances to axes in the plane}

Measuring distance from a point to an \emph{arbitrary} line sucks, but
if the line is a coordinate axis, it's easy.

\begin{itemize}[<+->]

\item
  What's the distance from $(-4,3)$ to the $x$-axis?
\item
  The $y$-axis?
\end{itemize}

\end{frame}

\begin{frame}\frametitle{Distances in space}

If coordinates of a point in $\mathbf{R}^2$ measure distances to axes,
what do coordinates of $(2,1,3)$ measure?

\begin{quote}
The distance from the complementary plane.
\end{quote}

\end{frame}

\begin{frame}\frametitle{Distances from axes}

It's fine that coordinates tell us distances from the coordinate planes,
but what about from the axes?

\begin{quote}
Axes are a more familiar way of picturing points' ``addresses''
\end{quote}

Imagine looking straight down at the $(x,y)$-plane, so that the positive
$z$-axis goes right between your eyes.

\end{frame}

\begin{frame}\frametitle{Right between the eyes}

This looks just like the ordinary plane! Here is the most important
metamathematical technique there is. You have used it hundreds of times
already.

\begin{quote}
Replace your problem by an easier problem that has the same solution.
\end{quote}

That's what we're doing when we visualize the ordinary plane as a
cross-section of space this way.

\end{frame}

\begin{frame}\frametitle{The distance to the $z$-axis}

Now what's the distance to the $z$-axis? Think in terms of the
cross-sectional picture.

\begin{quote}
Notice how the formula has no $z$ in it.
\end{quote}

\end{frame}

\begin{frame}[fragile]\frametitle{2-variable functions}

A 2-variable function is a rule $f$ that associates a number, called
$f(x,y)$, to each point $(x,y)$ in the ordinary plane.

\begin{verbatim}
cost = trip_cost + s*(shirt_price) + t*(trou_price)
bill = 10 + 4/100*(pages)*(copies) + 75/100*(copies)
\end{verbatim}

\[ f(x,y) = x + yx^4 \]

\end{frame}

\begin{frame}\frametitle{Contour plots}

One way to visualize 2-variable functions is with contour plots.

\begin{itemize}[<+->]

\item
  Each $(x,y)$ gets a value $f(x,y)$
\item
  Connect points whose values are the same
\item
  The ``contours'' are the connecting lines
\end{itemize}

\begin{quote}
In regions where contours are far apart, the values change slowly. If
contours are closely spaces, values are changing rapidly.
\end{quote}

\end{frame}

\end{document}
