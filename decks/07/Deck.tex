\documentclass[11pt,ignorenonframetext,]{beamer}
\usetheme{Szeged}
\usecolortheme{crane}
\usefonttheme{structurebold}
\usepackage{amssymb,amsmath}
\usepackage{ifxetex,ifluatex}
\usepackage{fixltx2e} % provides \textsubscript
\ifxetex
  \usepackage{fontspec,xltxtra,xunicode}
  \defaultfontfeatures{Mapping=tex-text,Scale=MatchLowercase}
\else
  \ifluatex
    \usepackage{fontspec}
    \defaultfontfeatures{Mapping=tex-text,Scale=MatchLowercase}
  \else
    \usepackage[utf8]{inputenc}
  \fi
\fi
\usepackage{listings}

% Comment these out if you don't want a slide with just the
% part/section/subsection/subsubsection title:
% \AtBeginPart{
%   \let\insertpartnumber\relax
%   \let\partname\relax
%   \frame{\partpage}
% }
% \AtBeginSection{
%   \let\insertsectionnumber\relax
%   \let\sectionname\relax
%   \frame{\sectionpage}
% }
% \AtBeginSubsection{
%   \let\insertsubsectionnumber\relax
%   \let\subsectionname\relax
%   \frame{\subsectionpage}
% }

\setlength{\parindent}{0pt}
\setlength{\parskip}{6pt plus 2pt minus 1pt}
\setlength{\emergencystretch}{3em}  % prevent overfull lines
\setcounter{secnumdepth}{0}

%%% begin dwr insert
\usepackage{patchcmd}
\usepackage{tabulary}   % Support longer table cells
\usepackage{booktabs}   % Support better tables
\usepackage[sort&compress]{natbib}

\usepackage{framed}     % Allow background color for images
\definecolor{shadecolor}{named}{white}

%\usepackage{paralist}
\usepackage{xparse}
\usepackage{subfigure}
\usepackage{hyperref}
%%% end dwr insert
\usepackage{rosoff-vector-macros}
\usepackage{rosoff}
\title{Tangent planes and differentials}
\author{Math 251 Calculus 3}
\date{October 1, 2013}


\begin{document}
\frame{\titlepage}

\section{Tangent planes and differentials}

\begin{frame}\frametitle{Recap: slices, partial derivatives, and tangent
lines}

We saw previously that if $f(x,y)$ is a function of two variables, each
of its partial derivatives $f_x(x,y)$ and $f_y(x,y)$ gives the slopes of
tangent lines to slice curves.

If $(a, b)$ is a point in the plane, the slice curves through $(a,b)$
are the graphs of $z = f(a, y)$ (in the plane $x = a$) and $z = f(x, b)$
(in the plane $y = b$).

\end{frame}

\begin{frame}\frametitle{Partial derivatives}

If the slice functions $f(a, y)$ and $f(x,b)$ are differentiable (in the
one-variable sense), their derivatives are $f_y(a,y)$ and $f_x(x,b)$.
These functions are ordinary derivatives, so they compute tangent slopes
in the usual way.

\begin{itemize}[<+->]
\itemsep1pt\parskip0pt\parsep0pt
\item
  The tangent line equations:
\item
  $z = f(a, b) + f_x(a,b)(x - a)$
\item
  $z = f(a, b) + f_y(a,b)(y - b)$
\end{itemize}

This is the old tangent line approximation formula, just twice in
different directions.

\end{frame}

\begin{frame}\frametitle{A candidate for the tangent plane}

The tangent lines in the $(x,y)$-direction determine a unique plane. Its
equation is \[ z = f(a,b) + f_x(a,b)(x-a) + f_y(a,b)(y - b). \]

Rearranging this a bit, we can confirm it's a plane:
\[ \angl{f_x(a,b), f_y(a,b), -1} \cdot \angl{x - a, y - b, z - f(a,b)} = 0. \]

By 12.5, this is the equation of a plane passing through the point
$(a, b, f(a, b))$ normal to the vector $\angl{f_x(a,b), f_y(a,b), -1}$.

We call the RHS $L(x,y)$ the \emph{linearization} of $f$ at the point
$(a,b)$.

\end{frame}

\begin{frame}\frametitle{Local linearity}

Having a tangent plane means the graph of the surface is \emph{locally
linear}. If we zoom way in, the surface looks flat. The algebraic
criterion for this turns out to be
\[ \lim_{(x, y) \to (a, b)} \frac{f(x, y) - L(x,y)}{\norm{\angl{x,y} - \angl{a,b}}} = 0\]
and so we adopt this as the definition of \emph{differentiability}.

\end{frame}

\begin{frame}\frametitle{Three ways to say it}

All of the following are equivalent:

\begin{itemize}[<+->]
\itemsep1pt\parskip0pt\parsep0pt
\item
  $f(x,y)$ is differentiable at $(a,b)$
\item
  $f(x,y)$ is locally linear at $(a,b)$
\item
  the graph of $L(x,y)$ is tangent to the graph of $f$ at $(a,b)$
\end{itemize}

\end{frame}

\begin{frame}\frametitle{Criteria for differentiability}

If \emph{any} plane is tangent to the graph of $f$ at $(a,b)$, it must
be the graph of $L(x,y) = f(a,b) + f_x(a,b)(x-a) + f_y(a,b)(y - b)$.
That's the only plane that contains the tangent lines.

\begin{quote}
Observe that $L(a, b) = f(a, b)$ and that the $x = a$ (resp. $y = b$)
slice of $L(x,y)$ is the tangent line to the slice curve in the plane
$x = a$ (resp. $y = b$).
\end{quote}

How do we know the tangent plane exists?

\begin{quote}
If $f_x$ and $f_y$ exist and are continuous at $(a,b)$, then the three
equivalent conditions given above are all true.
\end{quote}

\end{frame}

\begin{frame}\frametitle{The linearization approximates $f$}

If $(x,y)$ is near $(a,b)$ (in other words, if the magnitude
$\norm{\angl{x,y} - \angl{a,b}}$ is small), then
$L(x,y) \approx f(x,y)$, because $f$ is locally linear.

Let us write

\begin{itemize}[<+->]
\itemsep1pt\parskip0pt\parsep0pt
\item
  $\Delta x = x - a$,
\item
  $\Delta y = y - b$, and
\item
  $\Delta f = f(x,y) - f(a,b)$.
\end{itemize}

\end{frame}

\begin{frame}\frametitle{Introducing: differentials}

Then

\begin{align*}
\Delta f &= f(x,y) - f(a,b) \\
         & \approx L(x,y) - L(a, b) \\
         & \approx f_x(a,b)\Delta x + f_y(a,b) \Delta y, 
\end{align*}

at least when $\Delta x$ and $\Delta y$ are small.

\end{frame}

\begin{frame}\frametitle{Differentials and linearization}

The approximation:
$\Delta f \approx f_x(a,b) \Delta x + f_y(a,b) \Delta y$.

It is common to write

\begin{itemize}
\itemsep1pt\parskip0pt\parsep0pt
\item
  $dx = \Delta x$
\item
  $dy = \Delta y$
\item
  $df = L(x,y) - L(a,b) = f_x(a,b) dx + f_y(a,b) dy$.
\end{itemize}

Putting it all together, we have $\Delta f \approx df$, a compact
representation of the idea of the linearization. The symbols $dx$, $dy$,
$df$ are often called ``differentials''.

\end{frame}

\begin{frame}\frametitle{What are differentials?}

It is common to write $dx = \Delta x$, $dy = \Delta y$,
$df = L(x,y) - L(a,b) = f_x(a,b) dx + f_y(a,b) dy$.

For now, it's best to think of them as convenient abbreviations. The
convenience lies in that while $dx = x-a$, the $a$ is suppressed from
the notation. So we can abbreviate even further

\[ df = f_x dx + f_y dy = \frac{\partial f}{\partial x} dx + \frac{\partial f}{\partial y} dy. \]

Of course, one has to remember to evaluate the partials at the
appropriate point, but that detail aside, this is quite a compact form
for the equation of the tangent plane (supposing it exists).

\end{frame}

\begin{frame}\frametitle{Abstract nonsense concerning derivatives}

We've talked a lot by now about the \emph{partial} derivatives of
$f(x,y)$. So what about its actual \emph{derivative}? The best answer,
for now, is that the derivative of $f(x,y)$ is the differential $df$.
This is a little peculiar, at first. We won't have much occasion to use
this particular notion.

\end{frame}

\begin{frame}\frametitle{Nonsense, II}

Another answer might be that the derivative is the vector
$\Angl{\frac{\partial f}{\partial x},\frac{\partial f}{\partial y}}$.
This is satisfying, because this vector seems related to the normal to
the tangent plane, something like a ``tangent slope''. This vector is
the \emph{gradient vector} of the function $f$ and we'll see a lot more
of it.

\end{frame}

\begin{frame}\frametitle{Summary}

\begin{itemize}[<+->]
\itemsep1pt\parskip0pt\parsep0pt
\item
  If the partial derivatives $f_x(a,b)$ and $f_y(a,b)$ exist, we can
  discuss the linearization $L(x,y)$
\item
  $L(x,y) = f(a,b) + f_x(a,b)(x-a) + f_y(a,b)(y-b)$ is the linearization
  of $f$ at $(a,b)$
\item
  The graph of $L$ is a plane passing through $(a,b,f(a,b))$
\item
  If $f$ is locally linear (=differentiable), then the graph of $L$ is
  tangent to the graph of $f$
\end{itemize}

\end{frame}

\end{document}
