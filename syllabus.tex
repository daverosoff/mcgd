\documentclass[12pt,twoside]{amsart}
\usepackage{graphicx,wrapfig,amsmath,amsthm,amssymb,latexsym}
\usepackage{fourier, MnSymbol}
\DeclareGraphicsExtensions{.png}
\DeclareMathOperator{\im}{im}
\newcommand{\id}[1][{}]{\mathrm{Id}_{#1}}
\newcommand{\Q}{\mathbf Q}      % adds blackboard bold macros for
\newcommand{\R}{\mathbf R}      % conventional notation for
\newcommand{\Z}{\mathbf Z}      % rationals, reals, integers,
\newcommand{\C}{\mathbf C}      % complexes, naturals, field of p
\newcommand{\N}{\mathbf N}      % elements, p-adic integers
\newcommand{\D}{\mathbf D}
\newcommand{\Zp}{\mathbb{Z}_p}      % and numbers.
\newcommand{\Qp}{\mathbb{Q}_p}
\newcommand{\Fp}{\mathbb{F}_p}
\newcommand{\ie}{\emph{i.e.}}
\newcommand{\eg}{\emph{e.g.}}
\newcommand{\ind}[2]{[#1:#2]}       % index of groups, degree of field
\newcommand{\aut}[2][{}]{\mathrm{Aut}_{#1}(#2)}  % automorphism group
\newcommand{\isom}{\cong}       % isomorphic
\newcommand{\fk}[1]{\mathfrak{#1}}  % fraktur shorthand
\newcommand{\abs}[2][{}]{|#2|_{#1}} % generalized absolute value
\newcommand{\ord}[2][{}]{v_{#1}(#2)}    % roman ord(x), ord_p(x)
\newcommand{\Ex}{\begin{ex}}
\newcommand{\Eex}{\end{ex}}
\newcommand{\tr}[1]{#1^t}
\newcommand{\mat}[2]{M_{#1}(#2)}
\newcommand{\GL}[2]{\mathrm{GL}_{#1}(#2)}
\newcommand{\Lie}[1]{\mathrm{Lie}(#1)}
\newcommand{\ddt}[1]{\left. \frac{\mathrm{d}}{\mathrm{d}t} \right|_{t=#1}}
\newcommand{\ten}{\otimes}
\newcommand{\mg}[1]{{#1}^{\times}}
\newcommand{\wt}[1]{\widetilde{#1}}
\newcommand{\Tor}[2]{\mathrm{Tor}^{#1}_{#2}}

\theoremstyle{plain}            % define numbered and naked theorem
\newtheorem{ntheorem}{Theorem}[section]     % environments, numbered
\newtheorem{nlemma}[ntheorem]{Lemma}        % consecutively within
\newtheorem{nprop}[ntheorem]{Proposition}   % chapters
\newtheorem{ncor}[ntheorem]{Corollary}
\newtheorem*{theorem}{Theorem}
\newtheorem*{lemma}{Lemma}
\newtheorem*{prop}{Proposition}
\newtheorem*{cor}{Corollary}

% \newcounter{ntheorem}

\theoremstyle{definition}
\newtheorem{ndefn}[ntheorem]{Definition}
\newtheorem{nex}[ntheorem]{Example}
\newtheorem*{defn}{Definition}
\newtheorem*{ex}{Example}

%\theoremstyle{remark}
\newtheorem*{rmk}{Remark}
\newtheorem*{ntn}{Notation}

\usepackage[top=1.0in,bottom=1.0in,left=1.0in,right=1.0in]{geometry}
\setlength{\parskip}{1.2ex}
\setlength{\parindent}{0pt}
\title{{\LARGE Mathematics 251 \hfill Fall 2013}}
\pagestyle{empty}
\usepackage{hyperref}
\begin{document}
\maketitle
\thispagestyle{empty}
\vspace*{-2ex}
\begin{minipage}[t]{0.45\linewidth}
    \textbf{Instructor}: Dr.\ Dave Rosoff  \\
    \textbf{Office}: Boone Hall 102C \\
    \textbf{Office hours}: TBD, or by appointment \hspace*{0.25in}\\
    \textbf{Email}: \href{mailto:drosoff@collegeofidaho.edu}{drosoff@collegeofidaho.edu} \\
\end{minipage} 
\hspace*{0.09\linewidth}
\begin{minipage}[t]{0.45\linewidth}\begin{flushright}
    \textbf{Website}: \url{https://zeus.collegeofidaho.edu/academics/MathPhysics/courses/MAT-251/schedule.html}
\end{flushright}
\end{minipage}
\vspace*{-2ex}
\begin{center}

{\large \emph{The pursuit of knowledge, brother, is the askin' of many questions}\footnote{Raymond Chandler, \emph{Farewell, My Lovely}.}}.

\end{center}
\textbf{Text}: The text is \emph{Calculus: Early Transcendentals} by Jon Rogawski, second edition. It is OK if you have the first edition, although your section numbers may be different. A second, recommended text is \emph{Start R with Calculus}, by Daniel Kaplan, available from Amazon.com.

\textbf{Catalog description}: A study of real functions of several real variables. Topics include differentiability and continuity, differential geometry, extrema, Lagrange multipliers, multiple integration, line and surface integrals, and the theorems of Green, Gauss and Stokes. 

\textbf{Course overview}: We generalize the main ideas and results of single-variable calculus to multivariate situations, with particular attention to applications and modeling. 
%There are two ways to proceed. First, we keep the domains of our functions the usual real numbers (or appropriate subsets thereof) and let their codomains (ranges) be higher-dimensional Euclidean spaces. This approach entails an investigation into the related concepts of \emph{parametrization} and \emph{vectors}. We may also ask what happens in the reverse scenario, when the domain of the functions at hand is allowed to be 2-, 3-, or higher-dimensional, and the function's values are real numbers in the usual sense. 
Here we will meet the essential concepts of \emph{partial derivative} and \emph{total} or \emph{Jacobian derivative} and revisit the familiar themes of differential calculus (related rates, optimization, etc.). We will introduce the powerful modern algebraic framework of \emph{vectors} in which the rest of the theory is cast.
Just as functions of several variables may be differentiated via their partial derivatives, so too can they be integrated over appropriate \emph{regions} (rather than intervals) in their domains. We study the important change-of-variable theorem that allows integrals over complicated regions to be reckoned in terms of simpler ones (e.g., rectangles) and examine some applications. 

We then begin a study of the important field of \emph{differential geometry}. The laws of electromagnetism, Maxwell's equations, are formulated in this language. We will make as much progress as we can toward the fundamental theorems of G.\ Green, C.\ F.\ Gauss, and G.\ Stokes\footnote{Famously, Stokes's theorem is due not to Stokes, but to Lord Kelvin; Stokes merely \emph{assigned} it, in 1854.}. All three are vast generalizations of the usual Fundamental Theorem of Calculus, and all have extremely important implications for physics, engineering, and the rest of mathematics.

\textbf{Homework}: Homework in this class comprises both online and traditional written assignments.
\begin{itemize}
    \item WeBWorK assigned daily: \url{https://webwork.collegeofidaho.edu/webwork2/MAT251_01_F13/}
    \item \emph{Portfolio problems}, assigned and collected regularly throughout the term. These problems mostly come from the textbook and are intended to stretch your thinking beyond the level of the online homework.
\end{itemize}
Whenever you are writing a solution to a math problem, it is important to strive for the clearest exposition you can manage. Good mathematical writing is essential for anyone who wishes to think clearly about mathematics. The process of making your ideas and reasoning \emph{clear, complete, and unambiguously correct} is the most effective amplifier of mathematical power there is. Hence your solutions should be composed in brilliant English prose. This means employing accepted scientific usage, more or less correct grammar and spelling, and above all \emph{complete sentences}---sprinkled here and there with tangy, delicious equations. Solutions in the popular ``pile-of-equations'' style with little or no explanatory text will not get much credit. You must explain what is happening as the action unfolds. I encourage all students to form study groups and collaborate on homework; each student is of course individually responsible for their own work. Collaborators must be acknowledged. Portfolio problems may be submitted more than once and scores are not final until the last day of class. \emph{No portfolios will be accepted after the last day of class without a compelling, documented reason.}

\textbf{Presentations}: Each student should present a portfolio problem to the class six times over the semester, in a group of two to four. A presentation consists of an explanation, a solution, and any justification of the solution requested by the class. I will maintain a list of problems together with when they were presented on my office door and online for your reference.

\textbf{Quizzes}: To make sure each student is keeping up with the online problems, we will have frequent quizzes. Quiz problems are simple variants of WeBWorK problems and all quizzes are open-note. Therefore, if you have solved the WeBWorK problems prior to the quiz day, a score of 100\% is very possible.

\textbf{Exams}: Three exams are given in class (see below for dates). Exams comprise both portfolio problems and new problems intended to stretch your thinking. During the exam, you may refer freely to your portfolio. A missed exam results in an exam grade of zero. Arrangements for absences, again, must be made well in advance (two weeks suffices). \emph{If arrangements are not made in advance, I will consider make-ups only with compelling, documented reasons.} 
\begin{itemize}
	\item Exam 1 (tentative): Tuesday, September 24
	\item Exam 2 (tentative): Tuesday, October 15
    \item Exam 3 (tentative): Tuesday, November 19
	\item Final Exam (cannot be rescheduled \emph{for any reason}): Monday, December 9, 1:30--4:30
\end{itemize}

\textbf{Grading}: Scores are computed as a weighted average, with the following weights: WeBWorK $0.10 = 10\%$, presentations $0.10 = 10\%$, portfolios $0.10 = 10\%$, quizzes $0.30 = 30\%$, three in-class exams $0.27 = 27\%$, and final exam $0.13 = 13\%$. Observe that the weights sum to $1 = 100\%$. The exact determination of letter grades from these scores depends on the final distribution of scores in the class, but you can expect a C for earning 75\% of the points, a C+ for 78\%, a B-- for 80\%, and so on.

\textbf{Academic integrity}: Students are expected to complete all graded work in accordance with the College Honor Code. Plagiarism, cheating, or borrowing without proper credit will not be tolerated.  Violations of academic honesty can result in loss of credit on an assignment, failure on an exam, or failure in the course. A referral will be made to the Vice President for Academic Affairs for all parties involved in academic dishonesty.

% \textbf{A note on studying math}: By now you have studied enough mathematics to have learned something about how it is that the material passes through your shapely skull and into your soft, spongy brain. Nevertheless, you may find that this course is rather more difficult than your previous calculus courses. To really understand it, we will have to dig into subtle distinctions and nuances that no one has asked you to think about before. The reasons for this are outlined above: in the one-dimensional world, they simply do not signify. It is also more difficult to picture what is going on mentally, again owing to the presence of extra dimensions. Part of what I'm here to tell you is that while the material may seem wholly new and unfamiliar, the underlying principles of calculus (what do derivatives \emph{do}? what are integrals \emph{for}?) are immutable.

\textbf{Special accommodations}: Students who have documented disabilities as addressed by the Americans With Disabilities Act and who need any test or course materials to be furnished in an alternative format should notify me immediately (during the first week of class).  Reasonable efforts will be made to accommodate the needs of such students.

\vspace*{1ex}

\begin{center}
\emph{ {\LARGE Good luck this semester!} }
\end{center}

\end{document}
