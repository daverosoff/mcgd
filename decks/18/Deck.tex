\documentclass[11pt,ignorenonframetext,aspectratio=169,xcolor={svgnames}]{beamer}
\usetheme{Szeged}
\usecolortheme{crane}
\usefonttheme{structurebold}
\usefonttheme{professionalfonts}
\usepackage{amssymb,amsmath}
\usepackage{ifxetex,ifluatex}
\usepackage{fixltx2e} % provides \textsubscript
\ifxetex
  \usepackage{fontspec,xltxtra,xunicode}
  \defaultfontfeatures{Mapping=tex-text,Scale=MatchLowercase}
\else
  \ifluatex
    \usepackage{fontspec}
    \defaultfontfeatures{Mapping=tex-text,Scale=MatchLowercase}
  \else
    \usepackage[utf8]{inputenc}
  \fi
\fi
\usepackage{listings}

% Comment these out if you don't want a slide with just the
% part/section/subsection/subsubsection title:
% \AtBeginPart{
%   \let\insertpartnumber\relax
%   \let\partname\relax
%   \frame{\partpage}
% }
% \AtBeginSection{
%   \let\insertsectionnumber\relax
%   \let\sectionname\relax
%   \frame{\sectionpage}
% }
% \AtBeginSubsection{
%   \let\insertsubsectionnumber\relax
%   \let\subsectionname\relax
%   \frame{\subsectionpage}
% }

\setlength{\parindent}{0pt}
\setlength{\parskip}{6pt plus 2pt minus 1pt}
\setlength{\emergencystretch}{3em}  % prevent overfull lines
\setcounter{secnumdepth}{0}

%%% begin dwr insert
\usepackage{patchcmd}
\usepackage{tabulary}   % Support longer table cells
\usepackage{booktabs}   % Support better tables
\usepackage[sort&compress]{natbib}

\usepackage{framed}     % Allow background color for images
\definecolor{shadecolor}{named}{white}

%\usepackage{paralist}
\usepackage{xparse}
\usepackage{subfigure}
\usepackage{hyperref}
%%% end dwr insert
\usepackage{rosoff-vector-macros}
\usepackage[beamer]{cofi}
\title{Integrating vector fields along curves}
\author{Math 275 Calculus 3}
\date{November 20, 2013}


\begin{document}
\frame{\titlepage}

\section{Integrating vector fields along curves}

\begin{frame}\frametitle{Warm-up}

Set up the integral of the function $f(x,y) = 2x^2 y$ along the arc of
the parabola $y = x^2$ connecting $(0,0)$ to $(2,4)$.

\begin{itemize}[<+->]
\itemsep1pt\parskip0pt\parsep0pt
\item
  Tip: Recall that you can parametrize the graph of any function
  $y = g(x)$ as $\vec{r}(t) = \angl{t, g(t)}$.
\item
  Answer: Let $\vec{r}$ as above; then
  $\norm{\angl{\vec{r}'(t)}} = \sqrt{1+t^2}$. We also have
  $f(x(t),y(t)) = 2(t)^2 \cdot t^2 = 2t^4$ for points on the parabola.
  The parameter interval is $0 \leq t \leq 2$, so our integral should
  be\ldots{}
\item
  $\int_0^2 2t^4 \sqrt{1+t^2} \; dt$
\item
  This integral is hideously wretched; its value is
  $\frac{1}{24} \left( 266 \sqrt{5} + 3 \sinh^{-1}(2) \right) \approx 24.9635$.
\end{itemize}

\end{frame}

\begin{frame}\frametitle{Integrating vector fields along paths in the
plane}

The motivating example is a force field $\vec{F}(x,y)$ that acts
throughout some region in the plane. If an object is moved by this force
along the path $\mathcal{C}$, we want to call the work done

\begin{equation*}
    \int_{\mathcal{C}} \vec{F} \cdot d\vec{s}.
\end{equation*}

In the text these vectors are denoted with boldface as opposed to
arrows, but the meaning is the same. What is this meaning? On Tuesday
(the 19th) we observed that a good definition for $ds$ would be
$\norm{\vec{r}'(t)} \; dt$.

The scalar line element $ds$ is an incremental bit of arclength along
the curve. The vector line element $d\vec{s}$ is an incremental tangent
vector to the curve.

\end{frame}

\begin{frame}\frametitle{Interpreting the differential}

We set $d\vec{s} = \frac{\vec{r}'(t)}{\norm{\vec{r}'(t)}} ds$.

The tangent vector $\vec{r}(t)$ is the velocity vector for the curve.
When we divide it by its length, we get a pure direction vector: the
quotient is tangent to the curve and has length 1. When we multiply by
the ``incremental'' scalar $ds$, we get an ``incremental'' vector.

But $ds = \norm{\vec{r}'(t)} dt$.

Hence $d\vec{s} = \vec{r}'(t) dt$. Observe that there is no magnitude
operator and hence no square root. For this reason line integrals of
vector fields tend to be more tractable for hand calculation than do
line integrals of scalar functions.

\end{frame}

\begin{frame}\frametitle{Integrals of vector fields along parametrized
curves}

Putting it all together, if $\vec{r}(t)$ is our parametrized curve with
parameter interval $[a,b]$, we can compute the line integral of the
vector field $\vec{F}$ by the formula

\begin{equation*}
    \int_{\mathcal{C}} \vec{F} \cdot d\vec{s} = \int_{a}^b \vec{F}(\vec{r}(t)) \cdot \vec{r}'(t) \; dt.
\end{equation*}

\end{frame}

\begin{frame}\frametitle{The geometric meaning of the dot product}

Remember that the dot product of two vectors depends on their lengths
and the angle $\theta$ between them. In particular, if the vectors are
orthogonal, the dot product is zero. You might also remember:

\begin{itemize}
\itemsep1pt\parskip0pt\parsep0pt
\item
  If $\theta$ is acute, the dot product is positive, while
\item
  if $\theta$ is obtuse, the dot product is negative.
\end{itemize}

Dotting the vector field with the tangent vector to the curve produces
the component of the field in the direction of the curve.

\end{frame}

\begin{frame}\frametitle{Another convenient notation}

Since $d\vec{s}$ is a vector, it has components. We recognize them as
the familiar differentials $dx$ and $dy$ as follows. Let
$\vec{F}(x,y) = \angl{F_1(x,y), F_2(x,y)}$ and
$\vec{r}(t) = \angl{x, y}$. Then as we have seen (note the
$t$-dependence is suppressed for readability)

\begin{align*}
    \vec{F} \cdot d\vec{s} &= \vec{F}(\vec{r}(t)) \cdot \vec{r}'(t) \; dt \\
                           &= \angl{F_1(x, y), F_2(x, y)} \cdot \angl{x', y'} \; dt \\
                           &= F_1(x, y) x'(t) \; dt + F_2(x, y) y'(t) \; dt.
\end{align*}

We recognize some old friends: $dx = x'(t) \; dt$ and
$dy = y'(t) \; dt$. Evidently, we have obtained

\begin{align*}
    \vec{F} \cdot d\vec{s} &= F_1 \; dx + F_2 \; dy.
\end{align*}

\end{frame}

\begin{frame}\frametitle{Properties of line integrals}

\begin{enumerate}
\def\labelenumi{\arabic{enumi}.}
\itemsep1pt\parskip0pt\parsep0pt
\item
  \emph{Linearity}:
  $\displaystyle\int_{\mathcal{C}} (\vec{F} + \vec{G}) \cdot d\vec{s} = \int_{\mathcal{C}} F \cdot d\vec{s} + \int_{\mathcal{C}} G \cdot d\vec{s}$,
  $\displaystyle \int_{\mathcal{C}} (k\vec{F}) \cdot d\vec{s} = k\int_{\mathcal{C}} \vec{F} \cdot d\vec{s}$
\item
  \emph{Orientation}:
  $\displaystyle\int_{-\mathcal{C}} \vec{F} \cdot d\vec{s} = -\int_{\mathcal{C}} \vec{F} \cdot d\vec{s}$
\item
  \emph{Additivity}: If $\mathcal{C}$ is a union of $n$ smooth curves
  $\mathcal{C}_1, \ldots, \mathcal{C}_n$, then
  $\displaystyle \int_{\mathcal{C}} \vec{F} \cdot d\vec{s} = \int_{\mathcal{C}_1} \vec{F} \cdot d\vec{s} + \cdots + \int_{\mathcal{C}_n} \vec{F} \cdot d\vec{s}$
\end{enumerate}

\end{frame}

\begin{frame}\frametitle{Generalization to three dimensions}

It is pretty straightforward. All the two-dimensional vectors become
three-dimensional, but the dimension-free formulas stay the same. Refer
to the text for the complete details.

\end{frame}

\end{document}
