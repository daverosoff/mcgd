\documentclass[11pt,ignorenonframetext,aspectratio=169,xcolor={svgnames}]{beamer}
\usetheme{Szeged}
\usecolortheme{crane}
\usefonttheme{structurebold}
\usefonttheme{professionalfonts}
\usepackage{amssymb,amsmath}
\usepackage{ifxetex,ifluatex}
\usepackage{fixltx2e} % provides \textsubscript
\ifxetex
  \usepackage{fontspec,xltxtra,xunicode}
  \defaultfontfeatures{Mapping=tex-text,Scale=MatchLowercase}
\else
  \ifluatex
    \usepackage{fontspec}
    \defaultfontfeatures{Mapping=tex-text,Scale=MatchLowercase}
  \else
    \usepackage[utf8]{inputenc}
  \fi
\fi
\usepackage{listings}

% Comment these out if you don't want a slide with just the
% part/section/subsection/subsubsection title:
% \AtBeginPart{
%   \let\insertpartnumber\relax
%   \let\partname\relax
%   \frame{\partpage}
% }
% \AtBeginSection{
%   \let\insertsectionnumber\relax
%   \let\sectionname\relax
%   \frame{\sectionpage}
% }
% \AtBeginSubsection{
%   \let\insertsubsectionnumber\relax
%   \let\subsectionname\relax
%   \frame{\subsectionpage}
% }

\setlength{\parindent}{0pt}
\setlength{\parskip}{6pt plus 2pt minus 1pt}
\setlength{\emergencystretch}{3em}  % prevent overfull lines
\setcounter{secnumdepth}{0}

%%% begin dwr insert
\usepackage{patchcmd}
\usepackage{tabulary}   % Support longer table cells
\usepackage{booktabs}   % Support better tables
\usepackage[sort&compress]{natbib}

\usepackage{framed}     % Allow background color for images
\definecolor{shadecolor}{named}{white}

%\usepackage{paralist}
\usepackage{xparse}
\usepackage{subfigure}
\usepackage{hyperref}
%%% end dwr insert
\usepackage{rosoff-vector-macros}
\usepackage[beamer]{cofi}
\usepackage{siunitx}
\title{Double integrals}
\author{Math 275 Multivariable Calculus}
\date{October 28, 2013}


\begin{document}
\frame{\titlepage}

\section{Double integrals over nonrectangular regions}

\begin{frame}\frametitle{Review and setup}

Let $f(x,y)$ be continuous on the rectangle $R = (a,b) \times (c,d)$.
Then the limit

\[ \lim_{n \to \infty} \sum_{i=1}^n f(x_i, y_i) \Delta A_i, \]

representing $n$ box-volumes with bases $\Delta A_i$ and heights
$f(x_i,y_i)$, exists regardless of how the rectangle is subdivided. The
value of this limit is, by definition,

\[ \iint_R f(x,y) \; dA. \]

\end{frame}

\begin{frame}\frametitle{Computing via iterated integrals}

We find the value of the double integral $\iint_R f(x,y) \; dA$ using
Fubini's theorem.

\[ \iint_R f(x,y) \; dA = \int_c^d \int_a^b f(x,y) \; dx \; dy = \int_a^b \int_c^d f(x,y) \; dy \; dx \]

\begin{itemize}

\item
  Observe that the order of integration is different: $dx \; dy$ is not
  the same as $dy \; dx$. The limits change accordingly.
\end{itemize}

Depending on the function, one or the other order of integration might
be easier to compute, but that is the only novelty.

\end{frame}

\begin{frame}\frametitle{The secret to thinking about integrals}

\begin{itemize}

\item
  Do you picture the graph of every 1-variable function you integrate?
\item
  Only enough to check that it is continuous on the domain of
  integration. Algebra suffices for this.
\item
  It's the same for several variables. What we need to be careful about
  picturing is the domain of integration itself.
\end{itemize}

\end{frame}

\begin{frame}\frametitle{Decoding the iterated integral}

In the order $dy \; dx$, we integrate first in the $y$-direction---that
is, along a vertical segment in the $(x,y)$-plane. The partial integral

\[ A(x) = \int_{y = c}^d f(x,y) \; dy \]

is still a function of $x$, hence the notation. What is the significance
of some value $A(x_0)$? It measures the ``contribution'' of the vertical
segment $x=x_0$ to the integral.

Thus, integrating once more over $x$, $\int_{x=a}^b A(x) \; dx$ gives
the total ``contribution'' of all the vertical segments between $x=a$
and $x=b$.

\end{frame}

\begin{frame}\frametitle{Two orders}

Thus,
$\int_R f(x,y) \; dA = \int_{x=a}^b A(x) \; dx = \int_{x=a}^b \int_{y=c}^d f(x,y) \; dy \; dx$.

Writing the ``area element'' $dA$ as $dy \; dx$ this way corresponds to
this choice of ``slicing'' the domain: first, find the contribution of a
vertical segment, then integrate up the contributions of such segments.

If we write $dA = dx \; dy$ instead, we are integrating first over
$x$---to find the contribution of a horizontal segment---and then over
$y$, to total up the contributions of such segments.

\end{frame}

\begin{frame}\frametitle{Integrating over a triangle}

Let $R$ now be the triangular region in the plane bounded by the
coordinate axes and the line $x + 2y = 2$.

\end{frame}

\end{document}
