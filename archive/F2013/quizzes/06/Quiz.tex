\documentclass[12pt]{exam}
\usepackage{xparse}
\usepackage{rosoff-vector-macros}
\usepackage{rosoff}
\usepackage{graphicx}
\DeclareGraphicsExtensions{.jpg, .png}
\usepackage{fourier}
\usepackage{MnSymbol}
\usepackage{amsthm}
\usepackage{paralist,enumerate,listings}
\usepackage{siunitx}
\frenchspacing
\usepackage{parskip}
\usepackage{hyperref}
\firstpageheader{}{}{}
\runningheader{\textbf{Fall 2013}}
 {}
 {\textbf{Math 251}}
 %{\emph{Page \thepage~of \numpages}}
\runningheadrule

\pagestyle{head}

\begin{document}
\noindent
\textbf{{\large Math 251 \hfill Quiz 06 (WeBWorK 7)}}
% \hfill Name: \underline{\hspace{0.5in}Answers\hspace{2in}}

\noindent
November 4, 2013; 10 minutes \hfill Name: \underline{\hspace{3in}} 

\noindent

\noindent
This quiz is \emph{open-note}, but no books or calculators may be used.
In calculation, you can show work at your discretion, but remember that
I can't give partial credit for calculations I can't see. Explain
anything that seems to need explaining.

\begin{questions} 

\question[18] Find the critical points of the function 
\begin{equation*}
f(x,y) = x^3 + y^4 - 6x - 16y^2.
\end{equation*}
Use the second derivative test to determine whether each critical point is a local minimum, local maximum, or saddle point.

\dwrspace{1}

\end{questions} 

\end{document}