\documentclass[11pt,ignorenonframetext,xcolor={svgnames},aspectratio=169]{beamer}
\usetheme{Szeged}
\usecolortheme{crane}
\usefonttheme{structurebold}
\usefonttheme{professionalfonts}
\usepackage{amssymb,amsmath}
\usepackage{ifxetex,ifluatex}
\usepackage{fixltx2e} % provides \textsubscript
\ifxetex
  \usepackage{fontspec,xltxtra,xunicode}
  \defaultfontfeatures{Mapping=tex-text,Scale=MatchLowercase}
\else
  \ifluatex
    \usepackage{fontspec}
    \defaultfontfeatures{Mapping=tex-text,Scale=MatchLowercase}
  \else
    \usepackage[utf8]{inputenc}
  \fi
\fi
\usepackage{listings}

% Comment these out if you don't want a slide with just the
% part/section/subsection/subsubsection title:
% \AtBeginPart{
%   \let\insertpartnumber\relax
%   \let\partname\relax
%   \frame{\partpage}
% }
% \AtBeginSection{
%   \let\insertsectionnumber\relax
%   \let\sectionname\relax
%   \frame{\sectionpage}
% }
% \AtBeginSubsection{
%   \let\insertsubsectionnumber\relax
%   \let\subsectionname\relax
%   \frame{\subsectionpage}
% }

\setlength{\parindent}{0pt}
\setlength{\parskip}{6pt plus 2pt minus 1pt}
\setlength{\emergencystretch}{3em}  % prevent overfull lines
\setcounter{secnumdepth}{0}

%%% begin dwr insert
\usepackage{patchcmd}
\usepackage{tabulary}   % Support longer table cells
\usepackage{booktabs}   % Support better tables
\usepackage[sort&compress]{natbib}

\usepackage{framed}     % Allow background color for images
\definecolor{shadecolor}{named}{white}

%\usepackage{paralist}
\usepackage{xparse}
\usepackage{subfigure}
\usepackage{hyperref}
%%% end dwr insert
\usepackage{rosoff-vector-macros}
\usepackage[beamer]{cofi}
\title{Cross products; classifying planes}
\author{Math 275 Calculus 3}
\date{September 20, 2013 }


\begin{document}
\frame{\titlepage}

\section{Another take on computation}

\begin{frame}\frametitle{Distributing and the cyclic relation}

Remember: $\ii \times \jj = \kk$, $\jj \times \kk = \ii$, and
$\kk \times \ii = \jj$. Plus, scalars operate as expected. So,

\begin{align*}
    (\ii + \jj + \kk) &\times (2 \ii + 3 \jj + 4 \kk) \\
    &= \ii \times (2 \ii + 3 \jj + 4 \kk) + \jj \times (2 \ii + 3 \jj + 4 \kk) \\
    &\qquad + \kk \times (2 \ii + 3 \jj + 4 \kk) \\
    &= 2 \ii \times \ii + 3 \ii \times \jj + 4 \ii \times \kk \\
    &+ 2 \jj \times \ii + 3 \jj \times \jj + 4 \jj \times \kk \\
    &+ 2 \kk \times \ii + 3 \kk \times \jj + 4 \kk \times \kk \\
    &= (3-2)(\ii \times \jj) + (2-4)(\kk \times \ii) + (4-3)(\jj \times \kk) \\
    &= \kk - 2 \jj + \ii = \angl{1, -2, 1}.
\end{align*}

\end{frame}

\begin{frame}\frametitle{Warm-up, II}

Find the cross products:

\begin{itemize}

\item
  $\angl{2, 1, 3} \times \angl{4, 2, 1}$
\item
  $\angl{-1, 0, 1} \times \angl{2, -1, 5}$
\end{itemize}

\end{frame}

\begin{frame}\frametitle{Some setup for classifying planes}

Recall:

\begin{itemize}

\item
  Each ordered pair of points in $\R^3$ determines a vector. How is this
  vector determined? How do you get its entries?
  \pause
\item
  Subtract the tail from the head.
\end{itemize}

New fact/definition:

\begin{itemize}
\item
  If $\vv$ is a vector in $\R^3$ and $\mathbf{P}$ is a plane, we say
  that $\vv$ is \emph{normal} to $\mathbf{P}$ if, for each vector $\ww$
  contained in $\mathbf{P}$, we have $\vv \dot \ww  = 0$.
\item
  A vector is \emph{contained} in $\mathbf{P}$ if both its head and its
  tail (and hence, all the point on the vector's ``body'') are in
  $\mathbf{P}$.
\end{itemize}

\end{frame}

\begin{frame}\frametitle{Warm-up for Workshop 03}

\begin{itemize}
\item
  Choose a pair of orthogonal vectors and draw them in standard
  position. Your vectors must not be multiples of $\ii$ or $\jj$, but I
  would advise you to choose vectors with $z$-entry $0$---then, you can
  get away with drawing $\R^{3}$ as a plane viewed along the positive
  $z$-axis. \emph{Hint.} Use the dot product to make sure your vectors
  really are perpendicular.
\item
  Because your vectors are orthogonal, there is a rectangle based on
  these vectors (draw two more sides). Use the cross product--sine
  formula to verify that the area of the rectangle is equal to the
  length of the cross product of your vectors.
\end{itemize}

\end{frame}

\begin{frame}\frametitle{General parallelogram}

\begin{itemize}
\item
  Start a new picture, and draw a new pair of vectors in standard
  position. Make sure that the angle between them is \emph{not} a
  multiple of $\pi/2 = \tau/4$. Use the dot product--cosine formula to
  do this.
\item
  Having ensured your angle is not a multiple of $\pi/2 = \tau/4$, you
  know that your new pair of vectors forms a nonrectangular
  parallelogram. Use trigonometry to express the area of this
  parallelogram in terms of the vector lengths.
\item
  Check that the length of the cross product yields the area of the
  parallelogram in this case also.
\end{itemize}

\end{frame}

\begin{frame}\frametitle{Wrap-up}

Together with the orientation and complementarity properties, the cross
product--sine formula \emph{uniquely determines the cross product}. This
means that, for any pair of vectors $\vv$ and $\ww$, there is only one
vector $\uu$ satifying all three. No matter how it is obtained, it must
be equal to the cross product $\vv \times \ww$.

\begin{itemize}

\item
  Use the cross product--sine formula and the orientation property of
  the cross product to \emph{derive} the rule $\jj \times \ii = -\kk$.
\end{itemize}

\end{frame}

\end{document}
