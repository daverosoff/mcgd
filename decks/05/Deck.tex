\documentclass[11pt,ignorenonframetext,xcolor={svgnames},aspectratio=169]{beamer}
\usetheme{Szeged}
\usecolortheme{crane}
\usefonttheme{structurebold}
\usefonttheme{professionalfonts}
\usepackage{amssymb,amsmath}
\usepackage{ifxetex,ifluatex}
\usepackage{fixltx2e} % provides \textsubscript
\ifxetex
  \usepackage{fontspec,xltxtra,xunicode}
  \defaultfontfeatures{Mapping=tex-text,Scale=MatchLowercase}
\else
  \ifluatex
    \usepackage{fontspec}
    \defaultfontfeatures{Mapping=tex-text,Scale=MatchLowercase}
  \else
    \usepackage[utf8]{inputenc}
  \fi
\fi
\usepackage{listings}

% Comment these out if you don't want a slide with just the
% part/section/subsection/subsubsection title:
% \AtBeginPart{
%   \let\insertpartnumber\relax
%   \let\partname\relax
%   \frame{\partpage}
% }
% \AtBeginSection{
%   \let\insertsectionnumber\relax
%   \let\sectionname\relax
%   \frame{\sectionpage}
% }
% \AtBeginSubsection{
%   \let\insertsubsectionnumber\relax
%   \let\subsectionname\relax
%   \frame{\subsectionpage}
% }

\setlength{\parindent}{0pt}
\setlength{\parskip}{6pt plus 2pt minus 1pt}
\setlength{\emergencystretch}{3em}  % prevent overfull lines
\setcounter{secnumdepth}{0}

%%% begin dwr insert
\usepackage{patchcmd}
\usepackage{tabulary}   % Support longer table cells
\usepackage{booktabs}   % Support better tables
\usepackage[sort&compress]{natbib}

\usepackage{framed}     % Allow background color for images
\definecolor{shadecolor}{named}{white}

%\usepackage{paralist}
\usepackage{xparse}
\usepackage{subfigure}
\usepackage{hyperref}
%%% end dwr insert
\usepackage{rosoff-vector-macros}
\usepackage[beamer]{cofi}
\title{Dot and cross product}
\author{Math 275 Multivariable Calculus}
\date{September 18, 2013 }


\begin{document}
\frame{\titlepage}

\section{Dot and cross product}

\begin{frame}\frametitle{Warm-up, I}

\begin{itemize}
\item
  Ready?
  \pause
\item
  OK FAST what is the dot product--cosine formula?
  \pause
\item
  $\vv \cdot \ww = \norm{\vv} \norm{\ww} \cos{\theta}$
  \pause
\item
  Here, we agree to always choose $0 \leq \theta \leq \pi$.
  \pause
\item
  Suppose $\norm{\vv} = 3$ and $\norm{\ww} = 2$. What are the maximal
  and minimal possible values of $\vv \cdot \ww$?
\end{itemize}

\end{frame}

\begin{frame}\frametitle{Warm-up, II}

\begin{itemize}
\item
  Suppose $\vv \cdot \ww = \norm{\vv} \norm{\ww}$. What can you conclude
  about the orientation of the vectors? It helps to picture them in
  standard position.
  \pause
\item
  Suppose $\vv \cdot \ww = -\norm{\vv} \norm{\ww}$. Now what can you
  conclude?
  \pause
\item
  Suppose $\vv \cdot \ww = 0$. Now what can you conclude?
\end{itemize}

\end{frame}

\begin{frame}\frametitle{Frequent uses of dot product}

\begin{itemize}

\item
  Test for \emph{orthogonality}: i.e., whether two vectors are
  perpendicular
\item
  \emph{Projection}: part 3 of
  \href{../../workshops/02/Workshop.pdf}{Workshop 02}
\end{itemize}

\end{frame}

\begin{frame}\frametitle{Cross product}

\begin{itemize}

\item
  Cross product of vectors is specific to $\mathbf{R}^3$ \ldots kind of.
\item
  It is designed to ``complete'' an independent set.
\end{itemize}

\begin{itemize}
\item
  Fundamental geometric properties:

  \begin{itemize}
  \item
    \emph{Complementarity}: $\vv \times \ww$ is orthogonal to each of
    $\vv$ and $\ww$; in other words,
    $\vv \cdot (\vv \times \ww) = 0 = \ww \cdot (\vv \times \ww)$
    \pause
  \item
    \emph{Orientation}: The ordered system
    $\{ \vv, \ww, \vv \times \ww \}$ is right-handed
    \pause
  \item
    \emph{Cross product--sine formula}:
    $\norm{\vv \times \ww} = \norm{\vv} \norm{\ww} \sin{\theta}$.
  \end{itemize}
\end{itemize}

\end{frame}

\begin{frame}\frametitle{Algebraic properties of cross product}

The cross product has the following algebraic properties, as a
consequence of its geometric ones.

\begin{itemize}
\item
  \emph{Anticommutativity}: $\vv \times \ww = -\ww \times \vv$
\item
  \emph{Nilpotence}: $\vv \times \vv = \vec{0}$
\item
  \emph{Zerodivisors}: $\vv \times \ww = \vec{0}$ iff
  $\ww = \lambda \vv$ or $\vv = \vec{0}$
\item
  \emph{Bilinearity}:

  \begin{itemize}
  \item
    $(\lambda \vv) \times \ww = \vv \times (\lambda \ww) = \lambda (\vv \times \ww)$
  \item
    $(\vec{u} + \vv) \times \ww = \vec{u} \times \ww + \vv \times \ww$
  \item
    $\vec{u} \times (\vv + \ww) = \vec{u} \times \vv + \vec{u} \times \ww$
  \end{itemize}
\end{itemize}

\end{frame}

\begin{frame}\frametitle{Computing cross products}

\begin{itemize}
\item
  There is a method based on the formula for $3 \times 3$ determinants
  outlined in the text.
\item
  I prefer to use bilinearity/distributivity, combined with the
  fundamental relations

  \begin{itemize}
  
  \item
    $\ii \times \jj = \kk$
  \item
    $\jj \times \kk = \ii$
  \item
    $\kk \times \ii = \jj$
  \end{itemize}
\end{itemize}

\begin{itemize}
\item
  With anticommutativity, these generate another three relations

  \begin{itemize}
  
  \item
    $\jj \times \ii= -\kk$
  \item
    $\kk \times \jj= -\ii$
  \item
    $\ii \times \kk= -\jj$
  \end{itemize}
\end{itemize}

\end{frame}

\end{document}
