\documentclass[12pt]{exam}
% \usepackage{pslatex}
\usepackage{graphicx}
\DeclareGraphicsExtensions{.jpg}
\usepackage{amsmath}
\usepackage{amsfonts}
\usepackage{enumerate}
\firstpageheader{}{}{}
\runningheader{\textbf{Math 251: Fall 2011}}
 {\ifcontinuation{\textbf{Problem continues}}{}}
 {\emph{Page \thepage~of \numpages}}
\runningheadrule
\runningfooter{}{\ifincomplete{continues on next page}{}}{}
\setlength{\parskip}{1.2ex}
\setlength{\parindent}{0pt}
\pagestyle{headandfoot}
\newcommand{\N}{\mathbf{N}}
\newcommand{\Z}{\mathbf{Z}}
\newcommand{\R}{\mathbf{R}}
\DeclareGraphicsExtensions{.pdf,.png,.jpg}
\begin{document}

% \printanswers
\addpoints

\noindent
\textbf{{\large Mathematics 251 \\ Exam 3}}
% \hfill Name: \underline{\hspace{0.5in}Answers\hspace{2in}}

\noindent
November 22, 2011  \hfill Name: \underline{\hspace{3in}}


\noindent
\textbf{Instructions}: This exam is closed book: you may refer to one $8.5 \times 11$ page of handwritten notes, but no electronic aids or
other printed references are permitted. \emph{Justification of all answers is required
for partial credit.} Unless
specifically directed, leave all answers in \textbf{exact form}, e.g.\
$\sqrt{3}$ instead of~1.732 and $\pi/2$ instead of~$1.57$.

Show all pertinent work. \emph{Correct answers without accompanying work will receive little or no credit.} Results from homework or from class can be cited freely. It is in your interest to display your solution in a
clear, readable fashion. Note that standards of justification are not as high as for homework.

\emph{Budget your time wisely.} It is a good idea, whenever possible, to look through the entire exam before beginning work on a particular problem.

If your work continues onto the back of another page, please indicate
this. Check and make sure you have all of the pages in the
exam; there should be \numpages, including this one. If you have a
question, please raise your hand.

Be sure to read all questions carefully and completely.

\vspace*{1.0in}

\begin{center}
\combinedgradetable[h]
\end{center}

\vspace*{0.5in}

\begin{center}
{\Large \emph{Good luck!}}
\end{center}

\newpage

\begin{questions}

\question Consider a differentiable function $z = f(x,y)$ under the polar coordinate transform $x = r \cos \theta$, $y = r \sin \theta$.

\begin{parts}

\part[12] Use the chain rule to express the partial derivatives $\partial z / \partial r$ and $\partial z / \partial \theta$ in terms of the partial derivatives $\partial z / \partial x$ and $\partial z / \partial y$.

\vspace{\stretch{2}}

\part[12] Referring to your answer above, show that
\[
|| \nabla \! f ||^2 = \left( \frac{\partial z}{\partial r} \right)^2 + \frac{1}{r^2} \left( \frac{\partial z}{\partial \theta} \right)^2.
\]
\emph{Hint.} Expand the left-hand side using the definition of the gradient as a vector and properties of dot products.

\vspace{\stretch{3}}

\end{parts}

\newpage

\question Recall that the linearization of the function $f(x,y)$ at a point $(a,b)$ is the function $L(x,y) = f(a,b) + f_x(a,b)(x-a) + f_y(a,b)(y-b)$. Here $f_x$ and $f_y$ are the partial derivatives of $f(x,y)$.

\begin{parts}

\part[8] State a condition on $f_x$ and $f_y$ that will ensure that the plane $z = L(x,y)$ is tangent to the graph of $f$. No justification is necessary.

\vspace{\stretch{1}}

\part[8] Find the gradient of the function $F(x,y,z) = x^2 + y^2 + z^2$, and evaluate this gradient at the point $(\sqrt{3}, \sqrt{3}, \sqrt{3})$.

\vspace{\stretch{2}}

\part[8] Is there a plane that is tangent to the sphere $x^2 + y^2 + z^2 = 9$ at the point $(\sqrt{3}, \sqrt{3}, \sqrt{3})$? If not, explain why. If so, find an equation for such a tangent plane. You may use the previous parts of this problem if you wish, but it is not necessary to do so.

\vspace{\stretch{2}}

\end{parts}

\newpage

\question[12] Find the critical points of the function $f(x,y) = \sin x \cos y$ in the region $0 \leq x \leq \pi$, $0 \leq y \leq \pi$ and identify the global maximum and minimum of $f$ in the region. Justify your answers for full credit. Elementary properties of the sine and cosine functions may be cited without justification.

\vspace{\stretch{1}}

\question[12] Use Lagrange multipliers to find the maximum value of $g(x,y) = x^2+2y^2$ on the circle $x^2+y^2 = 4$. For full credit, give the coordinates of all the points on the circle where this maximum value is attained.

\vspace{\stretch{2}}

\newpage

\question Evaluate the double integrals.

\begin{parts}

\part[12] ${\displaystyle \iint_D 1 \; dA}$, where $D$ is the region bounded by the curve $x = y^2-4y/5$ and the line $y = 5x$.

\vspace{\stretch{3}}

\part[12] ${\displaystyle \iint_D x^2 + 2y \; dA}$, where $D$ is the region in the first quadrant bounded by the line $y = x$ and the curve $y = x^3$. You may leave your answer as a sum of rational numbers (e.g. $7/12 - 4/59 + 12/101$ would be OK here).

\vspace{\stretch{3}}

\end{parts}

\bonusquestion[4] I need new music to listen to. Do you have a recommendation? I like electronic music (especially IDM, jungle/drum and bass, and breakbeats), hip-hop, dub and roots reggae, straight-ahead rock, and related genres. For example: Squarepusher, Plastikman, Autechre, Blackalicious, Jurassic 5, Scientist, Dennis Brown, The White Stripes. Thanks for your suggestions.

\vspace{\stretch{1}}

\end{questions}

\end{document} 