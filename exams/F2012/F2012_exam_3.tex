\documentclass[11pt]{exam}
% \usepackage{pslatex}
\usepackage{graphicx}
\DeclareGraphicsExtensions{.jpg,.png}
\usepackage{amsmath}
\usepackage{amsfonts}
\usepackage{enumerate}
\firstpageheader{}{}{}
\runningheader{\textbf{Fall 2012}}
 {\textbf{Math 251}}
 {\emph{Page \thepage~of \numpages}}
\runningheadrule
\setlength{\parskip}{1ex}
\setlength{\parindent}{0pt}
\pagestyle{head}
\newcommand{\N}{\mathbb{N}}
\newcommand{\Z}{\mathbb{Z}}
\newcommand{\R}{\mathbb{R}}
\newcommand{\dwrspace}[1]{\vspace*{\stretch{#1}}}
\renewcommand{\vec}[1]{\mathbf{#1}}
\begin{document}

% \printanswers
\addpoints

\noindent
\textbf{{\large Mathematics 251 \\ Exam 3}}
% \hfill Name: \underline{\hspace{0.5in}Answers\hspace{2in}}

\noindent
November 20, 2012  \hfill Name: \underline{\hspace{3in}}


\noindent \textbf{Instructions}: This exam is closed book. You may refer to one
double-sided page of handwritten notes, but no electronic aids or other
printed references are permitted. Justification of all answers is required
for partial credit; please \fbox{\textbf{box}} your final answers. Unless
specifically directed, leave all answers in \textbf{exact form}, e.g.
$\sqrt{3}$ instead of~$1.732$ and~$\pi/2$ instead of~$1.57$.

Show all pertinent work. \emph{Correct answers without accompanying work will
receive little or no credit.} Results from homework or from class can be cited freely. It is in your interest to display your solution in
a clear, readable fashion.

Do not write your solutions on this exam sheet. Use the provided paper or paper of your own supply. Problems should be submitted in order. Problems that stop and start with other problems in between are not acceptable.

\begin{center}
\textbf{Write your full name in the upper right corner \\ of the front side of each page that you submit.} 

\vspace{\stretch{1}}

Be sure to read all questions carefully and completely.

\vspace{\stretch{1}}

{\Large \emph{Good luck!}}
\end{center}

%Check and make sure you have all of the pages in the
%exam; there should be \numpages, including this one. If you have a
%question, please raise your hand.


\vspace{\stretch{2}}

\begin{center}
\combinedgradetable[h]
\end{center}

\vspace{\stretch{1}}

%\begin{center}

%\end{center}

\newpage

\begin{questions}
\question In this problem, consider the function~$g(x,y) = 7 - 2xy^2$ and the point~$P = (1,-1)$. 

\begin{parts}
\part[12] Give a formula for the linearization~$L(x,y)$ of~$g$ at~$P$. Describe the geometric relationship between the graphs of the two functions $L(x,y)$ and $g(x,y)$ (in complete sentences).

\part[8] Use the linearization to estimate~$g(0.9, -1.1)$.
\end{parts}

\question In this problem, let~$f(x,y) = x^3 - 3xy^2 + y^3$ and let~$c(t) = (2+t, -3+\frac{t}{3})$. \emph{Hint.} One way to do parts (c) and (d) is by using the earlier parts of the problem.

\begin{parts}
\part[8] Find the gradient of~$f$ at the point~$c(0)$ and the tangent vector~$c'(0)$.

\part[6] A bug is sitting on the graph of~$f(x,y)$, directly over the point~$c(0)$. In which direction should the bug face if it wants to climb up the surface most quickly? Give your answer as a 2-dimensional vector that points in the appropriate direction (it doesn't have to be a unit vector). 

\part[8] Find the (instantaneous) rate of change of~$f(x,y)$ at~$c(0)$ in the~$y$-direction.

%\part Find the directional derivative of~$f(x,y)$ at~$c(0)$ in the direction given by the vector~$\langle 1, 1 \rangle$. 

\part[8] Find the (instantaneous) rate of change of~$f(c(t))$ at~$t = 0$. 
\end{parts}


\question[18] Give a good definition for what it means for the point~$(a,b)$ to be a local minimum of the function~$f(x,y)$. It doesn't have to be exactly the same as the one in the text, but it should have the same extent. This means that everything that counts as a local minimum according to the text's definition should also count according to yours and vice versa.

\question[16] Dotty and Wendell are playing a thrilling, action-packed game. A move in this game is a triple~$(x,y,z)$ of nonnegative numbers whose sum is 12. A legal move is a triple for which the product~$xyz$ is greater than the corresponding product for the preceding move. If a player whose turn it is cannot find a legal move, that player loses. For example, if Wendell plays $(6,3,3)$, Dotty may play $(5,4,3)$ in response. Wendell's move is valid because $6 + 3 + 3 = 12$, while Dotty's move is valid because $5 + 4 + 3 = 12$ and $5\cdot 4\cdot 3 = 60 > 54 = 6 \cdot 3 \cdot 3$.

By using the methodology of critical points and the second derivative test, argue convincingly that~$(4,4,4)$ is a winning play in all cases. \emph{Hint.} Consider the function $z(x,y) = 12 - x - y$. 

\question[16] For each region in the plane, say whether it is closed, bounded, both closed and bounded, or neither closed nor bounded. You don't have to justify your answers on this problem. \emph{Hint.} Draw pictures of the regions.

\begin{parts}
    \part The \emph{open unit disk} $\{ (x,y) \in \R^2 \colon x^2 + y^2 < 1\}$.
    \part The \emph{infinite strip} $\{ (x,y) \in \R^2 \colon 0 \leq x \leq 1\}$.
    \part The \emph{first quadrant} $\{ (x, y) \in \R^2 \colon x > 0, \; y > 0 \}$.
    \part The \emph{parabolic sector} $\{ (x, y) \in \R^2 \colon y \geq x^2, \; y \leq \frac{1}{2} x + 5\}$.
\end{parts}

\bonusquestion[12] Find the absolute maximum and minimum values of the function 
\[
    f(x,y) = 4 x - x^2 + 6 y - y^2
\] on the unit square
\[
    \{ (x,y) \in \R^2 \colon 0 \leq x \leq 1, \; 0 \leq y \leq 1 \}.
\]
Justify your answer.
\end{questions}
\end{document}