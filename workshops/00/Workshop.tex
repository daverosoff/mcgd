\documentclass[12pt]{exam}
\usepackage{xparse}
\usepackage{graphicx}
\DeclareGraphicsExtensions{.jpg, .png}
\usepackage{fourier}
\usepackage{MnSymbol}
\usepackage{amsthm}
\usepackage{paralist,enumerate,listings}
\usepackage{siunitx}
\usepackage{hyperref}
\firstpageheader{}{}{}
\runningheader{\textbf{Fall 2013}}
 {}
 {\textbf{Math 251}}
 %{\emph{Page \thepage~of \numpages}}
\runningheadrule
\setlength{\parskip}{1ex}
\setlength{\parindent}{0pt}
\pagestyle{head}
%\newcommand{\N}{\mathbb{N}}
%\newcommand{\Z}{\mathbb{Z}}
%\newcommand{\R}{\mathbb{R}}
%\newcommand{\dwrspace}[1]{\vspace*{\stretch{#1}}}
\NewDocumentCommand\N{}{\mathbf{N}}
\NewDocumentCommand\R{}{\mathbf{R}}
\NewDocumentCommand\Z{}{\mathbf{Z}}
\NewDocumentCommand\Q{}{\mathbf{Q}}
\NewDocumentCommand\dwrspace{m}{\vspace*{\stretch{#1}}}
\theoremstyle{definition}
\newtheorem{example}{Example}
\begin{document}
\lstset{language=R}
\noindent
\textbf{{\large Math 251 \hfill Workshop 00: Coordinates revisited}}
% \hfill Name: \underline{\hspace{0.5in}Answers\hspace{2in}}

\noindent
September 6, 2013 

\noindent
Due: Monday, September 9 \hfill Name: \underline{\hspace{3in}} 

\noindent

\subsection{Coordinates revisited}

The first major topic in this course is the geometry of 3-dimensional
Euclidean space $\mathbf{R}^3$. Many people struggle with visualizing in
this space or find it difficult to translate coordinate information into
geometric information. This workshop will encourage you to rethink
coordinates in the familiar Euclidean plane in preparation for the later
discussion.

\subsection{Origin and axes}

The plane doesn't come to us with a grid of coordinate axes pre-marked
on it. That is a structure we impose from outside. In particular, there
is no rule that says coordinate axes must be perpendicular to one
another. We choose them that way for convenience.

We'll take the meaning of ``perpendicular'' for granted. The next few
sections develop the idea of coordinates using it.

Picture the plane, blank and unmarked. Its expanse is infinite in all
directions, motionless and unchanging. Pick a line---any line at all.
We'll call it \emph{the axis} for now to distinguish it from other
lines. The axis can be used to divide the plane into a family of
parallel lines, each perpendicular to the axis. This family of lines
fills up the entire plane. That is, every point of the plane is on
exactly one such line. To see this, we must appeal to a bit of Greek
geometry. Euclid tells us how to draw a perpendicular to a line through
a given point in Propositions
\href{http://aleph0.clarku.edu/~djoyce/java/elements/bookI/propI11.html}{11}
and
\href{http://aleph0.clarku.edu/~djoyce/java/elements/bookI/propI12.html}{12}
of \href{http://aleph0.clarku.edu/~djoyce/java/elements/bookI.html}{Book
I of the Elements}.

Thus, we have a set of lines, each perpendicular to our axis, that fill
up the plane. This is \emph{almost} everything we need to
\emph{coordinatize} the set of lines. Coordinatizing means assigning a
different real number---a \emph{coordinate}---to each line. We don't do
it just at random: we do it so that if two lines are chosen, the one to
the left has the smaller coordinate. This ensures that the numeric order
and the geographic order are the same.

To coordinatize the set of lines perpendicular to the axis, we need to
choose which line has coordinate 1. Once this is done, there is a unique
choice of coordinate for each line

\subsection{Coordinates} 

\end{document}